\section{连通性}
\subsection{强连通分量}
\textbf{连通} 在有向图中存在$u$到$v$的路径,则称$u$可达$v$。如果$u,v$互相可达,则$u,v$连通。

\textbf{强连通} 有向图$G$强连通指$G$中任意两个结点连通。

\textbf{强联通分量} 有向图的极大强连通子图。

\textbf{DFS 生成树}

在有向图上进行 DFS 会形成森林。DFS会形成4 种边。
\begin{enumerate}
    \item Tree Edge,树边
    \item Back Edge,返祖边,指向祖先结点的边
    \item Cross Edge,衡叉边,指向搜索过程中已经访问过的结点,但是这个结点并不是祖先节点
    \item Forward Edge,前向边,指向子孙节点的边
\end{enumerate}

如果结点$u$是某个强连通分量在搜索树中遇到的第一个结点,那么这个强连通分量的其余结点肯定是在搜索树中以$u$为根的子树中。结点$u$被称为这个强连通分量的根。

\textbf{Tarjan}

\lstinputlisting{图论/连通性/Tarjan.cpp}

\textbf{缩点}

\lstinputlisting{图论/连通性/缩点.cpp}