\section{无向图连通性}

\subsection{割点}

若对于$x\in V$,从图中删除节点$x$以及所有$x$链接的边后,$G$分裂成两个或两个以上个不相连的子图,则称$x$为$G$的\textbf{割点}

\textbf{割点判定法则}

若$x$不是搜索树的根节点,则$x$是割点当且仅当搜索树上存在一个$x$的子节点$y$满足$dfn[x]\le low[y]$.

特别的,若$x$是搜索树的根节点,则$x$是割点的条件当且仅当搜索树上存在至少两个子节点满足.

\lstinputlisting{图论/无向图连通性/割点.cpp}

\subsection{桥}
若对于$e \in E $,从图中删除边$e$之后,$G$分裂成两个不相连的子图,则称$e$为$G$的\textbf{桥}或\textbf{割边}。

\textbf{割边判定法则}

无向边$(x,y)$是桥,当且仅当搜索树上存在$x$的一个子节点$y$,满足$dfn[x]\le low[y]$.

桥一定是搜索树中的边,一个简单环中的边一定都不是桥。

\lstinputlisting{图论/无向图连通性/桥.cpp}

