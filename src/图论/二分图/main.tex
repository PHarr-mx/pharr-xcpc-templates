\section{二分图}

\subsection{二分图}

\textbf{定义}

给一张无向图,可以把点分成两个不相交的非空集合,并且在同一集合的点之间没有边相连,那么称这张无向图为一个二分图。

\textbf{二分图的判定}

一张无向图是二分图,当且仅当图中不存在奇环。

\lstinputlisting{图论/二分图/check.cpp}

\subsection{二分图最大匹配}

“任意两条边都没有公共端点”的边集合被称为一组的匹配。

在二分图中包含边数最多的一组匹配被称为二分图的最大匹配。

对于任意一组匹配 $S$($S$是一个边集),属于$S$的边叫匹配边,匹配边的端点叫匹配点。

如果在二分图中存在连接两个非匹配点的路径$path$,则称$path$是$S$的增广路。

增广路存在以下性质:
\begin{enumerate}
    \item 长度是奇数
    \item 路径上第$1,3,5,\dots$是非匹配边,第$2,4,6,\dots$是匹配边
\end{enumerate}

所以对于一组匹配,把增广路上的边的状态全部取反,得到新的边集合$S’$,则$S'$也是一组匹配,且匹配变数多一。

所以二分图的一组最大匹配$S$,当且仅当二分图中不存在$S$的增广路。

\textbf{匈牙利算法}

算法流程
\begin{enumerate}
    \item 设$S=\emptyset $
    \item 求增广路$path$,把路径上边的状态取反得到新的匹配$S'$
    \item 重复第2步直到没有增广路
\end{enumerate}

代码实现采用深搜,从$x$出发寻找增广路,并且还回溯时把状态取反。

$N$个点$M$条的二分图,复杂度$O(NM)$

\lstinputlisting{图论/二分图/luoguP3386.cpp}

\subsection{二分图的最小的点覆盖}
给一张二分图,求最小点集$S$,使得图中任意一条边都至少有一个端点属于$S$。

\textbf{König定理}

二分图最小点覆盖包含的点数等于二分图最大匹配包含的边数。

构造方法
\begin{enumerate}
    \item 求出二分图的最大匹配
    \item 从左部每个非匹配点出发,再执行一次 DFS 求增广路的过程(一定会失败),标记访问过的所有点
    \item 取左部未被标记的点、右部被标记的点,就得到了最小点覆盖。
\end{enumerate}


\subsection{二分图的最大独立集}

给一张无向图,图的独立集就是任意两点之间没有边相连的点集,包含点数最多的独立集就是图的最大独立集。

任意两点之间都有一条边相连的子图被称作无向图的团,点数最多团就是图的最大团。

对于一般的无向图,最大团、最大独立集是 NP 完全问题。

\textbf{定理}

无向图的最大团就是补图的最大独立集

\textbf{定理}

$G$是$n$个点的二分图,$G$的最大独立集大小等于$n$减去最大匹配数。

对于二分图去掉最小点覆盖,剩余的点就构成了二分图的最大独立集。

