\section{图论杂项}

\subsection{判断简单无向图图}
根据图中的每个点的度可以判断,这个图是不是一个简单无向图(没有重边和自环的无向图)

\textbf{Havel-Hakimi定理}

\begin{enumerate}
    \item 对当前数列排序,使其呈递减
    \item 从S[2]开始对其后S[1]个数字-1
    \item 一直循环直到当前序列出现负数(即不可简单图化的情况)或者当前序列全为0 (可简单图化)时退出。
\end{enumerate}
\lstinputlisting{图论/图论杂项/Havel-Hakimi定理.cpp}

\subsection{2-SAT}
SAT是是适定性(Satisfiability)问题的简称。一般形式为 k-适定性问题,简称 k-SAT。而当$k>2$时该问题为 NP 完全的。

\textbf{定义}

有 $n$ 个布尔变量 $x_1$$\sim$$x_n$,另有 $m$ 个需要满足的条件,每个条件的形式都是 「$x_i$ 为 \verb|true/false| 或 $x_j$ 为 \verb|true/false|」。比如 「$x_1$ 为真或 $x_3$ 为假」、「$x_7$ 为假或 $x_2$ 为假」。
\[
    (x_i\or x_j) \and(\neg x_i\or x_j)\cdots
\]
注意这里的或为\textbf{排斥或}

2-SAT 问题的目标是给每个变量赋值使得所有条件得到满足。

\textbf{Tarjan SCC 缩点}

把每个变量拆成两个点,$X(True)$和$X(False)$。比如现在有一个要求$X|Y=True$,则把$X(False)$到$Y(True)$连一条边,把$X(True)$到$Y(False)$连一条边。连出来的图是对称的,然后跑一遍 Tarjan,如果存在某个变量的两个点在同一个强连通分量中的情况,则无解,否则有解。

构造方案时,我们可以通过缩点后的拓扑序确定变量的值。如果变量$x$的拓扑序在$\neg x$之后,则取$x$为真,反之取$x$为假。注意 Tarjan 求得的 SCC 的编号相同与反拓扑序。
\lstinputlisting{图论/图论杂项/2SAT.cpp}