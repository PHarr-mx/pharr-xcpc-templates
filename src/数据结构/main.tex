\chapter{数据结构}

\section{并查集}

\lstinputlisting{数据结构/并查集.cpp}

\section{链式前向星}
链式前向星又名邻接表,其实现在我已经几乎不会再手写链式前向星而是采用\verb|vector|来代替
\begin{lstlisting}
vector<int> e[N];// 无边权
vector< pair<int,int> > e[N]; 有边权

e[u].push_back(v);// 加边(u,v)
e[u].push_back( { v, w } ); //加有权边 (u,v,w)
// 无向边 反过来再做一次就好

for( auto v : e[u] ){ // 遍历
}
for( auto [ v , w ] : e[u] ) { // 遍历有权边
}

\end{lstlisting}

\section{Hash}
\subsection{Hash表}
对数字的 hash
\begin{lstlisting}
for( int i = 1 ; i <= n ; i ++ ) b[i] = a[i]; // 复制数组
sort( b + 1 , b + 1 + n ) , m = unique( b + 1 , b + 1 + n ) - b;// 排序去重
for( int i = 1 ; i <= n ; i ++ )//hash
    a[i] = lower_bound( b + 1 , b + 1 + m , a[i] ) - b;
\end{lstlisting}
除此之外,如果更加复杂的 hash 全部使用\verb|unordered_map|容器
\subsection{字符串 Hash}
把字符串当做一个 p 进制数,如果哈希冲突就双哈希

如果\verb|s = "abc"|那么$H(s) = a\times p^2 + b\times p + c $
\lstinputlisting{数据结构/字符串 Hash.cpp}

\chapter{数据结构}

\section{并查集}

\lstinputlisting{数据结构/并查集.cpp}

\section{链式前向星}
链式前向星又名邻接表,其实现在我已经几乎不会再手写链式前向星而是采用\verb|vector|来代替
\begin{lstlisting}
vector<int> e[N];// 无边权
vector< pair<int,int> > e[N]; 有边权

e[u].push_back(v);// 加边(u,v)
e[u].push_back( { v, w } ); //加有权边 (u,v,w)
// 无向边 反过来再做一次就好

for( auto v : e[u] ){ // 遍历
}
for( auto [ v , w ] : e[u] ) { // 遍历有权边
}

\end{lstlisting}

\section{Hash}
\subsection{Hash表}
对数字的 hash
\begin{lstlisting}
for( int i = 1 ; i <= n ; i ++ ) b[i] = a[i]; // 复制数组
sort( b + 1 , b + 1 + n ) , m = unique( b + 1 , b + 1 + n ) - b;// 排序去重
for( int i = 1 ; i <= n ; i ++ )//hash
    a[i] = lower_bound( b + 1 , b + 1 + m , a[i] ) - b;
\end{lstlisting}
除此之外,如果更加复杂的 hash 全部使用\verb|unordered_map|容器

\chapter{数据结构}

\section{并查集}

\lstinputlisting{数据结构/并查集.cpp}

\section{链式前向星}
链式前向星又名邻接表,其实现在我已经几乎不会再手写链式前向星而是采用\verb|vector|来代替
\begin{lstlisting}
vector<int> e[N];// 无边权
vector< pair<int,int> > e[N]; 有边权

e[u].push_back(v);// 加边(u,v)
e[u].push_back( { v, w } ); //加有权边 (u,v,w)
// 无向边 反过来再做一次就好

for( auto v : e[u] ){ // 遍历
}
for( auto [ v , w ] : e[u] ) { // 遍历有权边
}

\end{lstlisting}

\section{Hash}
\subsection{Hash表}
对数字的 hash
\begin{lstlisting}
for( int i = 1 ; i <= n ; i ++ ) b[i] = a[i]; // 复制数组
sort( b + 1 , b + 1 + n ) , m = unique( b + 1 , b + 1 + n ) - b;// 排序去重
for( int i = 1 ; i <= n ; i ++ )//hash
    a[i] = lower_bound( b + 1 , b + 1 + m , a[i] ) - b;
\end{lstlisting}
除此之外,如果更加复杂的 hash 全部使用\verb|unordered_map|容器

\chapter{数据结构}

\section{并查集}

\lstinputlisting{数据结构/并查集.cpp}

\section{链式前向星}
链式前向星又名邻接表,其实现在我已经几乎不会再手写链式前向星而是采用\verb|vector|来代替
\begin{lstlisting}
vector<int> e[N];// 无边权
vector< pair<int,int> > e[N]; 有边权

e[u].push_back(v);// 加边(u,v)
e[u].push_back( { v, w } ); //加有权边 (u,v,w)
// 无向边 反过来再做一次就好

for( auto v : e[u] ){ // 遍历
}
for( auto [ v , w ] : e[u] ) { // 遍历有权边
}

\end{lstlisting}

\section{Hash}
\subsection{Hash表}
对数字的 hash
\begin{lstlisting}
for( int i = 1 ; i <= n ; i ++ ) b[i] = a[i]; // 复制数组
sort( b + 1 , b + 1 + n ) , m = unique( b + 1 , b + 1 + n ) - b;// 排序去重
for( int i = 1 ; i <= n ; i ++ )//hash
    a[i] = lower_bound( b + 1 , b + 1 + m , a[i] ) - b;
\end{lstlisting}
除此之外,如果更加复杂的 hash 全部使用\verb|unordered_map|容器

\input{数据结构/栈/main.tex}

\input{数据结构/ST表/main.tex}

\input{数据结构/树状数组/main.tex}

\section{分块}
\lstinputlisting{数据结构/分块.cpp}

\section{ODT}
\lstinputlisting{数据结构/ODT.cpp}

\input{数据结构/线段树/main.tex}

\section{差分}
\subection{离散化差分}

\lstinputlisting{数据结构/离散化差分.cpp}

\subsection{二维前缀和、差分}
\lstinputlisting{数据结构/二维差分.cpp}

\input{数据结构/扫描线/扫描线.tex}

\input{数据结构/Splay/main.tex}


\chapter{数据结构}

\section{并查集}

\lstinputlisting{数据结构/并查集.cpp}

\section{链式前向星}
链式前向星又名邻接表,其实现在我已经几乎不会再手写链式前向星而是采用\verb|vector|来代替
\begin{lstlisting}
vector<int> e[N];// 无边权
vector< pair<int,int> > e[N]; 有边权

e[u].push_back(v);// 加边(u,v)
e[u].push_back( { v, w } ); //加有权边 (u,v,w)
// 无向边 反过来再做一次就好

for( auto v : e[u] ){ // 遍历
}
for( auto [ v , w ] : e[u] ) { // 遍历有权边
}

\end{lstlisting}

\section{Hash}
\subsection{Hash表}
对数字的 hash
\begin{lstlisting}
for( int i = 1 ; i <= n ; i ++ ) b[i] = a[i]; // 复制数组
sort( b + 1 , b + 1 + n ) , m = unique( b + 1 , b + 1 + n ) - b;// 排序去重
for( int i = 1 ; i <= n ; i ++ )//hash
    a[i] = lower_bound( b + 1 , b + 1 + m , a[i] ) - b;
\end{lstlisting}
除此之外,如果更加复杂的 hash 全部使用\verb|unordered_map|容器

\input{数据结构/栈/main.tex}

\input{数据结构/ST表/main.tex}

\input{数据结构/树状数组/main.tex}

\section{分块}
\lstinputlisting{数据结构/分块.cpp}

\section{ODT}
\lstinputlisting{数据结构/ODT.cpp}

\input{数据结构/线段树/main.tex}

\section{差分}
\subection{离散化差分}

\lstinputlisting{数据结构/离散化差分.cpp}

\subsection{二维前缀和、差分}
\lstinputlisting{数据结构/二维差分.cpp}

\input{数据结构/扫描线/扫描线.tex}

\input{数据结构/Splay/main.tex}


\chapter{数据结构}

\section{并查集}

\lstinputlisting{数据结构/并查集.cpp}

\section{链式前向星}
链式前向星又名邻接表,其实现在我已经几乎不会再手写链式前向星而是采用\verb|vector|来代替
\begin{lstlisting}
vector<int> e[N];// 无边权
vector< pair<int,int> > e[N]; 有边权

e[u].push_back(v);// 加边(u,v)
e[u].push_back( { v, w } ); //加有权边 (u,v,w)
// 无向边 反过来再做一次就好

for( auto v : e[u] ){ // 遍历
}
for( auto [ v , w ] : e[u] ) { // 遍历有权边
}

\end{lstlisting}

\section{Hash}
\subsection{Hash表}
对数字的 hash
\begin{lstlisting}
for( int i = 1 ; i <= n ; i ++ ) b[i] = a[i]; // 复制数组
sort( b + 1 , b + 1 + n ) , m = unique( b + 1 , b + 1 + n ) - b;// 排序去重
for( int i = 1 ; i <= n ; i ++ )//hash
    a[i] = lower_bound( b + 1 , b + 1 + m , a[i] ) - b;
\end{lstlisting}
除此之外,如果更加复杂的 hash 全部使用\verb|unordered_map|容器

\input{数据结构/栈/main.tex}

\input{数据结构/ST表/main.tex}

\input{数据结构/树状数组/main.tex}

\section{分块}
\lstinputlisting{数据结构/分块.cpp}

\section{ODT}
\lstinputlisting{数据结构/ODT.cpp}

\input{数据结构/线段树/main.tex}

\section{差分}
\subection{离散化差分}

\lstinputlisting{数据结构/离散化差分.cpp}

\subsection{二维前缀和、差分}
\lstinputlisting{数据结构/二维差分.cpp}

\input{数据结构/扫描线/扫描线.tex}

\input{数据结构/Splay/main.tex}


\section{分块}
\lstinputlisting{数据结构/分块.cpp}

\section{ODT}
\lstinputlisting{数据结构/ODT.cpp}

\chapter{数据结构}

\section{并查集}

\lstinputlisting{数据结构/并查集.cpp}

\section{链式前向星}
链式前向星又名邻接表,其实现在我已经几乎不会再手写链式前向星而是采用\verb|vector|来代替
\begin{lstlisting}
vector<int> e[N];// 无边权
vector< pair<int,int> > e[N]; 有边权

e[u].push_back(v);// 加边(u,v)
e[u].push_back( { v, w } ); //加有权边 (u,v,w)
// 无向边 反过来再做一次就好

for( auto v : e[u] ){ // 遍历
}
for( auto [ v , w ] : e[u] ) { // 遍历有权边
}

\end{lstlisting}

\section{Hash}
\subsection{Hash表}
对数字的 hash
\begin{lstlisting}
for( int i = 1 ; i <= n ; i ++ ) b[i] = a[i]; // 复制数组
sort( b + 1 , b + 1 + n ) , m = unique( b + 1 , b + 1 + n ) - b;// 排序去重
for( int i = 1 ; i <= n ; i ++ )//hash
    a[i] = lower_bound( b + 1 , b + 1 + m , a[i] ) - b;
\end{lstlisting}
除此之外,如果更加复杂的 hash 全部使用\verb|unordered_map|容器

\input{数据结构/栈/main.tex}

\input{数据结构/ST表/main.tex}

\input{数据结构/树状数组/main.tex}

\section{分块}
\lstinputlisting{数据结构/分块.cpp}

\section{ODT}
\lstinputlisting{数据结构/ODT.cpp}

\input{数据结构/线段树/main.tex}

\section{差分}
\subection{离散化差分}

\lstinputlisting{数据结构/离散化差分.cpp}

\subsection{二维前缀和、差分}
\lstinputlisting{数据结构/二维差分.cpp}

\input{数据结构/扫描线/扫描线.tex}

\input{数据结构/Splay/main.tex}


\section{差分}
\subection{离散化差分}

\lstinputlisting{数据结构/离散化差分.cpp}

\subsection{二维前缀和、差分}
\lstinputlisting{数据结构/二维差分.cpp}

\section{扫描线}

\subsection{求面积并}
求 $N$ 个矩形面积的并,每个矩形用$(x_a, y_a),(x_b, y_b)$表示。

将整个图形分为$2N$部分,这样的话每个矩形可以用两条线段$(x_a,y_a,y_b,1),(x_b,y_a,y_b,-1)$表示。

在需要用到扫描线的题目中$y$值通常很大,甚至可能不是整数,所以我们需要进行离散化。记$val(y)$为$y$离散化之后的值,$raw(i)$为$i$的原始坐标。在离散化之后有$tot$个$y$的坐标值,分别对应为$raw(1),raw(2),raw(3),\dots,raw(tot)$,则扫描线被分为$tot-1$段,其中第$i$段为$[raw(i),raw(i+1)]$。

将线段按照$x$值排序,初始每一段都是$0$。然后遍历每个线段$(x_i,y_a,y_b,v)$,如果到$x_{i-1}$的线段覆盖的中长度为$len$,则当前矩形的面积为$(x_i - x_{i-1})\times len$。然后给$[val(y_a),val(y_b)-1]$加$v$,相当于覆盖了$[x_i,x_{i+1}]$的部分。

\lstinputlisting{数据结构/扫描线/luoguP5490.cpp}

\subsection{二维数点}

单纯的二维数点数点问题,可以只用树状数组就可以维护。

$d(x,y)$表示从$(0,0)$到$(x,y)$中点的数量,因此从左下角$(a,b)$到右上角$(c,d)$中点的数量就可以表示为$d(c,d) - d(c,b-1) - d(a-1,d) + d( a-1,b-1)$ , 这个形式就是普通的二维前缀和。我们把式子稍作变形转换为$d((c,d) - d(c,b-1)) - (d(a-1,d) - d(a-1,b-1))$ 这样的话就可以用扫描线优化掉一维。

\lstinputlisting{数据结构/扫描线/luoguP2163.cpp}


\chapter{数据结构}

\section{并查集}

\lstinputlisting{数据结构/并查集.cpp}

\section{链式前向星}
链式前向星又名邻接表,其实现在我已经几乎不会再手写链式前向星而是采用\verb|vector|来代替
\begin{lstlisting}
vector<int> e[N];// 无边权
vector< pair<int,int> > e[N]; 有边权

e[u].push_back(v);// 加边(u,v)
e[u].push_back( { v, w } ); //加有权边 (u,v,w)
// 无向边 反过来再做一次就好

for( auto v : e[u] ){ // 遍历
}
for( auto [ v , w ] : e[u] ) { // 遍历有权边
}

\end{lstlisting}

\section{Hash}
\subsection{Hash表}
对数字的 hash
\begin{lstlisting}
for( int i = 1 ; i <= n ; i ++ ) b[i] = a[i]; // 复制数组
sort( b + 1 , b + 1 + n ) , m = unique( b + 1 , b + 1 + n ) - b;// 排序去重
for( int i = 1 ; i <= n ; i ++ )//hash
    a[i] = lower_bound( b + 1 , b + 1 + m , a[i] ) - b;
\end{lstlisting}
除此之外,如果更加复杂的 hash 全部使用\verb|unordered_map|容器

\input{数据结构/栈/main.tex}

\input{数据结构/ST表/main.tex}

\input{数据结构/树状数组/main.tex}

\section{分块}
\lstinputlisting{数据结构/分块.cpp}

\section{ODT}
\lstinputlisting{数据结构/ODT.cpp}

\input{数据结构/线段树/main.tex}

\section{差分}
\subection{离散化差分}

\lstinputlisting{数据结构/离散化差分.cpp}

\subsection{二维前缀和、差分}
\lstinputlisting{数据结构/二维差分.cpp}

\input{数据结构/扫描线/扫描线.tex}

\input{数据结构/Splay/main.tex}



\chapter{数据结构}

\section{并查集}

\lstinputlisting{数据结构/并查集.cpp}

\section{链式前向星}
链式前向星又名邻接表,其实现在我已经几乎不会再手写链式前向星而是采用\verb|vector|来代替
\begin{lstlisting}
vector<int> e[N];// 无边权
vector< pair<int,int> > e[N]; 有边权

e[u].push_back(v);// 加边(u,v)
e[u].push_back( { v, w } ); //加有权边 (u,v,w)
// 无向边 反过来再做一次就好

for( auto v : e[u] ){ // 遍历
}
for( auto [ v , w ] : e[u] ) { // 遍历有权边
}

\end{lstlisting}

\section{Hash}
\subsection{Hash表}
对数字的 hash
\begin{lstlisting}
for( int i = 1 ; i <= n ; i ++ ) b[i] = a[i]; // 复制数组
sort( b + 1 , b + 1 + n ) , m = unique( b + 1 , b + 1 + n ) - b;// 排序去重
for( int i = 1 ; i <= n ; i ++ )//hash
    a[i] = lower_bound( b + 1 , b + 1 + m , a[i] ) - b;
\end{lstlisting}
除此之外,如果更加复杂的 hash 全部使用\verb|unordered_map|容器

\chapter{数据结构}

\section{并查集}

\lstinputlisting{数据结构/并查集.cpp}

\section{链式前向星}
链式前向星又名邻接表,其实现在我已经几乎不会再手写链式前向星而是采用\verb|vector|来代替
\begin{lstlisting}
vector<int> e[N];// 无边权
vector< pair<int,int> > e[N]; 有边权

e[u].push_back(v);// 加边(u,v)
e[u].push_back( { v, w } ); //加有权边 (u,v,w)
// 无向边 反过来再做一次就好

for( auto v : e[u] ){ // 遍历
}
for( auto [ v , w ] : e[u] ) { // 遍历有权边
}

\end{lstlisting}

\section{Hash}
\subsection{Hash表}
对数字的 hash
\begin{lstlisting}
for( int i = 1 ; i <= n ; i ++ ) b[i] = a[i]; // 复制数组
sort( b + 1 , b + 1 + n ) , m = unique( b + 1 , b + 1 + n ) - b;// 排序去重
for( int i = 1 ; i <= n ; i ++ )//hash
    a[i] = lower_bound( b + 1 , b + 1 + m , a[i] ) - b;
\end{lstlisting}
除此之外,如果更加复杂的 hash 全部使用\verb|unordered_map|容器

\input{数据结构/栈/main.tex}

\input{数据结构/ST表/main.tex}

\input{数据结构/树状数组/main.tex}

\section{分块}
\lstinputlisting{数据结构/分块.cpp}

\section{ODT}
\lstinputlisting{数据结构/ODT.cpp}

\input{数据结构/线段树/main.tex}

\section{差分}
\subection{离散化差分}

\lstinputlisting{数据结构/离散化差分.cpp}

\subsection{二维前缀和、差分}
\lstinputlisting{数据结构/二维差分.cpp}

\input{数据结构/扫描线/扫描线.tex}

\input{数据结构/Splay/main.tex}


\chapter{数据结构}

\section{并查集}

\lstinputlisting{数据结构/并查集.cpp}

\section{链式前向星}
链式前向星又名邻接表,其实现在我已经几乎不会再手写链式前向星而是采用\verb|vector|来代替
\begin{lstlisting}
vector<int> e[N];// 无边权
vector< pair<int,int> > e[N]; 有边权

e[u].push_back(v);// 加边(u,v)
e[u].push_back( { v, w } ); //加有权边 (u,v,w)
// 无向边 反过来再做一次就好

for( auto v : e[u] ){ // 遍历
}
for( auto [ v , w ] : e[u] ) { // 遍历有权边
}

\end{lstlisting}

\section{Hash}
\subsection{Hash表}
对数字的 hash
\begin{lstlisting}
for( int i = 1 ; i <= n ; i ++ ) b[i] = a[i]; // 复制数组
sort( b + 1 , b + 1 + n ) , m = unique( b + 1 , b + 1 + n ) - b;// 排序去重
for( int i = 1 ; i <= n ; i ++ )//hash
    a[i] = lower_bound( b + 1 , b + 1 + m , a[i] ) - b;
\end{lstlisting}
除此之外,如果更加复杂的 hash 全部使用\verb|unordered_map|容器

\input{数据结构/栈/main.tex}

\input{数据结构/ST表/main.tex}

\input{数据结构/树状数组/main.tex}

\section{分块}
\lstinputlisting{数据结构/分块.cpp}

\section{ODT}
\lstinputlisting{数据结构/ODT.cpp}

\input{数据结构/线段树/main.tex}

\section{差分}
\subection{离散化差分}

\lstinputlisting{数据结构/离散化差分.cpp}

\subsection{二维前缀和、差分}
\lstinputlisting{数据结构/二维差分.cpp}

\input{数据结构/扫描线/扫描线.tex}

\input{数据结构/Splay/main.tex}


\chapter{数据结构}

\section{并查集}

\lstinputlisting{数据结构/并查集.cpp}

\section{链式前向星}
链式前向星又名邻接表,其实现在我已经几乎不会再手写链式前向星而是采用\verb|vector|来代替
\begin{lstlisting}
vector<int> e[N];// 无边权
vector< pair<int,int> > e[N]; 有边权

e[u].push_back(v);// 加边(u,v)
e[u].push_back( { v, w } ); //加有权边 (u,v,w)
// 无向边 反过来再做一次就好

for( auto v : e[u] ){ // 遍历
}
for( auto [ v , w ] : e[u] ) { // 遍历有权边
}

\end{lstlisting}

\section{Hash}
\subsection{Hash表}
对数字的 hash
\begin{lstlisting}
for( int i = 1 ; i <= n ; i ++ ) b[i] = a[i]; // 复制数组
sort( b + 1 , b + 1 + n ) , m = unique( b + 1 , b + 1 + n ) - b;// 排序去重
for( int i = 1 ; i <= n ; i ++ )//hash
    a[i] = lower_bound( b + 1 , b + 1 + m , a[i] ) - b;
\end{lstlisting}
除此之外,如果更加复杂的 hash 全部使用\verb|unordered_map|容器

\input{数据结构/栈/main.tex}

\input{数据结构/ST表/main.tex}

\input{数据结构/树状数组/main.tex}

\section{分块}
\lstinputlisting{数据结构/分块.cpp}

\section{ODT}
\lstinputlisting{数据结构/ODT.cpp}

\input{数据结构/线段树/main.tex}

\section{差分}
\subection{离散化差分}

\lstinputlisting{数据结构/离散化差分.cpp}

\subsection{二维前缀和、差分}
\lstinputlisting{数据结构/二维差分.cpp}

\input{数据结构/扫描线/扫描线.tex}

\input{数据结构/Splay/main.tex}


\section{分块}
\lstinputlisting{数据结构/分块.cpp}

\section{ODT}
\lstinputlisting{数据结构/ODT.cpp}

\chapter{数据结构}

\section{并查集}

\lstinputlisting{数据结构/并查集.cpp}

\section{链式前向星}
链式前向星又名邻接表,其实现在我已经几乎不会再手写链式前向星而是采用\verb|vector|来代替
\begin{lstlisting}
vector<int> e[N];// 无边权
vector< pair<int,int> > e[N]; 有边权

e[u].push_back(v);// 加边(u,v)
e[u].push_back( { v, w } ); //加有权边 (u,v,w)
// 无向边 反过来再做一次就好

for( auto v : e[u] ){ // 遍历
}
for( auto [ v , w ] : e[u] ) { // 遍历有权边
}

\end{lstlisting}

\section{Hash}
\subsection{Hash表}
对数字的 hash
\begin{lstlisting}
for( int i = 1 ; i <= n ; i ++ ) b[i] = a[i]; // 复制数组
sort( b + 1 , b + 1 + n ) , m = unique( b + 1 , b + 1 + n ) - b;// 排序去重
for( int i = 1 ; i <= n ; i ++ )//hash
    a[i] = lower_bound( b + 1 , b + 1 + m , a[i] ) - b;
\end{lstlisting}
除此之外,如果更加复杂的 hash 全部使用\verb|unordered_map|容器

\input{数据结构/栈/main.tex}

\input{数据结构/ST表/main.tex}

\input{数据结构/树状数组/main.tex}

\section{分块}
\lstinputlisting{数据结构/分块.cpp}

\section{ODT}
\lstinputlisting{数据结构/ODT.cpp}

\input{数据结构/线段树/main.tex}

\section{差分}
\subection{离散化差分}

\lstinputlisting{数据结构/离散化差分.cpp}

\subsection{二维前缀和、差分}
\lstinputlisting{数据结构/二维差分.cpp}

\input{数据结构/扫描线/扫描线.tex}

\input{数据结构/Splay/main.tex}


\section{差分}
\subection{离散化差分}

\lstinputlisting{数据结构/离散化差分.cpp}

\subsection{二维前缀和、差分}
\lstinputlisting{数据结构/二维差分.cpp}

\section{扫描线}

\subsection{求面积并}
求 $N$ 个矩形面积的并,每个矩形用$(x_a, y_a),(x_b, y_b)$表示。

将整个图形分为$2N$部分,这样的话每个矩形可以用两条线段$(x_a,y_a,y_b,1),(x_b,y_a,y_b,-1)$表示。

在需要用到扫描线的题目中$y$值通常很大,甚至可能不是整数,所以我们需要进行离散化。记$val(y)$为$y$离散化之后的值,$raw(i)$为$i$的原始坐标。在离散化之后有$tot$个$y$的坐标值,分别对应为$raw(1),raw(2),raw(3),\dots,raw(tot)$,则扫描线被分为$tot-1$段,其中第$i$段为$[raw(i),raw(i+1)]$。

将线段按照$x$值排序,初始每一段都是$0$。然后遍历每个线段$(x_i,y_a,y_b,v)$,如果到$x_{i-1}$的线段覆盖的中长度为$len$,则当前矩形的面积为$(x_i - x_{i-1})\times len$。然后给$[val(y_a),val(y_b)-1]$加$v$,相当于覆盖了$[x_i,x_{i+1}]$的部分。

\lstinputlisting{数据结构/扫描线/luoguP5490.cpp}

\subsection{二维数点}

单纯的二维数点数点问题,可以只用树状数组就可以维护。

$d(x,y)$表示从$(0,0)$到$(x,y)$中点的数量,因此从左下角$(a,b)$到右上角$(c,d)$中点的数量就可以表示为$d(c,d) - d(c,b-1) - d(a-1,d) + d( a-1,b-1)$ , 这个形式就是普通的二维前缀和。我们把式子稍作变形转换为$d((c,d) - d(c,b-1)) - (d(a-1,d) - d(a-1,b-1))$ 这样的话就可以用扫描线优化掉一维。

\lstinputlisting{数据结构/扫描线/luoguP2163.cpp}


\chapter{数据结构}

\section{并查集}

\lstinputlisting{数据结构/并查集.cpp}

\section{链式前向星}
链式前向星又名邻接表,其实现在我已经几乎不会再手写链式前向星而是采用\verb|vector|来代替
\begin{lstlisting}
vector<int> e[N];// 无边权
vector< pair<int,int> > e[N]; 有边权

e[u].push_back(v);// 加边(u,v)
e[u].push_back( { v, w } ); //加有权边 (u,v,w)
// 无向边 反过来再做一次就好

for( auto v : e[u] ){ // 遍历
}
for( auto [ v , w ] : e[u] ) { // 遍历有权边
}

\end{lstlisting}

\section{Hash}
\subsection{Hash表}
对数字的 hash
\begin{lstlisting}
for( int i = 1 ; i <= n ; i ++ ) b[i] = a[i]; // 复制数组
sort( b + 1 , b + 1 + n ) , m = unique( b + 1 , b + 1 + n ) - b;// 排序去重
for( int i = 1 ; i <= n ; i ++ )//hash
    a[i] = lower_bound( b + 1 , b + 1 + m , a[i] ) - b;
\end{lstlisting}
除此之外,如果更加复杂的 hash 全部使用\verb|unordered_map|容器

\input{数据结构/栈/main.tex}

\input{数据结构/ST表/main.tex}

\input{数据结构/树状数组/main.tex}

\section{分块}
\lstinputlisting{数据结构/分块.cpp}

\section{ODT}
\lstinputlisting{数据结构/ODT.cpp}

\input{数据结构/线段树/main.tex}

\section{差分}
\subection{离散化差分}

\lstinputlisting{数据结构/离散化差分.cpp}

\subsection{二维前缀和、差分}
\lstinputlisting{数据结构/二维差分.cpp}

\input{数据结构/扫描线/扫描线.tex}

\input{数据结构/Splay/main.tex}



\chapter{数据结构}

\section{并查集}

\lstinputlisting{数据结构/并查集.cpp}

\section{链式前向星}
链式前向星又名邻接表,其实现在我已经几乎不会再手写链式前向星而是采用\verb|vector|来代替
\begin{lstlisting}
vector<int> e[N];// 无边权
vector< pair<int,int> > e[N]; 有边权

e[u].push_back(v);// 加边(u,v)
e[u].push_back( { v, w } ); //加有权边 (u,v,w)
// 无向边 反过来再做一次就好

for( auto v : e[u] ){ // 遍历
}
for( auto [ v , w ] : e[u] ) { // 遍历有权边
}

\end{lstlisting}

\section{Hash}
\subsection{Hash表}
对数字的 hash
\begin{lstlisting}
for( int i = 1 ; i <= n ; i ++ ) b[i] = a[i]; // 复制数组
sort( b + 1 , b + 1 + n ) , m = unique( b + 1 , b + 1 + n ) - b;// 排序去重
for( int i = 1 ; i <= n ; i ++ )//hash
    a[i] = lower_bound( b + 1 , b + 1 + m , a[i] ) - b;
\end{lstlisting}
除此之外,如果更加复杂的 hash 全部使用\verb|unordered_map|容器

\chapter{数据结构}

\section{并查集}

\lstinputlisting{数据结构/并查集.cpp}

\section{链式前向星}
链式前向星又名邻接表,其实现在我已经几乎不会再手写链式前向星而是采用\verb|vector|来代替
\begin{lstlisting}
vector<int> e[N];// 无边权
vector< pair<int,int> > e[N]; 有边权

e[u].push_back(v);// 加边(u,v)
e[u].push_back( { v, w } ); //加有权边 (u,v,w)
// 无向边 反过来再做一次就好

for( auto v : e[u] ){ // 遍历
}
for( auto [ v , w ] : e[u] ) { // 遍历有权边
}

\end{lstlisting}

\section{Hash}
\subsection{Hash表}
对数字的 hash
\begin{lstlisting}
for( int i = 1 ; i <= n ; i ++ ) b[i] = a[i]; // 复制数组
sort( b + 1 , b + 1 + n ) , m = unique( b + 1 , b + 1 + n ) - b;// 排序去重
for( int i = 1 ; i <= n ; i ++ )//hash
    a[i] = lower_bound( b + 1 , b + 1 + m , a[i] ) - b;
\end{lstlisting}
除此之外,如果更加复杂的 hash 全部使用\verb|unordered_map|容器

\input{数据结构/栈/main.tex}

\input{数据结构/ST表/main.tex}

\input{数据结构/树状数组/main.tex}

\section{分块}
\lstinputlisting{数据结构/分块.cpp}

\section{ODT}
\lstinputlisting{数据结构/ODT.cpp}

\input{数据结构/线段树/main.tex}

\section{差分}
\subection{离散化差分}

\lstinputlisting{数据结构/离散化差分.cpp}

\subsection{二维前缀和、差分}
\lstinputlisting{数据结构/二维差分.cpp}

\input{数据结构/扫描线/扫描线.tex}

\input{数据结构/Splay/main.tex}


\chapter{数据结构}

\section{并查集}

\lstinputlisting{数据结构/并查集.cpp}

\section{链式前向星}
链式前向星又名邻接表,其实现在我已经几乎不会再手写链式前向星而是采用\verb|vector|来代替
\begin{lstlisting}
vector<int> e[N];// 无边权
vector< pair<int,int> > e[N]; 有边权

e[u].push_back(v);// 加边(u,v)
e[u].push_back( { v, w } ); //加有权边 (u,v,w)
// 无向边 反过来再做一次就好

for( auto v : e[u] ){ // 遍历
}
for( auto [ v , w ] : e[u] ) { // 遍历有权边
}

\end{lstlisting}

\section{Hash}
\subsection{Hash表}
对数字的 hash
\begin{lstlisting}
for( int i = 1 ; i <= n ; i ++ ) b[i] = a[i]; // 复制数组
sort( b + 1 , b + 1 + n ) , m = unique( b + 1 , b + 1 + n ) - b;// 排序去重
for( int i = 1 ; i <= n ; i ++ )//hash
    a[i] = lower_bound( b + 1 , b + 1 + m , a[i] ) - b;
\end{lstlisting}
除此之外,如果更加复杂的 hash 全部使用\verb|unordered_map|容器

\input{数据结构/栈/main.tex}

\input{数据结构/ST表/main.tex}

\input{数据结构/树状数组/main.tex}

\section{分块}
\lstinputlisting{数据结构/分块.cpp}

\section{ODT}
\lstinputlisting{数据结构/ODT.cpp}

\input{数据结构/线段树/main.tex}

\section{差分}
\subection{离散化差分}

\lstinputlisting{数据结构/离散化差分.cpp}

\subsection{二维前缀和、差分}
\lstinputlisting{数据结构/二维差分.cpp}

\input{数据结构/扫描线/扫描线.tex}

\input{数据结构/Splay/main.tex}


\chapter{数据结构}

\section{并查集}

\lstinputlisting{数据结构/并查集.cpp}

\section{链式前向星}
链式前向星又名邻接表,其实现在我已经几乎不会再手写链式前向星而是采用\verb|vector|来代替
\begin{lstlisting}
vector<int> e[N];// 无边权
vector< pair<int,int> > e[N]; 有边权

e[u].push_back(v);// 加边(u,v)
e[u].push_back( { v, w } ); //加有权边 (u,v,w)
// 无向边 反过来再做一次就好

for( auto v : e[u] ){ // 遍历
}
for( auto [ v , w ] : e[u] ) { // 遍历有权边
}

\end{lstlisting}

\section{Hash}
\subsection{Hash表}
对数字的 hash
\begin{lstlisting}
for( int i = 1 ; i <= n ; i ++ ) b[i] = a[i]; // 复制数组
sort( b + 1 , b + 1 + n ) , m = unique( b + 1 , b + 1 + n ) - b;// 排序去重
for( int i = 1 ; i <= n ; i ++ )//hash
    a[i] = lower_bound( b + 1 , b + 1 + m , a[i] ) - b;
\end{lstlisting}
除此之外,如果更加复杂的 hash 全部使用\verb|unordered_map|容器

\input{数据结构/栈/main.tex}

\input{数据结构/ST表/main.tex}

\input{数据结构/树状数组/main.tex}

\section{分块}
\lstinputlisting{数据结构/分块.cpp}

\section{ODT}
\lstinputlisting{数据结构/ODT.cpp}

\input{数据结构/线段树/main.tex}

\section{差分}
\subection{离散化差分}

\lstinputlisting{数据结构/离散化差分.cpp}

\subsection{二维前缀和、差分}
\lstinputlisting{数据结构/二维差分.cpp}

\input{数据结构/扫描线/扫描线.tex}

\input{数据结构/Splay/main.tex}


\section{分块}
\lstinputlisting{数据结构/分块.cpp}

\section{ODT}
\lstinputlisting{数据结构/ODT.cpp}

\chapter{数据结构}

\section{并查集}

\lstinputlisting{数据结构/并查集.cpp}

\section{链式前向星}
链式前向星又名邻接表,其实现在我已经几乎不会再手写链式前向星而是采用\verb|vector|来代替
\begin{lstlisting}
vector<int> e[N];// 无边权
vector< pair<int,int> > e[N]; 有边权

e[u].push_back(v);// 加边(u,v)
e[u].push_back( { v, w } ); //加有权边 (u,v,w)
// 无向边 反过来再做一次就好

for( auto v : e[u] ){ // 遍历
}
for( auto [ v , w ] : e[u] ) { // 遍历有权边
}

\end{lstlisting}

\section{Hash}
\subsection{Hash表}
对数字的 hash
\begin{lstlisting}
for( int i = 1 ; i <= n ; i ++ ) b[i] = a[i]; // 复制数组
sort( b + 1 , b + 1 + n ) , m = unique( b + 1 , b + 1 + n ) - b;// 排序去重
for( int i = 1 ; i <= n ; i ++ )//hash
    a[i] = lower_bound( b + 1 , b + 1 + m , a[i] ) - b;
\end{lstlisting}
除此之外,如果更加复杂的 hash 全部使用\verb|unordered_map|容器

\input{数据结构/栈/main.tex}

\input{数据结构/ST表/main.tex}

\input{数据结构/树状数组/main.tex}

\section{分块}
\lstinputlisting{数据结构/分块.cpp}

\section{ODT}
\lstinputlisting{数据结构/ODT.cpp}

\input{数据结构/线段树/main.tex}

\section{差分}
\subection{离散化差分}

\lstinputlisting{数据结构/离散化差分.cpp}

\subsection{二维前缀和、差分}
\lstinputlisting{数据结构/二维差分.cpp}

\input{数据结构/扫描线/扫描线.tex}

\input{数据结构/Splay/main.tex}


\section{差分}
\subection{离散化差分}

\lstinputlisting{数据结构/离散化差分.cpp}

\subsection{二维前缀和、差分}
\lstinputlisting{数据结构/二维差分.cpp}

\section{扫描线}

\subsection{求面积并}
求 $N$ 个矩形面积的并,每个矩形用$(x_a, y_a),(x_b, y_b)$表示。

将整个图形分为$2N$部分,这样的话每个矩形可以用两条线段$(x_a,y_a,y_b,1),(x_b,y_a,y_b,-1)$表示。

在需要用到扫描线的题目中$y$值通常很大,甚至可能不是整数,所以我们需要进行离散化。记$val(y)$为$y$离散化之后的值,$raw(i)$为$i$的原始坐标。在离散化之后有$tot$个$y$的坐标值,分别对应为$raw(1),raw(2),raw(3),\dots,raw(tot)$,则扫描线被分为$tot-1$段,其中第$i$段为$[raw(i),raw(i+1)]$。

将线段按照$x$值排序,初始每一段都是$0$。然后遍历每个线段$(x_i,y_a,y_b,v)$,如果到$x_{i-1}$的线段覆盖的中长度为$len$,则当前矩形的面积为$(x_i - x_{i-1})\times len$。然后给$[val(y_a),val(y_b)-1]$加$v$,相当于覆盖了$[x_i,x_{i+1}]$的部分。

\lstinputlisting{数据结构/扫描线/luoguP5490.cpp}

\subsection{二维数点}

单纯的二维数点数点问题,可以只用树状数组就可以维护。

$d(x,y)$表示从$(0,0)$到$(x,y)$中点的数量,因此从左下角$(a,b)$到右上角$(c,d)$中点的数量就可以表示为$d(c,d) - d(c,b-1) - d(a-1,d) + d( a-1,b-1)$ , 这个形式就是普通的二维前缀和。我们把式子稍作变形转换为$d((c,d) - d(c,b-1)) - (d(a-1,d) - d(a-1,b-1))$ 这样的话就可以用扫描线优化掉一维。

\lstinputlisting{数据结构/扫描线/luoguP2163.cpp}


\chapter{数据结构}

\section{并查集}

\lstinputlisting{数据结构/并查集.cpp}

\section{链式前向星}
链式前向星又名邻接表,其实现在我已经几乎不会再手写链式前向星而是采用\verb|vector|来代替
\begin{lstlisting}
vector<int> e[N];// 无边权
vector< pair<int,int> > e[N]; 有边权

e[u].push_back(v);// 加边(u,v)
e[u].push_back( { v, w } ); //加有权边 (u,v,w)
// 无向边 反过来再做一次就好

for( auto v : e[u] ){ // 遍历
}
for( auto [ v , w ] : e[u] ) { // 遍历有权边
}

\end{lstlisting}

\section{Hash}
\subsection{Hash表}
对数字的 hash
\begin{lstlisting}
for( int i = 1 ; i <= n ; i ++ ) b[i] = a[i]; // 复制数组
sort( b + 1 , b + 1 + n ) , m = unique( b + 1 , b + 1 + n ) - b;// 排序去重
for( int i = 1 ; i <= n ; i ++ )//hash
    a[i] = lower_bound( b + 1 , b + 1 + m , a[i] ) - b;
\end{lstlisting}
除此之外,如果更加复杂的 hash 全部使用\verb|unordered_map|容器

\input{数据结构/栈/main.tex}

\input{数据结构/ST表/main.tex}

\input{数据结构/树状数组/main.tex}

\section{分块}
\lstinputlisting{数据结构/分块.cpp}

\section{ODT}
\lstinputlisting{数据结构/ODT.cpp}

\input{数据结构/线段树/main.tex}

\section{差分}
\subection{离散化差分}

\lstinputlisting{数据结构/离散化差分.cpp}

\subsection{二维前缀和、差分}
\lstinputlisting{数据结构/二维差分.cpp}

\input{数据结构/扫描线/扫描线.tex}

\input{数据结构/Splay/main.tex}



\section{分块}
\lstinputlisting{数据结构/分块.cpp}

\section{ODT}
\lstinputlisting{数据结构/ODT.cpp}

\chapter{数据结构}

\section{并查集}

\lstinputlisting{数据结构/并查集.cpp}

\section{链式前向星}
链式前向星又名邻接表,其实现在我已经几乎不会再手写链式前向星而是采用\verb|vector|来代替
\begin{lstlisting}
vector<int> e[N];// 无边权
vector< pair<int,int> > e[N]; 有边权

e[u].push_back(v);// 加边(u,v)
e[u].push_back( { v, w } ); //加有权边 (u,v,w)
// 无向边 反过来再做一次就好

for( auto v : e[u] ){ // 遍历
}
for( auto [ v , w ] : e[u] ) { // 遍历有权边
}

\end{lstlisting}

\section{Hash}
\subsection{Hash表}
对数字的 hash
\begin{lstlisting}
for( int i = 1 ; i <= n ; i ++ ) b[i] = a[i]; // 复制数组
sort( b + 1 , b + 1 + n ) , m = unique( b + 1 , b + 1 + n ) - b;// 排序去重
for( int i = 1 ; i <= n ; i ++ )//hash
    a[i] = lower_bound( b + 1 , b + 1 + m , a[i] ) - b;
\end{lstlisting}
除此之外,如果更加复杂的 hash 全部使用\verb|unordered_map|容器

\chapter{数据结构}

\section{并查集}

\lstinputlisting{数据结构/并查集.cpp}

\section{链式前向星}
链式前向星又名邻接表,其实现在我已经几乎不会再手写链式前向星而是采用\verb|vector|来代替
\begin{lstlisting}
vector<int> e[N];// 无边权
vector< pair<int,int> > e[N]; 有边权

e[u].push_back(v);// 加边(u,v)
e[u].push_back( { v, w } ); //加有权边 (u,v,w)
// 无向边 反过来再做一次就好

for( auto v : e[u] ){ // 遍历
}
for( auto [ v , w ] : e[u] ) { // 遍历有权边
}

\end{lstlisting}

\section{Hash}
\subsection{Hash表}
对数字的 hash
\begin{lstlisting}
for( int i = 1 ; i <= n ; i ++ ) b[i] = a[i]; // 复制数组
sort( b + 1 , b + 1 + n ) , m = unique( b + 1 , b + 1 + n ) - b;// 排序去重
for( int i = 1 ; i <= n ; i ++ )//hash
    a[i] = lower_bound( b + 1 , b + 1 + m , a[i] ) - b;
\end{lstlisting}
除此之外,如果更加复杂的 hash 全部使用\verb|unordered_map|容器

\input{数据结构/栈/main.tex}

\input{数据结构/ST表/main.tex}

\input{数据结构/树状数组/main.tex}

\section{分块}
\lstinputlisting{数据结构/分块.cpp}

\section{ODT}
\lstinputlisting{数据结构/ODT.cpp}

\input{数据结构/线段树/main.tex}

\section{差分}
\subection{离散化差分}

\lstinputlisting{数据结构/离散化差分.cpp}

\subsection{二维前缀和、差分}
\lstinputlisting{数据结构/二维差分.cpp}

\input{数据结构/扫描线/扫描线.tex}

\input{数据结构/Splay/main.tex}


\chapter{数据结构}

\section{并查集}

\lstinputlisting{数据结构/并查集.cpp}

\section{链式前向星}
链式前向星又名邻接表,其实现在我已经几乎不会再手写链式前向星而是采用\verb|vector|来代替
\begin{lstlisting}
vector<int> e[N];// 无边权
vector< pair<int,int> > e[N]; 有边权

e[u].push_back(v);// 加边(u,v)
e[u].push_back( { v, w } ); //加有权边 (u,v,w)
// 无向边 反过来再做一次就好

for( auto v : e[u] ){ // 遍历
}
for( auto [ v , w ] : e[u] ) { // 遍历有权边
}

\end{lstlisting}

\section{Hash}
\subsection{Hash表}
对数字的 hash
\begin{lstlisting}
for( int i = 1 ; i <= n ; i ++ ) b[i] = a[i]; // 复制数组
sort( b + 1 , b + 1 + n ) , m = unique( b + 1 , b + 1 + n ) - b;// 排序去重
for( int i = 1 ; i <= n ; i ++ )//hash
    a[i] = lower_bound( b + 1 , b + 1 + m , a[i] ) - b;
\end{lstlisting}
除此之外,如果更加复杂的 hash 全部使用\verb|unordered_map|容器

\input{数据结构/栈/main.tex}

\input{数据结构/ST表/main.tex}

\input{数据结构/树状数组/main.tex}

\section{分块}
\lstinputlisting{数据结构/分块.cpp}

\section{ODT}
\lstinputlisting{数据结构/ODT.cpp}

\input{数据结构/线段树/main.tex}

\section{差分}
\subection{离散化差分}

\lstinputlisting{数据结构/离散化差分.cpp}

\subsection{二维前缀和、差分}
\lstinputlisting{数据结构/二维差分.cpp}

\input{数据结构/扫描线/扫描线.tex}

\input{数据结构/Splay/main.tex}


\chapter{数据结构}

\section{并查集}

\lstinputlisting{数据结构/并查集.cpp}

\section{链式前向星}
链式前向星又名邻接表,其实现在我已经几乎不会再手写链式前向星而是采用\verb|vector|来代替
\begin{lstlisting}
vector<int> e[N];// 无边权
vector< pair<int,int> > e[N]; 有边权

e[u].push_back(v);// 加边(u,v)
e[u].push_back( { v, w } ); //加有权边 (u,v,w)
// 无向边 反过来再做一次就好

for( auto v : e[u] ){ // 遍历
}
for( auto [ v , w ] : e[u] ) { // 遍历有权边
}

\end{lstlisting}

\section{Hash}
\subsection{Hash表}
对数字的 hash
\begin{lstlisting}
for( int i = 1 ; i <= n ; i ++ ) b[i] = a[i]; // 复制数组
sort( b + 1 , b + 1 + n ) , m = unique( b + 1 , b + 1 + n ) - b;// 排序去重
for( int i = 1 ; i <= n ; i ++ )//hash
    a[i] = lower_bound( b + 1 , b + 1 + m , a[i] ) - b;
\end{lstlisting}
除此之外,如果更加复杂的 hash 全部使用\verb|unordered_map|容器

\input{数据结构/栈/main.tex}

\input{数据结构/ST表/main.tex}

\input{数据结构/树状数组/main.tex}

\section{分块}
\lstinputlisting{数据结构/分块.cpp}

\section{ODT}
\lstinputlisting{数据结构/ODT.cpp}

\input{数据结构/线段树/main.tex}

\section{差分}
\subection{离散化差分}

\lstinputlisting{数据结构/离散化差分.cpp}

\subsection{二维前缀和、差分}
\lstinputlisting{数据结构/二维差分.cpp}

\input{数据结构/扫描线/扫描线.tex}

\input{数据结构/Splay/main.tex}


\section{分块}
\lstinputlisting{数据结构/分块.cpp}

\section{ODT}
\lstinputlisting{数据结构/ODT.cpp}

\chapter{数据结构}

\section{并查集}

\lstinputlisting{数据结构/并查集.cpp}

\section{链式前向星}
链式前向星又名邻接表,其实现在我已经几乎不会再手写链式前向星而是采用\verb|vector|来代替
\begin{lstlisting}
vector<int> e[N];// 无边权
vector< pair<int,int> > e[N]; 有边权

e[u].push_back(v);// 加边(u,v)
e[u].push_back( { v, w } ); //加有权边 (u,v,w)
// 无向边 反过来再做一次就好

for( auto v : e[u] ){ // 遍历
}
for( auto [ v , w ] : e[u] ) { // 遍历有权边
}

\end{lstlisting}

\section{Hash}
\subsection{Hash表}
对数字的 hash
\begin{lstlisting}
for( int i = 1 ; i <= n ; i ++ ) b[i] = a[i]; // 复制数组
sort( b + 1 , b + 1 + n ) , m = unique( b + 1 , b + 1 + n ) - b;// 排序去重
for( int i = 1 ; i <= n ; i ++ )//hash
    a[i] = lower_bound( b + 1 , b + 1 + m , a[i] ) - b;
\end{lstlisting}
除此之外,如果更加复杂的 hash 全部使用\verb|unordered_map|容器

\input{数据结构/栈/main.tex}

\input{数据结构/ST表/main.tex}

\input{数据结构/树状数组/main.tex}

\section{分块}
\lstinputlisting{数据结构/分块.cpp}

\section{ODT}
\lstinputlisting{数据结构/ODT.cpp}

\input{数据结构/线段树/main.tex}

\section{差分}
\subection{离散化差分}

\lstinputlisting{数据结构/离散化差分.cpp}

\subsection{二维前缀和、差分}
\lstinputlisting{数据结构/二维差分.cpp}

\input{数据结构/扫描线/扫描线.tex}

\input{数据结构/Splay/main.tex}


\section{差分}
\subection{离散化差分}

\lstinputlisting{数据结构/离散化差分.cpp}

\subsection{二维前缀和、差分}
\lstinputlisting{数据结构/二维差分.cpp}

\section{扫描线}

\subsection{求面积并}
求 $N$ 个矩形面积的并,每个矩形用$(x_a, y_a),(x_b, y_b)$表示。

将整个图形分为$2N$部分,这样的话每个矩形可以用两条线段$(x_a,y_a,y_b,1),(x_b,y_a,y_b,-1)$表示。

在需要用到扫描线的题目中$y$值通常很大,甚至可能不是整数,所以我们需要进行离散化。记$val(y)$为$y$离散化之后的值,$raw(i)$为$i$的原始坐标。在离散化之后有$tot$个$y$的坐标值,分别对应为$raw(1),raw(2),raw(3),\dots,raw(tot)$,则扫描线被分为$tot-1$段,其中第$i$段为$[raw(i),raw(i+1)]$。

将线段按照$x$值排序,初始每一段都是$0$。然后遍历每个线段$(x_i,y_a,y_b,v)$,如果到$x_{i-1}$的线段覆盖的中长度为$len$,则当前矩形的面积为$(x_i - x_{i-1})\times len$。然后给$[val(y_a),val(y_b)-1]$加$v$,相当于覆盖了$[x_i,x_{i+1}]$的部分。

\lstinputlisting{数据结构/扫描线/luoguP5490.cpp}

\subsection{二维数点}

单纯的二维数点数点问题,可以只用树状数组就可以维护。

$d(x,y)$表示从$(0,0)$到$(x,y)$中点的数量,因此从左下角$(a,b)$到右上角$(c,d)$中点的数量就可以表示为$d(c,d) - d(c,b-1) - d(a-1,d) + d( a-1,b-1)$ , 这个形式就是普通的二维前缀和。我们把式子稍作变形转换为$d((c,d) - d(c,b-1)) - (d(a-1,d) - d(a-1,b-1))$ 这样的话就可以用扫描线优化掉一维。

\lstinputlisting{数据结构/扫描线/luoguP2163.cpp}


\chapter{数据结构}

\section{并查集}

\lstinputlisting{数据结构/并查集.cpp}

\section{链式前向星}
链式前向星又名邻接表,其实现在我已经几乎不会再手写链式前向星而是采用\verb|vector|来代替
\begin{lstlisting}
vector<int> e[N];// 无边权
vector< pair<int,int> > e[N]; 有边权

e[u].push_back(v);// 加边(u,v)
e[u].push_back( { v, w } ); //加有权边 (u,v,w)
// 无向边 反过来再做一次就好

for( auto v : e[u] ){ // 遍历
}
for( auto [ v , w ] : e[u] ) { // 遍历有权边
}

\end{lstlisting}

\section{Hash}
\subsection{Hash表}
对数字的 hash
\begin{lstlisting}
for( int i = 1 ; i <= n ; i ++ ) b[i] = a[i]; // 复制数组
sort( b + 1 , b + 1 + n ) , m = unique( b + 1 , b + 1 + n ) - b;// 排序去重
for( int i = 1 ; i <= n ; i ++ )//hash
    a[i] = lower_bound( b + 1 , b + 1 + m , a[i] ) - b;
\end{lstlisting}
除此之外,如果更加复杂的 hash 全部使用\verb|unordered_map|容器

\input{数据结构/栈/main.tex}

\input{数据结构/ST表/main.tex}

\input{数据结构/树状数组/main.tex}

\section{分块}
\lstinputlisting{数据结构/分块.cpp}

\section{ODT}
\lstinputlisting{数据结构/ODT.cpp}

\input{数据结构/线段树/main.tex}

\section{差分}
\subection{离散化差分}

\lstinputlisting{数据结构/离散化差分.cpp}

\subsection{二维前缀和、差分}
\lstinputlisting{数据结构/二维差分.cpp}

\input{数据结构/扫描线/扫描线.tex}

\input{数据结构/Splay/main.tex}



\section{差分}
\subection{离散化差分}

\lstinputlisting{数据结构/离散化差分.cpp}

\subsection{二维前缀和、差分}
\lstinputlisting{数据结构/二维差分.cpp}

\section{扫描线}

\subsection{求面积并}
求 $N$ 个矩形面积的并,每个矩形用$(x_a, y_a),(x_b, y_b)$表示。

将整个图形分为$2N$部分,这样的话每个矩形可以用两条线段$(x_a,y_a,y_b,1),(x_b,y_a,y_b,-1)$表示。

在需要用到扫描线的题目中$y$值通常很大,甚至可能不是整数,所以我们需要进行离散化。记$val(y)$为$y$离散化之后的值,$raw(i)$为$i$的原始坐标。在离散化之后有$tot$个$y$的坐标值,分别对应为$raw(1),raw(2),raw(3),\dots,raw(tot)$,则扫描线被分为$tot-1$段,其中第$i$段为$[raw(i),raw(i+1)]$。

将线段按照$x$值排序,初始每一段都是$0$。然后遍历每个线段$(x_i,y_a,y_b,v)$,如果到$x_{i-1}$的线段覆盖的中长度为$len$,则当前矩形的面积为$(x_i - x_{i-1})\times len$。然后给$[val(y_a),val(y_b)-1]$加$v$,相当于覆盖了$[x_i,x_{i+1}]$的部分。

\lstinputlisting{数据结构/扫描线/luoguP5490.cpp}

\subsection{二维数点}

单纯的二维数点数点问题,可以只用树状数组就可以维护。

$d(x,y)$表示从$(0,0)$到$(x,y)$中点的数量,因此从左下角$(a,b)$到右上角$(c,d)$中点的数量就可以表示为$d(c,d) - d(c,b-1) - d(a-1,d) + d( a-1,b-1)$ , 这个形式就是普通的二维前缀和。我们把式子稍作变形转换为$d((c,d) - d(c,b-1)) - (d(a-1,d) - d(a-1,b-1))$ 这样的话就可以用扫描线优化掉一维。

\lstinputlisting{数据结构/扫描线/luoguP2163.cpp}


\chapter{数据结构}

\section{并查集}

\lstinputlisting{数据结构/并查集.cpp}

\section{链式前向星}
链式前向星又名邻接表,其实现在我已经几乎不会再手写链式前向星而是采用\verb|vector|来代替
\begin{lstlisting}
vector<int> e[N];// 无边权
vector< pair<int,int> > e[N]; 有边权

e[u].push_back(v);// 加边(u,v)
e[u].push_back( { v, w } ); //加有权边 (u,v,w)
// 无向边 反过来再做一次就好

for( auto v : e[u] ){ // 遍历
}
for( auto [ v , w ] : e[u] ) { // 遍历有权边
}

\end{lstlisting}

\section{Hash}
\subsection{Hash表}
对数字的 hash
\begin{lstlisting}
for( int i = 1 ; i <= n ; i ++ ) b[i] = a[i]; // 复制数组
sort( b + 1 , b + 1 + n ) , m = unique( b + 1 , b + 1 + n ) - b;// 排序去重
for( int i = 1 ; i <= n ; i ++ )//hash
    a[i] = lower_bound( b + 1 , b + 1 + m , a[i] ) - b;
\end{lstlisting}
除此之外,如果更加复杂的 hash 全部使用\verb|unordered_map|容器

\chapter{数据结构}

\section{并查集}

\lstinputlisting{数据结构/并查集.cpp}

\section{链式前向星}
链式前向星又名邻接表,其实现在我已经几乎不会再手写链式前向星而是采用\verb|vector|来代替
\begin{lstlisting}
vector<int> e[N];// 无边权
vector< pair<int,int> > e[N]; 有边权

e[u].push_back(v);// 加边(u,v)
e[u].push_back( { v, w } ); //加有权边 (u,v,w)
// 无向边 反过来再做一次就好

for( auto v : e[u] ){ // 遍历
}
for( auto [ v , w ] : e[u] ) { // 遍历有权边
}

\end{lstlisting}

\section{Hash}
\subsection{Hash表}
对数字的 hash
\begin{lstlisting}
for( int i = 1 ; i <= n ; i ++ ) b[i] = a[i]; // 复制数组
sort( b + 1 , b + 1 + n ) , m = unique( b + 1 , b + 1 + n ) - b;// 排序去重
for( int i = 1 ; i <= n ; i ++ )//hash
    a[i] = lower_bound( b + 1 , b + 1 + m , a[i] ) - b;
\end{lstlisting}
除此之外,如果更加复杂的 hash 全部使用\verb|unordered_map|容器

\input{数据结构/栈/main.tex}

\input{数据结构/ST表/main.tex}

\input{数据结构/树状数组/main.tex}

\section{分块}
\lstinputlisting{数据结构/分块.cpp}

\section{ODT}
\lstinputlisting{数据结构/ODT.cpp}

\input{数据结构/线段树/main.tex}

\section{差分}
\subection{离散化差分}

\lstinputlisting{数据结构/离散化差分.cpp}

\subsection{二维前缀和、差分}
\lstinputlisting{数据结构/二维差分.cpp}

\input{数据结构/扫描线/扫描线.tex}

\input{数据结构/Splay/main.tex}


\chapter{数据结构}

\section{并查集}

\lstinputlisting{数据结构/并查集.cpp}

\section{链式前向星}
链式前向星又名邻接表,其实现在我已经几乎不会再手写链式前向星而是采用\verb|vector|来代替
\begin{lstlisting}
vector<int> e[N];// 无边权
vector< pair<int,int> > e[N]; 有边权

e[u].push_back(v);// 加边(u,v)
e[u].push_back( { v, w } ); //加有权边 (u,v,w)
// 无向边 反过来再做一次就好

for( auto v : e[u] ){ // 遍历
}
for( auto [ v , w ] : e[u] ) { // 遍历有权边
}

\end{lstlisting}

\section{Hash}
\subsection{Hash表}
对数字的 hash
\begin{lstlisting}
for( int i = 1 ; i <= n ; i ++ ) b[i] = a[i]; // 复制数组
sort( b + 1 , b + 1 + n ) , m = unique( b + 1 , b + 1 + n ) - b;// 排序去重
for( int i = 1 ; i <= n ; i ++ )//hash
    a[i] = lower_bound( b + 1 , b + 1 + m , a[i] ) - b;
\end{lstlisting}
除此之外,如果更加复杂的 hash 全部使用\verb|unordered_map|容器

\input{数据结构/栈/main.tex}

\input{数据结构/ST表/main.tex}

\input{数据结构/树状数组/main.tex}

\section{分块}
\lstinputlisting{数据结构/分块.cpp}

\section{ODT}
\lstinputlisting{数据结构/ODT.cpp}

\input{数据结构/线段树/main.tex}

\section{差分}
\subection{离散化差分}

\lstinputlisting{数据结构/离散化差分.cpp}

\subsection{二维前缀和、差分}
\lstinputlisting{数据结构/二维差分.cpp}

\input{数据结构/扫描线/扫描线.tex}

\input{数据结构/Splay/main.tex}


\chapter{数据结构}

\section{并查集}

\lstinputlisting{数据结构/并查集.cpp}

\section{链式前向星}
链式前向星又名邻接表,其实现在我已经几乎不会再手写链式前向星而是采用\verb|vector|来代替
\begin{lstlisting}
vector<int> e[N];// 无边权
vector< pair<int,int> > e[N]; 有边权

e[u].push_back(v);// 加边(u,v)
e[u].push_back( { v, w } ); //加有权边 (u,v,w)
// 无向边 反过来再做一次就好

for( auto v : e[u] ){ // 遍历
}
for( auto [ v , w ] : e[u] ) { // 遍历有权边
}

\end{lstlisting}

\section{Hash}
\subsection{Hash表}
对数字的 hash
\begin{lstlisting}
for( int i = 1 ; i <= n ; i ++ ) b[i] = a[i]; // 复制数组
sort( b + 1 , b + 1 + n ) , m = unique( b + 1 , b + 1 + n ) - b;// 排序去重
for( int i = 1 ; i <= n ; i ++ )//hash
    a[i] = lower_bound( b + 1 , b + 1 + m , a[i] ) - b;
\end{lstlisting}
除此之外,如果更加复杂的 hash 全部使用\verb|unordered_map|容器

\input{数据结构/栈/main.tex}

\input{数据结构/ST表/main.tex}

\input{数据结构/树状数组/main.tex}

\section{分块}
\lstinputlisting{数据结构/分块.cpp}

\section{ODT}
\lstinputlisting{数据结构/ODT.cpp}

\input{数据结构/线段树/main.tex}

\section{差分}
\subection{离散化差分}

\lstinputlisting{数据结构/离散化差分.cpp}

\subsection{二维前缀和、差分}
\lstinputlisting{数据结构/二维差分.cpp}

\input{数据结构/扫描线/扫描线.tex}

\input{数据结构/Splay/main.tex}


\section{分块}
\lstinputlisting{数据结构/分块.cpp}

\section{ODT}
\lstinputlisting{数据结构/ODT.cpp}

\chapter{数据结构}

\section{并查集}

\lstinputlisting{数据结构/并查集.cpp}

\section{链式前向星}
链式前向星又名邻接表,其实现在我已经几乎不会再手写链式前向星而是采用\verb|vector|来代替
\begin{lstlisting}
vector<int> e[N];// 无边权
vector< pair<int,int> > e[N]; 有边权

e[u].push_back(v);// 加边(u,v)
e[u].push_back( { v, w } ); //加有权边 (u,v,w)
// 无向边 反过来再做一次就好

for( auto v : e[u] ){ // 遍历
}
for( auto [ v , w ] : e[u] ) { // 遍历有权边
}

\end{lstlisting}

\section{Hash}
\subsection{Hash表}
对数字的 hash
\begin{lstlisting}
for( int i = 1 ; i <= n ; i ++ ) b[i] = a[i]; // 复制数组
sort( b + 1 , b + 1 + n ) , m = unique( b + 1 , b + 1 + n ) - b;// 排序去重
for( int i = 1 ; i <= n ; i ++ )//hash
    a[i] = lower_bound( b + 1 , b + 1 + m , a[i] ) - b;
\end{lstlisting}
除此之外,如果更加复杂的 hash 全部使用\verb|unordered_map|容器

\input{数据结构/栈/main.tex}

\input{数据结构/ST表/main.tex}

\input{数据结构/树状数组/main.tex}

\section{分块}
\lstinputlisting{数据结构/分块.cpp}

\section{ODT}
\lstinputlisting{数据结构/ODT.cpp}

\input{数据结构/线段树/main.tex}

\section{差分}
\subection{离散化差分}

\lstinputlisting{数据结构/离散化差分.cpp}

\subsection{二维前缀和、差分}
\lstinputlisting{数据结构/二维差分.cpp}

\input{数据结构/扫描线/扫描线.tex}

\input{数据结构/Splay/main.tex}


\section{差分}
\subection{离散化差分}

\lstinputlisting{数据结构/离散化差分.cpp}

\subsection{二维前缀和、差分}
\lstinputlisting{数据结构/二维差分.cpp}

\section{扫描线}

\subsection{求面积并}
求 $N$ 个矩形面积的并,每个矩形用$(x_a, y_a),(x_b, y_b)$表示。

将整个图形分为$2N$部分,这样的话每个矩形可以用两条线段$(x_a,y_a,y_b,1),(x_b,y_a,y_b,-1)$表示。

在需要用到扫描线的题目中$y$值通常很大,甚至可能不是整数,所以我们需要进行离散化。记$val(y)$为$y$离散化之后的值,$raw(i)$为$i$的原始坐标。在离散化之后有$tot$个$y$的坐标值,分别对应为$raw(1),raw(2),raw(3),\dots,raw(tot)$,则扫描线被分为$tot-1$段,其中第$i$段为$[raw(i),raw(i+1)]$。

将线段按照$x$值排序,初始每一段都是$0$。然后遍历每个线段$(x_i,y_a,y_b,v)$,如果到$x_{i-1}$的线段覆盖的中长度为$len$,则当前矩形的面积为$(x_i - x_{i-1})\times len$。然后给$[val(y_a),val(y_b)-1]$加$v$,相当于覆盖了$[x_i,x_{i+1}]$的部分。

\lstinputlisting{数据结构/扫描线/luoguP5490.cpp}

\subsection{二维数点}

单纯的二维数点数点问题,可以只用树状数组就可以维护。

$d(x,y)$表示从$(0,0)$到$(x,y)$中点的数量,因此从左下角$(a,b)$到右上角$(c,d)$中点的数量就可以表示为$d(c,d) - d(c,b-1) - d(a-1,d) + d( a-1,b-1)$ , 这个形式就是普通的二维前缀和。我们把式子稍作变形转换为$d((c,d) - d(c,b-1)) - (d(a-1,d) - d(a-1,b-1))$ 这样的话就可以用扫描线优化掉一维。

\lstinputlisting{数据结构/扫描线/luoguP2163.cpp}


\chapter{数据结构}

\section{并查集}

\lstinputlisting{数据结构/并查集.cpp}

\section{链式前向星}
链式前向星又名邻接表,其实现在我已经几乎不会再手写链式前向星而是采用\verb|vector|来代替
\begin{lstlisting}
vector<int> e[N];// 无边权
vector< pair<int,int> > e[N]; 有边权

e[u].push_back(v);// 加边(u,v)
e[u].push_back( { v, w } ); //加有权边 (u,v,w)
// 无向边 反过来再做一次就好

for( auto v : e[u] ){ // 遍历
}
for( auto [ v , w ] : e[u] ) { // 遍历有权边
}

\end{lstlisting}

\section{Hash}
\subsection{Hash表}
对数字的 hash
\begin{lstlisting}
for( int i = 1 ; i <= n ; i ++ ) b[i] = a[i]; // 复制数组
sort( b + 1 , b + 1 + n ) , m = unique( b + 1 , b + 1 + n ) - b;// 排序去重
for( int i = 1 ; i <= n ; i ++ )//hash
    a[i] = lower_bound( b + 1 , b + 1 + m , a[i] ) - b;
\end{lstlisting}
除此之外,如果更加复杂的 hash 全部使用\verb|unordered_map|容器

\input{数据结构/栈/main.tex}

\input{数据结构/ST表/main.tex}

\input{数据结构/树状数组/main.tex}

\section{分块}
\lstinputlisting{数据结构/分块.cpp}

\section{ODT}
\lstinputlisting{数据结构/ODT.cpp}

\input{数据结构/线段树/main.tex}

\section{差分}
\subection{离散化差分}

\lstinputlisting{数据结构/离散化差分.cpp}

\subsection{二维前缀和、差分}
\lstinputlisting{数据结构/二维差分.cpp}

\input{数据结构/扫描线/扫描线.tex}

\input{数据结构/Splay/main.tex}




\chapter{数据结构}

\section{并查集}

\lstinputlisting{数据结构/并查集.cpp}

\section{链式前向星}
链式前向星又名邻接表,其实现在我已经几乎不会再手写链式前向星而是采用\verb|vector|来代替
\begin{lstlisting}
vector<int> e[N];// 无边权
vector< pair<int,int> > e[N]; 有边权

e[u].push_back(v);// 加边(u,v)
e[u].push_back( { v, w } ); //加有权边 (u,v,w)
// 无向边 反过来再做一次就好

for( auto v : e[u] ){ // 遍历
}
for( auto [ v , w ] : e[u] ) { // 遍历有权边
}

\end{lstlisting}

\section{Hash}
\subsection{Hash表}
对数字的 hash
\begin{lstlisting}
for( int i = 1 ; i <= n ; i ++ ) b[i] = a[i]; // 复制数组
sort( b + 1 , b + 1 + n ) , m = unique( b + 1 , b + 1 + n ) - b;// 排序去重
for( int i = 1 ; i <= n ; i ++ )//hash
    a[i] = lower_bound( b + 1 , b + 1 + m , a[i] ) - b;
\end{lstlisting}
除此之外,如果更加复杂的 hash 全部使用\verb|unordered_map|容器

\chapter{数据结构}

\section{并查集}

\lstinputlisting{数据结构/并查集.cpp}

\section{链式前向星}
链式前向星又名邻接表,其实现在我已经几乎不会再手写链式前向星而是采用\verb|vector|来代替
\begin{lstlisting}
vector<int> e[N];// 无边权
vector< pair<int,int> > e[N]; 有边权

e[u].push_back(v);// 加边(u,v)
e[u].push_back( { v, w } ); //加有权边 (u,v,w)
// 无向边 反过来再做一次就好

for( auto v : e[u] ){ // 遍历
}
for( auto [ v , w ] : e[u] ) { // 遍历有权边
}

\end{lstlisting}

\section{Hash}
\subsection{Hash表}
对数字的 hash
\begin{lstlisting}
for( int i = 1 ; i <= n ; i ++ ) b[i] = a[i]; // 复制数组
sort( b + 1 , b + 1 + n ) , m = unique( b + 1 , b + 1 + n ) - b;// 排序去重
for( int i = 1 ; i <= n ; i ++ )//hash
    a[i] = lower_bound( b + 1 , b + 1 + m , a[i] ) - b;
\end{lstlisting}
除此之外,如果更加复杂的 hash 全部使用\verb|unordered_map|容器

\chapter{数据结构}

\section{并查集}

\lstinputlisting{数据结构/并查集.cpp}

\section{链式前向星}
链式前向星又名邻接表,其实现在我已经几乎不会再手写链式前向星而是采用\verb|vector|来代替
\begin{lstlisting}
vector<int> e[N];// 无边权
vector< pair<int,int> > e[N]; 有边权

e[u].push_back(v);// 加边(u,v)
e[u].push_back( { v, w } ); //加有权边 (u,v,w)
// 无向边 反过来再做一次就好

for( auto v : e[u] ){ // 遍历
}
for( auto [ v , w ] : e[u] ) { // 遍历有权边
}

\end{lstlisting}

\section{Hash}
\subsection{Hash表}
对数字的 hash
\begin{lstlisting}
for( int i = 1 ; i <= n ; i ++ ) b[i] = a[i]; // 复制数组
sort( b + 1 , b + 1 + n ) , m = unique( b + 1 , b + 1 + n ) - b;// 排序去重
for( int i = 1 ; i <= n ; i ++ )//hash
    a[i] = lower_bound( b + 1 , b + 1 + m , a[i] ) - b;
\end{lstlisting}
除此之外,如果更加复杂的 hash 全部使用\verb|unordered_map|容器

\input{数据结构/栈/main.tex}

\input{数据结构/ST表/main.tex}

\input{数据结构/树状数组/main.tex}

\section{分块}
\lstinputlisting{数据结构/分块.cpp}

\section{ODT}
\lstinputlisting{数据结构/ODT.cpp}

\input{数据结构/线段树/main.tex}

\section{差分}
\subection{离散化差分}

\lstinputlisting{数据结构/离散化差分.cpp}

\subsection{二维前缀和、差分}
\lstinputlisting{数据结构/二维差分.cpp}

\input{数据结构/扫描线/扫描线.tex}

\input{数据结构/Splay/main.tex}


\chapter{数据结构}

\section{并查集}

\lstinputlisting{数据结构/并查集.cpp}

\section{链式前向星}
链式前向星又名邻接表,其实现在我已经几乎不会再手写链式前向星而是采用\verb|vector|来代替
\begin{lstlisting}
vector<int> e[N];// 无边权
vector< pair<int,int> > e[N]; 有边权

e[u].push_back(v);// 加边(u,v)
e[u].push_back( { v, w } ); //加有权边 (u,v,w)
// 无向边 反过来再做一次就好

for( auto v : e[u] ){ // 遍历
}
for( auto [ v , w ] : e[u] ) { // 遍历有权边
}

\end{lstlisting}

\section{Hash}
\subsection{Hash表}
对数字的 hash
\begin{lstlisting}
for( int i = 1 ; i <= n ; i ++ ) b[i] = a[i]; // 复制数组
sort( b + 1 , b + 1 + n ) , m = unique( b + 1 , b + 1 + n ) - b;// 排序去重
for( int i = 1 ; i <= n ; i ++ )//hash
    a[i] = lower_bound( b + 1 , b + 1 + m , a[i] ) - b;
\end{lstlisting}
除此之外,如果更加复杂的 hash 全部使用\verb|unordered_map|容器

\input{数据结构/栈/main.tex}

\input{数据结构/ST表/main.tex}

\input{数据结构/树状数组/main.tex}

\section{分块}
\lstinputlisting{数据结构/分块.cpp}

\section{ODT}
\lstinputlisting{数据结构/ODT.cpp}

\input{数据结构/线段树/main.tex}

\section{差分}
\subection{离散化差分}

\lstinputlisting{数据结构/离散化差分.cpp}

\subsection{二维前缀和、差分}
\lstinputlisting{数据结构/二维差分.cpp}

\input{数据结构/扫描线/扫描线.tex}

\input{数据结构/Splay/main.tex}


\chapter{数据结构}

\section{并查集}

\lstinputlisting{数据结构/并查集.cpp}

\section{链式前向星}
链式前向星又名邻接表,其实现在我已经几乎不会再手写链式前向星而是采用\verb|vector|来代替
\begin{lstlisting}
vector<int> e[N];// 无边权
vector< pair<int,int> > e[N]; 有边权

e[u].push_back(v);// 加边(u,v)
e[u].push_back( { v, w } ); //加有权边 (u,v,w)
// 无向边 反过来再做一次就好

for( auto v : e[u] ){ // 遍历
}
for( auto [ v , w ] : e[u] ) { // 遍历有权边
}

\end{lstlisting}

\section{Hash}
\subsection{Hash表}
对数字的 hash
\begin{lstlisting}
for( int i = 1 ; i <= n ; i ++ ) b[i] = a[i]; // 复制数组
sort( b + 1 , b + 1 + n ) , m = unique( b + 1 , b + 1 + n ) - b;// 排序去重
for( int i = 1 ; i <= n ; i ++ )//hash
    a[i] = lower_bound( b + 1 , b + 1 + m , a[i] ) - b;
\end{lstlisting}
除此之外,如果更加复杂的 hash 全部使用\verb|unordered_map|容器

\input{数据结构/栈/main.tex}

\input{数据结构/ST表/main.tex}

\input{数据结构/树状数组/main.tex}

\section{分块}
\lstinputlisting{数据结构/分块.cpp}

\section{ODT}
\lstinputlisting{数据结构/ODT.cpp}

\input{数据结构/线段树/main.tex}

\section{差分}
\subection{离散化差分}

\lstinputlisting{数据结构/离散化差分.cpp}

\subsection{二维前缀和、差分}
\lstinputlisting{数据结构/二维差分.cpp}

\input{数据结构/扫描线/扫描线.tex}

\input{数据结构/Splay/main.tex}


\section{分块}
\lstinputlisting{数据结构/分块.cpp}

\section{ODT}
\lstinputlisting{数据结构/ODT.cpp}

\chapter{数据结构}

\section{并查集}

\lstinputlisting{数据结构/并查集.cpp}

\section{链式前向星}
链式前向星又名邻接表,其实现在我已经几乎不会再手写链式前向星而是采用\verb|vector|来代替
\begin{lstlisting}
vector<int> e[N];// 无边权
vector< pair<int,int> > e[N]; 有边权

e[u].push_back(v);// 加边(u,v)
e[u].push_back( { v, w } ); //加有权边 (u,v,w)
// 无向边 反过来再做一次就好

for( auto v : e[u] ){ // 遍历
}
for( auto [ v , w ] : e[u] ) { // 遍历有权边
}

\end{lstlisting}

\section{Hash}
\subsection{Hash表}
对数字的 hash
\begin{lstlisting}
for( int i = 1 ; i <= n ; i ++ ) b[i] = a[i]; // 复制数组
sort( b + 1 , b + 1 + n ) , m = unique( b + 1 , b + 1 + n ) - b;// 排序去重
for( int i = 1 ; i <= n ; i ++ )//hash
    a[i] = lower_bound( b + 1 , b + 1 + m , a[i] ) - b;
\end{lstlisting}
除此之外,如果更加复杂的 hash 全部使用\verb|unordered_map|容器

\input{数据结构/栈/main.tex}

\input{数据结构/ST表/main.tex}

\input{数据结构/树状数组/main.tex}

\section{分块}
\lstinputlisting{数据结构/分块.cpp}

\section{ODT}
\lstinputlisting{数据结构/ODT.cpp}

\input{数据结构/线段树/main.tex}

\section{差分}
\subection{离散化差分}

\lstinputlisting{数据结构/离散化差分.cpp}

\subsection{二维前缀和、差分}
\lstinputlisting{数据结构/二维差分.cpp}

\input{数据结构/扫描线/扫描线.tex}

\input{数据结构/Splay/main.tex}


\section{差分}
\subection{离散化差分}

\lstinputlisting{数据结构/离散化差分.cpp}

\subsection{二维前缀和、差分}
\lstinputlisting{数据结构/二维差分.cpp}

\section{扫描线}

\subsection{求面积并}
求 $N$ 个矩形面积的并,每个矩形用$(x_a, y_a),(x_b, y_b)$表示。

将整个图形分为$2N$部分,这样的话每个矩形可以用两条线段$(x_a,y_a,y_b,1),(x_b,y_a,y_b,-1)$表示。

在需要用到扫描线的题目中$y$值通常很大,甚至可能不是整数,所以我们需要进行离散化。记$val(y)$为$y$离散化之后的值,$raw(i)$为$i$的原始坐标。在离散化之后有$tot$个$y$的坐标值,分别对应为$raw(1),raw(2),raw(3),\dots,raw(tot)$,则扫描线被分为$tot-1$段,其中第$i$段为$[raw(i),raw(i+1)]$。

将线段按照$x$值排序,初始每一段都是$0$。然后遍历每个线段$(x_i,y_a,y_b,v)$,如果到$x_{i-1}$的线段覆盖的中长度为$len$,则当前矩形的面积为$(x_i - x_{i-1})\times len$。然后给$[val(y_a),val(y_b)-1]$加$v$,相当于覆盖了$[x_i,x_{i+1}]$的部分。

\lstinputlisting{数据结构/扫描线/luoguP5490.cpp}

\subsection{二维数点}

单纯的二维数点数点问题,可以只用树状数组就可以维护。

$d(x,y)$表示从$(0,0)$到$(x,y)$中点的数量,因此从左下角$(a,b)$到右上角$(c,d)$中点的数量就可以表示为$d(c,d) - d(c,b-1) - d(a-1,d) + d( a-1,b-1)$ , 这个形式就是普通的二维前缀和。我们把式子稍作变形转换为$d((c,d) - d(c,b-1)) - (d(a-1,d) - d(a-1,b-1))$ 这样的话就可以用扫描线优化掉一维。

\lstinputlisting{数据结构/扫描线/luoguP2163.cpp}


\chapter{数据结构}

\section{并查集}

\lstinputlisting{数据结构/并查集.cpp}

\section{链式前向星}
链式前向星又名邻接表,其实现在我已经几乎不会再手写链式前向星而是采用\verb|vector|来代替
\begin{lstlisting}
vector<int> e[N];// 无边权
vector< pair<int,int> > e[N]; 有边权

e[u].push_back(v);// 加边(u,v)
e[u].push_back( { v, w } ); //加有权边 (u,v,w)
// 无向边 反过来再做一次就好

for( auto v : e[u] ){ // 遍历
}
for( auto [ v , w ] : e[u] ) { // 遍历有权边
}

\end{lstlisting}

\section{Hash}
\subsection{Hash表}
对数字的 hash
\begin{lstlisting}
for( int i = 1 ; i <= n ; i ++ ) b[i] = a[i]; // 复制数组
sort( b + 1 , b + 1 + n ) , m = unique( b + 1 , b + 1 + n ) - b;// 排序去重
for( int i = 1 ; i <= n ; i ++ )//hash
    a[i] = lower_bound( b + 1 , b + 1 + m , a[i] ) - b;
\end{lstlisting}
除此之外,如果更加复杂的 hash 全部使用\verb|unordered_map|容器

\input{数据结构/栈/main.tex}

\input{数据结构/ST表/main.tex}

\input{数据结构/树状数组/main.tex}

\section{分块}
\lstinputlisting{数据结构/分块.cpp}

\section{ODT}
\lstinputlisting{数据结构/ODT.cpp}

\input{数据结构/线段树/main.tex}

\section{差分}
\subection{离散化差分}

\lstinputlisting{数据结构/离散化差分.cpp}

\subsection{二维前缀和、差分}
\lstinputlisting{数据结构/二维差分.cpp}

\input{数据结构/扫描线/扫描线.tex}

\input{数据结构/Splay/main.tex}



\chapter{数据结构}

\section{并查集}

\lstinputlisting{数据结构/并查集.cpp}

\section{链式前向星}
链式前向星又名邻接表,其实现在我已经几乎不会再手写链式前向星而是采用\verb|vector|来代替
\begin{lstlisting}
vector<int> e[N];// 无边权
vector< pair<int,int> > e[N]; 有边权

e[u].push_back(v);// 加边(u,v)
e[u].push_back( { v, w } ); //加有权边 (u,v,w)
// 无向边 反过来再做一次就好

for( auto v : e[u] ){ // 遍历
}
for( auto [ v , w ] : e[u] ) { // 遍历有权边
}

\end{lstlisting}

\section{Hash}
\subsection{Hash表}
对数字的 hash
\begin{lstlisting}
for( int i = 1 ; i <= n ; i ++ ) b[i] = a[i]; // 复制数组
sort( b + 1 , b + 1 + n ) , m = unique( b + 1 , b + 1 + n ) - b;// 排序去重
for( int i = 1 ; i <= n ; i ++ )//hash
    a[i] = lower_bound( b + 1 , b + 1 + m , a[i] ) - b;
\end{lstlisting}
除此之外,如果更加复杂的 hash 全部使用\verb|unordered_map|容器

\chapter{数据结构}

\section{并查集}

\lstinputlisting{数据结构/并查集.cpp}

\section{链式前向星}
链式前向星又名邻接表,其实现在我已经几乎不会再手写链式前向星而是采用\verb|vector|来代替
\begin{lstlisting}
vector<int> e[N];// 无边权
vector< pair<int,int> > e[N]; 有边权

e[u].push_back(v);// 加边(u,v)
e[u].push_back( { v, w } ); //加有权边 (u,v,w)
// 无向边 反过来再做一次就好

for( auto v : e[u] ){ // 遍历
}
for( auto [ v , w ] : e[u] ) { // 遍历有权边
}

\end{lstlisting}

\section{Hash}
\subsection{Hash表}
对数字的 hash
\begin{lstlisting}
for( int i = 1 ; i <= n ; i ++ ) b[i] = a[i]; // 复制数组
sort( b + 1 , b + 1 + n ) , m = unique( b + 1 , b + 1 + n ) - b;// 排序去重
for( int i = 1 ; i <= n ; i ++ )//hash
    a[i] = lower_bound( b + 1 , b + 1 + m , a[i] ) - b;
\end{lstlisting}
除此之外,如果更加复杂的 hash 全部使用\verb|unordered_map|容器

\input{数据结构/栈/main.tex}

\input{数据结构/ST表/main.tex}

\input{数据结构/树状数组/main.tex}

\section{分块}
\lstinputlisting{数据结构/分块.cpp}

\section{ODT}
\lstinputlisting{数据结构/ODT.cpp}

\input{数据结构/线段树/main.tex}

\section{差分}
\subection{离散化差分}

\lstinputlisting{数据结构/离散化差分.cpp}

\subsection{二维前缀和、差分}
\lstinputlisting{数据结构/二维差分.cpp}

\input{数据结构/扫描线/扫描线.tex}

\input{数据结构/Splay/main.tex}


\chapter{数据结构}

\section{并查集}

\lstinputlisting{数据结构/并查集.cpp}

\section{链式前向星}
链式前向星又名邻接表,其实现在我已经几乎不会再手写链式前向星而是采用\verb|vector|来代替
\begin{lstlisting}
vector<int> e[N];// 无边权
vector< pair<int,int> > e[N]; 有边权

e[u].push_back(v);// 加边(u,v)
e[u].push_back( { v, w } ); //加有权边 (u,v,w)
// 无向边 反过来再做一次就好

for( auto v : e[u] ){ // 遍历
}
for( auto [ v , w ] : e[u] ) { // 遍历有权边
}

\end{lstlisting}

\section{Hash}
\subsection{Hash表}
对数字的 hash
\begin{lstlisting}
for( int i = 1 ; i <= n ; i ++ ) b[i] = a[i]; // 复制数组
sort( b + 1 , b + 1 + n ) , m = unique( b + 1 , b + 1 + n ) - b;// 排序去重
for( int i = 1 ; i <= n ; i ++ )//hash
    a[i] = lower_bound( b + 1 , b + 1 + m , a[i] ) - b;
\end{lstlisting}
除此之外,如果更加复杂的 hash 全部使用\verb|unordered_map|容器

\input{数据结构/栈/main.tex}

\input{数据结构/ST表/main.tex}

\input{数据结构/树状数组/main.tex}

\section{分块}
\lstinputlisting{数据结构/分块.cpp}

\section{ODT}
\lstinputlisting{数据结构/ODT.cpp}

\input{数据结构/线段树/main.tex}

\section{差分}
\subection{离散化差分}

\lstinputlisting{数据结构/离散化差分.cpp}

\subsection{二维前缀和、差分}
\lstinputlisting{数据结构/二维差分.cpp}

\input{数据结构/扫描线/扫描线.tex}

\input{数据结构/Splay/main.tex}


\chapter{数据结构}

\section{并查集}

\lstinputlisting{数据结构/并查集.cpp}

\section{链式前向星}
链式前向星又名邻接表,其实现在我已经几乎不会再手写链式前向星而是采用\verb|vector|来代替
\begin{lstlisting}
vector<int> e[N];// 无边权
vector< pair<int,int> > e[N]; 有边权

e[u].push_back(v);// 加边(u,v)
e[u].push_back( { v, w } ); //加有权边 (u,v,w)
// 无向边 反过来再做一次就好

for( auto v : e[u] ){ // 遍历
}
for( auto [ v , w ] : e[u] ) { // 遍历有权边
}

\end{lstlisting}

\section{Hash}
\subsection{Hash表}
对数字的 hash
\begin{lstlisting}
for( int i = 1 ; i <= n ; i ++ ) b[i] = a[i]; // 复制数组
sort( b + 1 , b + 1 + n ) , m = unique( b + 1 , b + 1 + n ) - b;// 排序去重
for( int i = 1 ; i <= n ; i ++ )//hash
    a[i] = lower_bound( b + 1 , b + 1 + m , a[i] ) - b;
\end{lstlisting}
除此之外,如果更加复杂的 hash 全部使用\verb|unordered_map|容器

\input{数据结构/栈/main.tex}

\input{数据结构/ST表/main.tex}

\input{数据结构/树状数组/main.tex}

\section{分块}
\lstinputlisting{数据结构/分块.cpp}

\section{ODT}
\lstinputlisting{数据结构/ODT.cpp}

\input{数据结构/线段树/main.tex}

\section{差分}
\subection{离散化差分}

\lstinputlisting{数据结构/离散化差分.cpp}

\subsection{二维前缀和、差分}
\lstinputlisting{数据结构/二维差分.cpp}

\input{数据结构/扫描线/扫描线.tex}

\input{数据结构/Splay/main.tex}


\section{分块}
\lstinputlisting{数据结构/分块.cpp}

\section{ODT}
\lstinputlisting{数据结构/ODT.cpp}

\chapter{数据结构}

\section{并查集}

\lstinputlisting{数据结构/并查集.cpp}

\section{链式前向星}
链式前向星又名邻接表,其实现在我已经几乎不会再手写链式前向星而是采用\verb|vector|来代替
\begin{lstlisting}
vector<int> e[N];// 无边权
vector< pair<int,int> > e[N]; 有边权

e[u].push_back(v);// 加边(u,v)
e[u].push_back( { v, w } ); //加有权边 (u,v,w)
// 无向边 反过来再做一次就好

for( auto v : e[u] ){ // 遍历
}
for( auto [ v , w ] : e[u] ) { // 遍历有权边
}

\end{lstlisting}

\section{Hash}
\subsection{Hash表}
对数字的 hash
\begin{lstlisting}
for( int i = 1 ; i <= n ; i ++ ) b[i] = a[i]; // 复制数组
sort( b + 1 , b + 1 + n ) , m = unique( b + 1 , b + 1 + n ) - b;// 排序去重
for( int i = 1 ; i <= n ; i ++ )//hash
    a[i] = lower_bound( b + 1 , b + 1 + m , a[i] ) - b;
\end{lstlisting}
除此之外,如果更加复杂的 hash 全部使用\verb|unordered_map|容器

\input{数据结构/栈/main.tex}

\input{数据结构/ST表/main.tex}

\input{数据结构/树状数组/main.tex}

\section{分块}
\lstinputlisting{数据结构/分块.cpp}

\section{ODT}
\lstinputlisting{数据结构/ODT.cpp}

\input{数据结构/线段树/main.tex}

\section{差分}
\subection{离散化差分}

\lstinputlisting{数据结构/离散化差分.cpp}

\subsection{二维前缀和、差分}
\lstinputlisting{数据结构/二维差分.cpp}

\input{数据结构/扫描线/扫描线.tex}

\input{数据结构/Splay/main.tex}


\section{差分}
\subection{离散化差分}

\lstinputlisting{数据结构/离散化差分.cpp}

\subsection{二维前缀和、差分}
\lstinputlisting{数据结构/二维差分.cpp}

\section{扫描线}

\subsection{求面积并}
求 $N$ 个矩形面积的并,每个矩形用$(x_a, y_a),(x_b, y_b)$表示。

将整个图形分为$2N$部分,这样的话每个矩形可以用两条线段$(x_a,y_a,y_b,1),(x_b,y_a,y_b,-1)$表示。

在需要用到扫描线的题目中$y$值通常很大,甚至可能不是整数,所以我们需要进行离散化。记$val(y)$为$y$离散化之后的值,$raw(i)$为$i$的原始坐标。在离散化之后有$tot$个$y$的坐标值,分别对应为$raw(1),raw(2),raw(3),\dots,raw(tot)$,则扫描线被分为$tot-1$段,其中第$i$段为$[raw(i),raw(i+1)]$。

将线段按照$x$值排序,初始每一段都是$0$。然后遍历每个线段$(x_i,y_a,y_b,v)$,如果到$x_{i-1}$的线段覆盖的中长度为$len$,则当前矩形的面积为$(x_i - x_{i-1})\times len$。然后给$[val(y_a),val(y_b)-1]$加$v$,相当于覆盖了$[x_i,x_{i+1}]$的部分。

\lstinputlisting{数据结构/扫描线/luoguP5490.cpp}

\subsection{二维数点}

单纯的二维数点数点问题,可以只用树状数组就可以维护。

$d(x,y)$表示从$(0,0)$到$(x,y)$中点的数量,因此从左下角$(a,b)$到右上角$(c,d)$中点的数量就可以表示为$d(c,d) - d(c,b-1) - d(a-1,d) + d( a-1,b-1)$ , 这个形式就是普通的二维前缀和。我们把式子稍作变形转换为$d((c,d) - d(c,b-1)) - (d(a-1,d) - d(a-1,b-1))$ 这样的话就可以用扫描线优化掉一维。

\lstinputlisting{数据结构/扫描线/luoguP2163.cpp}


\chapter{数据结构}

\section{并查集}

\lstinputlisting{数据结构/并查集.cpp}

\section{链式前向星}
链式前向星又名邻接表,其实现在我已经几乎不会再手写链式前向星而是采用\verb|vector|来代替
\begin{lstlisting}
vector<int> e[N];// 无边权
vector< pair<int,int> > e[N]; 有边权

e[u].push_back(v);// 加边(u,v)
e[u].push_back( { v, w } ); //加有权边 (u,v,w)
// 无向边 反过来再做一次就好

for( auto v : e[u] ){ // 遍历
}
for( auto [ v , w ] : e[u] ) { // 遍历有权边
}

\end{lstlisting}

\section{Hash}
\subsection{Hash表}
对数字的 hash
\begin{lstlisting}
for( int i = 1 ; i <= n ; i ++ ) b[i] = a[i]; // 复制数组
sort( b + 1 , b + 1 + n ) , m = unique( b + 1 , b + 1 + n ) - b;// 排序去重
for( int i = 1 ; i <= n ; i ++ )//hash
    a[i] = lower_bound( b + 1 , b + 1 + m , a[i] ) - b;
\end{lstlisting}
除此之外,如果更加复杂的 hash 全部使用\verb|unordered_map|容器

\input{数据结构/栈/main.tex}

\input{数据结构/ST表/main.tex}

\input{数据结构/树状数组/main.tex}

\section{分块}
\lstinputlisting{数据结构/分块.cpp}

\section{ODT}
\lstinputlisting{数据结构/ODT.cpp}

\input{数据结构/线段树/main.tex}

\section{差分}
\subection{离散化差分}

\lstinputlisting{数据结构/离散化差分.cpp}

\subsection{二维前缀和、差分}
\lstinputlisting{数据结构/二维差分.cpp}

\input{数据结构/扫描线/扫描线.tex}

\input{数据结构/Splay/main.tex}



\chapter{数据结构}

\section{并查集}

\lstinputlisting{数据结构/并查集.cpp}

\section{链式前向星}
链式前向星又名邻接表,其实现在我已经几乎不会再手写链式前向星而是采用\verb|vector|来代替
\begin{lstlisting}
vector<int> e[N];// 无边权
vector< pair<int,int> > e[N]; 有边权

e[u].push_back(v);// 加边(u,v)
e[u].push_back( { v, w } ); //加有权边 (u,v,w)
// 无向边 反过来再做一次就好

for( auto v : e[u] ){ // 遍历
}
for( auto [ v , w ] : e[u] ) { // 遍历有权边
}

\end{lstlisting}

\section{Hash}
\subsection{Hash表}
对数字的 hash
\begin{lstlisting}
for( int i = 1 ; i <= n ; i ++ ) b[i] = a[i]; // 复制数组
sort( b + 1 , b + 1 + n ) , m = unique( b + 1 , b + 1 + n ) - b;// 排序去重
for( int i = 1 ; i <= n ; i ++ )//hash
    a[i] = lower_bound( b + 1 , b + 1 + m , a[i] ) - b;
\end{lstlisting}
除此之外,如果更加复杂的 hash 全部使用\verb|unordered_map|容器

\chapter{数据结构}

\section{并查集}

\lstinputlisting{数据结构/并查集.cpp}

\section{链式前向星}
链式前向星又名邻接表,其实现在我已经几乎不会再手写链式前向星而是采用\verb|vector|来代替
\begin{lstlisting}
vector<int> e[N];// 无边权
vector< pair<int,int> > e[N]; 有边权

e[u].push_back(v);// 加边(u,v)
e[u].push_back( { v, w } ); //加有权边 (u,v,w)
// 无向边 反过来再做一次就好

for( auto v : e[u] ){ // 遍历
}
for( auto [ v , w ] : e[u] ) { // 遍历有权边
}

\end{lstlisting}

\section{Hash}
\subsection{Hash表}
对数字的 hash
\begin{lstlisting}
for( int i = 1 ; i <= n ; i ++ ) b[i] = a[i]; // 复制数组
sort( b + 1 , b + 1 + n ) , m = unique( b + 1 , b + 1 + n ) - b;// 排序去重
for( int i = 1 ; i <= n ; i ++ )//hash
    a[i] = lower_bound( b + 1 , b + 1 + m , a[i] ) - b;
\end{lstlisting}
除此之外,如果更加复杂的 hash 全部使用\verb|unordered_map|容器

\input{数据结构/栈/main.tex}

\input{数据结构/ST表/main.tex}

\input{数据结构/树状数组/main.tex}

\section{分块}
\lstinputlisting{数据结构/分块.cpp}

\section{ODT}
\lstinputlisting{数据结构/ODT.cpp}

\input{数据结构/线段树/main.tex}

\section{差分}
\subection{离散化差分}

\lstinputlisting{数据结构/离散化差分.cpp}

\subsection{二维前缀和、差分}
\lstinputlisting{数据结构/二维差分.cpp}

\input{数据结构/扫描线/扫描线.tex}

\input{数据结构/Splay/main.tex}


\chapter{数据结构}

\section{并查集}

\lstinputlisting{数据结构/并查集.cpp}

\section{链式前向星}
链式前向星又名邻接表,其实现在我已经几乎不会再手写链式前向星而是采用\verb|vector|来代替
\begin{lstlisting}
vector<int> e[N];// 无边权
vector< pair<int,int> > e[N]; 有边权

e[u].push_back(v);// 加边(u,v)
e[u].push_back( { v, w } ); //加有权边 (u,v,w)
// 无向边 反过来再做一次就好

for( auto v : e[u] ){ // 遍历
}
for( auto [ v , w ] : e[u] ) { // 遍历有权边
}

\end{lstlisting}

\section{Hash}
\subsection{Hash表}
对数字的 hash
\begin{lstlisting}
for( int i = 1 ; i <= n ; i ++ ) b[i] = a[i]; // 复制数组
sort( b + 1 , b + 1 + n ) , m = unique( b + 1 , b + 1 + n ) - b;// 排序去重
for( int i = 1 ; i <= n ; i ++ )//hash
    a[i] = lower_bound( b + 1 , b + 1 + m , a[i] ) - b;
\end{lstlisting}
除此之外,如果更加复杂的 hash 全部使用\verb|unordered_map|容器

\input{数据结构/栈/main.tex}

\input{数据结构/ST表/main.tex}

\input{数据结构/树状数组/main.tex}

\section{分块}
\lstinputlisting{数据结构/分块.cpp}

\section{ODT}
\lstinputlisting{数据结构/ODT.cpp}

\input{数据结构/线段树/main.tex}

\section{差分}
\subection{离散化差分}

\lstinputlisting{数据结构/离散化差分.cpp}

\subsection{二维前缀和、差分}
\lstinputlisting{数据结构/二维差分.cpp}

\input{数据结构/扫描线/扫描线.tex}

\input{数据结构/Splay/main.tex}


\chapter{数据结构}

\section{并查集}

\lstinputlisting{数据结构/并查集.cpp}

\section{链式前向星}
链式前向星又名邻接表,其实现在我已经几乎不会再手写链式前向星而是采用\verb|vector|来代替
\begin{lstlisting}
vector<int> e[N];// 无边权
vector< pair<int,int> > e[N]; 有边权

e[u].push_back(v);// 加边(u,v)
e[u].push_back( { v, w } ); //加有权边 (u,v,w)
// 无向边 反过来再做一次就好

for( auto v : e[u] ){ // 遍历
}
for( auto [ v , w ] : e[u] ) { // 遍历有权边
}

\end{lstlisting}

\section{Hash}
\subsection{Hash表}
对数字的 hash
\begin{lstlisting}
for( int i = 1 ; i <= n ; i ++ ) b[i] = a[i]; // 复制数组
sort( b + 1 , b + 1 + n ) , m = unique( b + 1 , b + 1 + n ) - b;// 排序去重
for( int i = 1 ; i <= n ; i ++ )//hash
    a[i] = lower_bound( b + 1 , b + 1 + m , a[i] ) - b;
\end{lstlisting}
除此之外,如果更加复杂的 hash 全部使用\verb|unordered_map|容器

\input{数据结构/栈/main.tex}

\input{数据结构/ST表/main.tex}

\input{数据结构/树状数组/main.tex}

\section{分块}
\lstinputlisting{数据结构/分块.cpp}

\section{ODT}
\lstinputlisting{数据结构/ODT.cpp}

\input{数据结构/线段树/main.tex}

\section{差分}
\subection{离散化差分}

\lstinputlisting{数据结构/离散化差分.cpp}

\subsection{二维前缀和、差分}
\lstinputlisting{数据结构/二维差分.cpp}

\input{数据结构/扫描线/扫描线.tex}

\input{数据结构/Splay/main.tex}


\section{分块}
\lstinputlisting{数据结构/分块.cpp}

\section{ODT}
\lstinputlisting{数据结构/ODT.cpp}

\chapter{数据结构}

\section{并查集}

\lstinputlisting{数据结构/并查集.cpp}

\section{链式前向星}
链式前向星又名邻接表,其实现在我已经几乎不会再手写链式前向星而是采用\verb|vector|来代替
\begin{lstlisting}
vector<int> e[N];// 无边权
vector< pair<int,int> > e[N]; 有边权

e[u].push_back(v);// 加边(u,v)
e[u].push_back( { v, w } ); //加有权边 (u,v,w)
// 无向边 反过来再做一次就好

for( auto v : e[u] ){ // 遍历
}
for( auto [ v , w ] : e[u] ) { // 遍历有权边
}

\end{lstlisting}

\section{Hash}
\subsection{Hash表}
对数字的 hash
\begin{lstlisting}
for( int i = 1 ; i <= n ; i ++ ) b[i] = a[i]; // 复制数组
sort( b + 1 , b + 1 + n ) , m = unique( b + 1 , b + 1 + n ) - b;// 排序去重
for( int i = 1 ; i <= n ; i ++ )//hash
    a[i] = lower_bound( b + 1 , b + 1 + m , a[i] ) - b;
\end{lstlisting}
除此之外,如果更加复杂的 hash 全部使用\verb|unordered_map|容器

\input{数据结构/栈/main.tex}

\input{数据结构/ST表/main.tex}

\input{数据结构/树状数组/main.tex}

\section{分块}
\lstinputlisting{数据结构/分块.cpp}

\section{ODT}
\lstinputlisting{数据结构/ODT.cpp}

\input{数据结构/线段树/main.tex}

\section{差分}
\subection{离散化差分}

\lstinputlisting{数据结构/离散化差分.cpp}

\subsection{二维前缀和、差分}
\lstinputlisting{数据结构/二维差分.cpp}

\input{数据结构/扫描线/扫描线.tex}

\input{数据结构/Splay/main.tex}


\section{差分}
\subection{离散化差分}

\lstinputlisting{数据结构/离散化差分.cpp}

\subsection{二维前缀和、差分}
\lstinputlisting{数据结构/二维差分.cpp}

\section{扫描线}

\subsection{求面积并}
求 $N$ 个矩形面积的并,每个矩形用$(x_a, y_a),(x_b, y_b)$表示。

将整个图形分为$2N$部分,这样的话每个矩形可以用两条线段$(x_a,y_a,y_b,1),(x_b,y_a,y_b,-1)$表示。

在需要用到扫描线的题目中$y$值通常很大,甚至可能不是整数,所以我们需要进行离散化。记$val(y)$为$y$离散化之后的值,$raw(i)$为$i$的原始坐标。在离散化之后有$tot$个$y$的坐标值,分别对应为$raw(1),raw(2),raw(3),\dots,raw(tot)$,则扫描线被分为$tot-1$段,其中第$i$段为$[raw(i),raw(i+1)]$。

将线段按照$x$值排序,初始每一段都是$0$。然后遍历每个线段$(x_i,y_a,y_b,v)$,如果到$x_{i-1}$的线段覆盖的中长度为$len$,则当前矩形的面积为$(x_i - x_{i-1})\times len$。然后给$[val(y_a),val(y_b)-1]$加$v$,相当于覆盖了$[x_i,x_{i+1}]$的部分。

\lstinputlisting{数据结构/扫描线/luoguP5490.cpp}

\subsection{二维数点}

单纯的二维数点数点问题,可以只用树状数组就可以维护。

$d(x,y)$表示从$(0,0)$到$(x,y)$中点的数量,因此从左下角$(a,b)$到右上角$(c,d)$中点的数量就可以表示为$d(c,d) - d(c,b-1) - d(a-1,d) + d( a-1,b-1)$ , 这个形式就是普通的二维前缀和。我们把式子稍作变形转换为$d((c,d) - d(c,b-1)) - (d(a-1,d) - d(a-1,b-1))$ 这样的话就可以用扫描线优化掉一维。

\lstinputlisting{数据结构/扫描线/luoguP2163.cpp}


\chapter{数据结构}

\section{并查集}

\lstinputlisting{数据结构/并查集.cpp}

\section{链式前向星}
链式前向星又名邻接表,其实现在我已经几乎不会再手写链式前向星而是采用\verb|vector|来代替
\begin{lstlisting}
vector<int> e[N];// 无边权
vector< pair<int,int> > e[N]; 有边权

e[u].push_back(v);// 加边(u,v)
e[u].push_back( { v, w } ); //加有权边 (u,v,w)
// 无向边 反过来再做一次就好

for( auto v : e[u] ){ // 遍历
}
for( auto [ v , w ] : e[u] ) { // 遍历有权边
}

\end{lstlisting}

\section{Hash}
\subsection{Hash表}
对数字的 hash
\begin{lstlisting}
for( int i = 1 ; i <= n ; i ++ ) b[i] = a[i]; // 复制数组
sort( b + 1 , b + 1 + n ) , m = unique( b + 1 , b + 1 + n ) - b;// 排序去重
for( int i = 1 ; i <= n ; i ++ )//hash
    a[i] = lower_bound( b + 1 , b + 1 + m , a[i] ) - b;
\end{lstlisting}
除此之外,如果更加复杂的 hash 全部使用\verb|unordered_map|容器

\input{数据结构/栈/main.tex}

\input{数据结构/ST表/main.tex}

\input{数据结构/树状数组/main.tex}

\section{分块}
\lstinputlisting{数据结构/分块.cpp}

\section{ODT}
\lstinputlisting{数据结构/ODT.cpp}

\input{数据结构/线段树/main.tex}

\section{差分}
\subection{离散化差分}

\lstinputlisting{数据结构/离散化差分.cpp}

\subsection{二维前缀和、差分}
\lstinputlisting{数据结构/二维差分.cpp}

\input{数据结构/扫描线/扫描线.tex}

\input{数据结构/Splay/main.tex}



\section{分块}
\lstinputlisting{数据结构/分块.cpp}

\section{ODT}
\lstinputlisting{数据结构/ODT.cpp}

\chapter{数据结构}

\section{并查集}

\lstinputlisting{数据结构/并查集.cpp}

\section{链式前向星}
链式前向星又名邻接表,其实现在我已经几乎不会再手写链式前向星而是采用\verb|vector|来代替
\begin{lstlisting}
vector<int> e[N];// 无边权
vector< pair<int,int> > e[N]; 有边权

e[u].push_back(v);// 加边(u,v)
e[u].push_back( { v, w } ); //加有权边 (u,v,w)
// 无向边 反过来再做一次就好

for( auto v : e[u] ){ // 遍历
}
for( auto [ v , w ] : e[u] ) { // 遍历有权边
}

\end{lstlisting}

\section{Hash}
\subsection{Hash表}
对数字的 hash
\begin{lstlisting}
for( int i = 1 ; i <= n ; i ++ ) b[i] = a[i]; // 复制数组
sort( b + 1 , b + 1 + n ) , m = unique( b + 1 , b + 1 + n ) - b;// 排序去重
for( int i = 1 ; i <= n ; i ++ )//hash
    a[i] = lower_bound( b + 1 , b + 1 + m , a[i] ) - b;
\end{lstlisting}
除此之外,如果更加复杂的 hash 全部使用\verb|unordered_map|容器

\chapter{数据结构}

\section{并查集}

\lstinputlisting{数据结构/并查集.cpp}

\section{链式前向星}
链式前向星又名邻接表,其实现在我已经几乎不会再手写链式前向星而是采用\verb|vector|来代替
\begin{lstlisting}
vector<int> e[N];// 无边权
vector< pair<int,int> > e[N]; 有边权

e[u].push_back(v);// 加边(u,v)
e[u].push_back( { v, w } ); //加有权边 (u,v,w)
// 无向边 反过来再做一次就好

for( auto v : e[u] ){ // 遍历
}
for( auto [ v , w ] : e[u] ) { // 遍历有权边
}

\end{lstlisting}

\section{Hash}
\subsection{Hash表}
对数字的 hash
\begin{lstlisting}
for( int i = 1 ; i <= n ; i ++ ) b[i] = a[i]; // 复制数组
sort( b + 1 , b + 1 + n ) , m = unique( b + 1 , b + 1 + n ) - b;// 排序去重
for( int i = 1 ; i <= n ; i ++ )//hash
    a[i] = lower_bound( b + 1 , b + 1 + m , a[i] ) - b;
\end{lstlisting}
除此之外,如果更加复杂的 hash 全部使用\verb|unordered_map|容器

\input{数据结构/栈/main.tex}

\input{数据结构/ST表/main.tex}

\input{数据结构/树状数组/main.tex}

\section{分块}
\lstinputlisting{数据结构/分块.cpp}

\section{ODT}
\lstinputlisting{数据结构/ODT.cpp}

\input{数据结构/线段树/main.tex}

\section{差分}
\subection{离散化差分}

\lstinputlisting{数据结构/离散化差分.cpp}

\subsection{二维前缀和、差分}
\lstinputlisting{数据结构/二维差分.cpp}

\input{数据结构/扫描线/扫描线.tex}

\input{数据结构/Splay/main.tex}


\chapter{数据结构}

\section{并查集}

\lstinputlisting{数据结构/并查集.cpp}

\section{链式前向星}
链式前向星又名邻接表,其实现在我已经几乎不会再手写链式前向星而是采用\verb|vector|来代替
\begin{lstlisting}
vector<int> e[N];// 无边权
vector< pair<int,int> > e[N]; 有边权

e[u].push_back(v);// 加边(u,v)
e[u].push_back( { v, w } ); //加有权边 (u,v,w)
// 无向边 反过来再做一次就好

for( auto v : e[u] ){ // 遍历
}
for( auto [ v , w ] : e[u] ) { // 遍历有权边
}

\end{lstlisting}

\section{Hash}
\subsection{Hash表}
对数字的 hash
\begin{lstlisting}
for( int i = 1 ; i <= n ; i ++ ) b[i] = a[i]; // 复制数组
sort( b + 1 , b + 1 + n ) , m = unique( b + 1 , b + 1 + n ) - b;// 排序去重
for( int i = 1 ; i <= n ; i ++ )//hash
    a[i] = lower_bound( b + 1 , b + 1 + m , a[i] ) - b;
\end{lstlisting}
除此之外,如果更加复杂的 hash 全部使用\verb|unordered_map|容器

\input{数据结构/栈/main.tex}

\input{数据结构/ST表/main.tex}

\input{数据结构/树状数组/main.tex}

\section{分块}
\lstinputlisting{数据结构/分块.cpp}

\section{ODT}
\lstinputlisting{数据结构/ODT.cpp}

\input{数据结构/线段树/main.tex}

\section{差分}
\subection{离散化差分}

\lstinputlisting{数据结构/离散化差分.cpp}

\subsection{二维前缀和、差分}
\lstinputlisting{数据结构/二维差分.cpp}

\input{数据结构/扫描线/扫描线.tex}

\input{数据结构/Splay/main.tex}


\chapter{数据结构}

\section{并查集}

\lstinputlisting{数据结构/并查集.cpp}

\section{链式前向星}
链式前向星又名邻接表,其实现在我已经几乎不会再手写链式前向星而是采用\verb|vector|来代替
\begin{lstlisting}
vector<int> e[N];// 无边权
vector< pair<int,int> > e[N]; 有边权

e[u].push_back(v);// 加边(u,v)
e[u].push_back( { v, w } ); //加有权边 (u,v,w)
// 无向边 反过来再做一次就好

for( auto v : e[u] ){ // 遍历
}
for( auto [ v , w ] : e[u] ) { // 遍历有权边
}

\end{lstlisting}

\section{Hash}
\subsection{Hash表}
对数字的 hash
\begin{lstlisting}
for( int i = 1 ; i <= n ; i ++ ) b[i] = a[i]; // 复制数组
sort( b + 1 , b + 1 + n ) , m = unique( b + 1 , b + 1 + n ) - b;// 排序去重
for( int i = 1 ; i <= n ; i ++ )//hash
    a[i] = lower_bound( b + 1 , b + 1 + m , a[i] ) - b;
\end{lstlisting}
除此之外,如果更加复杂的 hash 全部使用\verb|unordered_map|容器

\input{数据结构/栈/main.tex}

\input{数据结构/ST表/main.tex}

\input{数据结构/树状数组/main.tex}

\section{分块}
\lstinputlisting{数据结构/分块.cpp}

\section{ODT}
\lstinputlisting{数据结构/ODT.cpp}

\input{数据结构/线段树/main.tex}

\section{差分}
\subection{离散化差分}

\lstinputlisting{数据结构/离散化差分.cpp}

\subsection{二维前缀和、差分}
\lstinputlisting{数据结构/二维差分.cpp}

\input{数据结构/扫描线/扫描线.tex}

\input{数据结构/Splay/main.tex}


\section{分块}
\lstinputlisting{数据结构/分块.cpp}

\section{ODT}
\lstinputlisting{数据结构/ODT.cpp}

\chapter{数据结构}

\section{并查集}

\lstinputlisting{数据结构/并查集.cpp}

\section{链式前向星}
链式前向星又名邻接表,其实现在我已经几乎不会再手写链式前向星而是采用\verb|vector|来代替
\begin{lstlisting}
vector<int> e[N];// 无边权
vector< pair<int,int> > e[N]; 有边权

e[u].push_back(v);// 加边(u,v)
e[u].push_back( { v, w } ); //加有权边 (u,v,w)
// 无向边 反过来再做一次就好

for( auto v : e[u] ){ // 遍历
}
for( auto [ v , w ] : e[u] ) { // 遍历有权边
}

\end{lstlisting}

\section{Hash}
\subsection{Hash表}
对数字的 hash
\begin{lstlisting}
for( int i = 1 ; i <= n ; i ++ ) b[i] = a[i]; // 复制数组
sort( b + 1 , b + 1 + n ) , m = unique( b + 1 , b + 1 + n ) - b;// 排序去重
for( int i = 1 ; i <= n ; i ++ )//hash
    a[i] = lower_bound( b + 1 , b + 1 + m , a[i] ) - b;
\end{lstlisting}
除此之外,如果更加复杂的 hash 全部使用\verb|unordered_map|容器

\input{数据结构/栈/main.tex}

\input{数据结构/ST表/main.tex}

\input{数据结构/树状数组/main.tex}

\section{分块}
\lstinputlisting{数据结构/分块.cpp}

\section{ODT}
\lstinputlisting{数据结构/ODT.cpp}

\input{数据结构/线段树/main.tex}

\section{差分}
\subection{离散化差分}

\lstinputlisting{数据结构/离散化差分.cpp}

\subsection{二维前缀和、差分}
\lstinputlisting{数据结构/二维差分.cpp}

\input{数据结构/扫描线/扫描线.tex}

\input{数据结构/Splay/main.tex}


\section{差分}
\subection{离散化差分}

\lstinputlisting{数据结构/离散化差分.cpp}

\subsection{二维前缀和、差分}
\lstinputlisting{数据结构/二维差分.cpp}

\section{扫描线}

\subsection{求面积并}
求 $N$ 个矩形面积的并,每个矩形用$(x_a, y_a),(x_b, y_b)$表示。

将整个图形分为$2N$部分,这样的话每个矩形可以用两条线段$(x_a,y_a,y_b,1),(x_b,y_a,y_b,-1)$表示。

在需要用到扫描线的题目中$y$值通常很大,甚至可能不是整数,所以我们需要进行离散化。记$val(y)$为$y$离散化之后的值,$raw(i)$为$i$的原始坐标。在离散化之后有$tot$个$y$的坐标值,分别对应为$raw(1),raw(2),raw(3),\dots,raw(tot)$,则扫描线被分为$tot-1$段,其中第$i$段为$[raw(i),raw(i+1)]$。

将线段按照$x$值排序,初始每一段都是$0$。然后遍历每个线段$(x_i,y_a,y_b,v)$,如果到$x_{i-1}$的线段覆盖的中长度为$len$,则当前矩形的面积为$(x_i - x_{i-1})\times len$。然后给$[val(y_a),val(y_b)-1]$加$v$,相当于覆盖了$[x_i,x_{i+1}]$的部分。

\lstinputlisting{数据结构/扫描线/luoguP5490.cpp}

\subsection{二维数点}

单纯的二维数点数点问题,可以只用树状数组就可以维护。

$d(x,y)$表示从$(0,0)$到$(x,y)$中点的数量,因此从左下角$(a,b)$到右上角$(c,d)$中点的数量就可以表示为$d(c,d) - d(c,b-1) - d(a-1,d) + d( a-1,b-1)$ , 这个形式就是普通的二维前缀和。我们把式子稍作变形转换为$d((c,d) - d(c,b-1)) - (d(a-1,d) - d(a-1,b-1))$ 这样的话就可以用扫描线优化掉一维。

\lstinputlisting{数据结构/扫描线/luoguP2163.cpp}


\chapter{数据结构}

\section{并查集}

\lstinputlisting{数据结构/并查集.cpp}

\section{链式前向星}
链式前向星又名邻接表,其实现在我已经几乎不会再手写链式前向星而是采用\verb|vector|来代替
\begin{lstlisting}
vector<int> e[N];// 无边权
vector< pair<int,int> > e[N]; 有边权

e[u].push_back(v);// 加边(u,v)
e[u].push_back( { v, w } ); //加有权边 (u,v,w)
// 无向边 反过来再做一次就好

for( auto v : e[u] ){ // 遍历
}
for( auto [ v , w ] : e[u] ) { // 遍历有权边
}

\end{lstlisting}

\section{Hash}
\subsection{Hash表}
对数字的 hash
\begin{lstlisting}
for( int i = 1 ; i <= n ; i ++ ) b[i] = a[i]; // 复制数组
sort( b + 1 , b + 1 + n ) , m = unique( b + 1 , b + 1 + n ) - b;// 排序去重
for( int i = 1 ; i <= n ; i ++ )//hash
    a[i] = lower_bound( b + 1 , b + 1 + m , a[i] ) - b;
\end{lstlisting}
除此之外,如果更加复杂的 hash 全部使用\verb|unordered_map|容器

\input{数据结构/栈/main.tex}

\input{数据结构/ST表/main.tex}

\input{数据结构/树状数组/main.tex}

\section{分块}
\lstinputlisting{数据结构/分块.cpp}

\section{ODT}
\lstinputlisting{数据结构/ODT.cpp}

\input{数据结构/线段树/main.tex}

\section{差分}
\subection{离散化差分}

\lstinputlisting{数据结构/离散化差分.cpp}

\subsection{二维前缀和、差分}
\lstinputlisting{数据结构/二维差分.cpp}

\input{数据结构/扫描线/扫描线.tex}

\input{数据结构/Splay/main.tex}



\section{差分}
\subection{离散化差分}

\lstinputlisting{数据结构/离散化差分.cpp}

\subsection{二维前缀和、差分}
\lstinputlisting{数据结构/二维差分.cpp}

\section{扫描线}

\subsection{求面积并}
求 $N$ 个矩形面积的并,每个矩形用$(x_a, y_a),(x_b, y_b)$表示。

将整个图形分为$2N$部分,这样的话每个矩形可以用两条线段$(x_a,y_a,y_b,1),(x_b,y_a,y_b,-1)$表示。

在需要用到扫描线的题目中$y$值通常很大,甚至可能不是整数,所以我们需要进行离散化。记$val(y)$为$y$离散化之后的值,$raw(i)$为$i$的原始坐标。在离散化之后有$tot$个$y$的坐标值,分别对应为$raw(1),raw(2),raw(3),\dots,raw(tot)$,则扫描线被分为$tot-1$段,其中第$i$段为$[raw(i),raw(i+1)]$。

将线段按照$x$值排序,初始每一段都是$0$。然后遍历每个线段$(x_i,y_a,y_b,v)$,如果到$x_{i-1}$的线段覆盖的中长度为$len$,则当前矩形的面积为$(x_i - x_{i-1})\times len$。然后给$[val(y_a),val(y_b)-1]$加$v$,相当于覆盖了$[x_i,x_{i+1}]$的部分。

\lstinputlisting{数据结构/扫描线/luoguP5490.cpp}

\subsection{二维数点}

单纯的二维数点数点问题,可以只用树状数组就可以维护。

$d(x,y)$表示从$(0,0)$到$(x,y)$中点的数量,因此从左下角$(a,b)$到右上角$(c,d)$中点的数量就可以表示为$d(c,d) - d(c,b-1) - d(a-1,d) + d( a-1,b-1)$ , 这个形式就是普通的二维前缀和。我们把式子稍作变形转换为$d((c,d) - d(c,b-1)) - (d(a-1,d) - d(a-1,b-1))$ 这样的话就可以用扫描线优化掉一维。

\lstinputlisting{数据结构/扫描线/luoguP2163.cpp}


\chapter{数据结构}

\section{并查集}

\lstinputlisting{数据结构/并查集.cpp}

\section{链式前向星}
链式前向星又名邻接表,其实现在我已经几乎不会再手写链式前向星而是采用\verb|vector|来代替
\begin{lstlisting}
vector<int> e[N];// 无边权
vector< pair<int,int> > e[N]; 有边权

e[u].push_back(v);// 加边(u,v)
e[u].push_back( { v, w } ); //加有权边 (u,v,w)
// 无向边 反过来再做一次就好

for( auto v : e[u] ){ // 遍历
}
for( auto [ v , w ] : e[u] ) { // 遍历有权边
}

\end{lstlisting}

\section{Hash}
\subsection{Hash表}
对数字的 hash
\begin{lstlisting}
for( int i = 1 ; i <= n ; i ++ ) b[i] = a[i]; // 复制数组
sort( b + 1 , b + 1 + n ) , m = unique( b + 1 , b + 1 + n ) - b;// 排序去重
for( int i = 1 ; i <= n ; i ++ )//hash
    a[i] = lower_bound( b + 1 , b + 1 + m , a[i] ) - b;
\end{lstlisting}
除此之外,如果更加复杂的 hash 全部使用\verb|unordered_map|容器

\chapter{数据结构}

\section{并查集}

\lstinputlisting{数据结构/并查集.cpp}

\section{链式前向星}
链式前向星又名邻接表,其实现在我已经几乎不会再手写链式前向星而是采用\verb|vector|来代替
\begin{lstlisting}
vector<int> e[N];// 无边权
vector< pair<int,int> > e[N]; 有边权

e[u].push_back(v);// 加边(u,v)
e[u].push_back( { v, w } ); //加有权边 (u,v,w)
// 无向边 反过来再做一次就好

for( auto v : e[u] ){ // 遍历
}
for( auto [ v , w ] : e[u] ) { // 遍历有权边
}

\end{lstlisting}

\section{Hash}
\subsection{Hash表}
对数字的 hash
\begin{lstlisting}
for( int i = 1 ; i <= n ; i ++ ) b[i] = a[i]; // 复制数组
sort( b + 1 , b + 1 + n ) , m = unique( b + 1 , b + 1 + n ) - b;// 排序去重
for( int i = 1 ; i <= n ; i ++ )//hash
    a[i] = lower_bound( b + 1 , b + 1 + m , a[i] ) - b;
\end{lstlisting}
除此之外,如果更加复杂的 hash 全部使用\verb|unordered_map|容器

\input{数据结构/栈/main.tex}

\input{数据结构/ST表/main.tex}

\input{数据结构/树状数组/main.tex}

\section{分块}
\lstinputlisting{数据结构/分块.cpp}

\section{ODT}
\lstinputlisting{数据结构/ODT.cpp}

\input{数据结构/线段树/main.tex}

\section{差分}
\subection{离散化差分}

\lstinputlisting{数据结构/离散化差分.cpp}

\subsection{二维前缀和、差分}
\lstinputlisting{数据结构/二维差分.cpp}

\input{数据结构/扫描线/扫描线.tex}

\input{数据结构/Splay/main.tex}


\chapter{数据结构}

\section{并查集}

\lstinputlisting{数据结构/并查集.cpp}

\section{链式前向星}
链式前向星又名邻接表,其实现在我已经几乎不会再手写链式前向星而是采用\verb|vector|来代替
\begin{lstlisting}
vector<int> e[N];// 无边权
vector< pair<int,int> > e[N]; 有边权

e[u].push_back(v);// 加边(u,v)
e[u].push_back( { v, w } ); //加有权边 (u,v,w)
// 无向边 反过来再做一次就好

for( auto v : e[u] ){ // 遍历
}
for( auto [ v , w ] : e[u] ) { // 遍历有权边
}

\end{lstlisting}

\section{Hash}
\subsection{Hash表}
对数字的 hash
\begin{lstlisting}
for( int i = 1 ; i <= n ; i ++ ) b[i] = a[i]; // 复制数组
sort( b + 1 , b + 1 + n ) , m = unique( b + 1 , b + 1 + n ) - b;// 排序去重
for( int i = 1 ; i <= n ; i ++ )//hash
    a[i] = lower_bound( b + 1 , b + 1 + m , a[i] ) - b;
\end{lstlisting}
除此之外,如果更加复杂的 hash 全部使用\verb|unordered_map|容器

\input{数据结构/栈/main.tex}

\input{数据结构/ST表/main.tex}

\input{数据结构/树状数组/main.tex}

\section{分块}
\lstinputlisting{数据结构/分块.cpp}

\section{ODT}
\lstinputlisting{数据结构/ODT.cpp}

\input{数据结构/线段树/main.tex}

\section{差分}
\subection{离散化差分}

\lstinputlisting{数据结构/离散化差分.cpp}

\subsection{二维前缀和、差分}
\lstinputlisting{数据结构/二维差分.cpp}

\input{数据结构/扫描线/扫描线.tex}

\input{数据结构/Splay/main.tex}


\chapter{数据结构}

\section{并查集}

\lstinputlisting{数据结构/并查集.cpp}

\section{链式前向星}
链式前向星又名邻接表,其实现在我已经几乎不会再手写链式前向星而是采用\verb|vector|来代替
\begin{lstlisting}
vector<int> e[N];// 无边权
vector< pair<int,int> > e[N]; 有边权

e[u].push_back(v);// 加边(u,v)
e[u].push_back( { v, w } ); //加有权边 (u,v,w)
// 无向边 反过来再做一次就好

for( auto v : e[u] ){ // 遍历
}
for( auto [ v , w ] : e[u] ) { // 遍历有权边
}

\end{lstlisting}

\section{Hash}
\subsection{Hash表}
对数字的 hash
\begin{lstlisting}
for( int i = 1 ; i <= n ; i ++ ) b[i] = a[i]; // 复制数组
sort( b + 1 , b + 1 + n ) , m = unique( b + 1 , b + 1 + n ) - b;// 排序去重
for( int i = 1 ; i <= n ; i ++ )//hash
    a[i] = lower_bound( b + 1 , b + 1 + m , a[i] ) - b;
\end{lstlisting}
除此之外,如果更加复杂的 hash 全部使用\verb|unordered_map|容器

\input{数据结构/栈/main.tex}

\input{数据结构/ST表/main.tex}

\input{数据结构/树状数组/main.tex}

\section{分块}
\lstinputlisting{数据结构/分块.cpp}

\section{ODT}
\lstinputlisting{数据结构/ODT.cpp}

\input{数据结构/线段树/main.tex}

\section{差分}
\subection{离散化差分}

\lstinputlisting{数据结构/离散化差分.cpp}

\subsection{二维前缀和、差分}
\lstinputlisting{数据结构/二维差分.cpp}

\input{数据结构/扫描线/扫描线.tex}

\input{数据结构/Splay/main.tex}


\section{分块}
\lstinputlisting{数据结构/分块.cpp}

\section{ODT}
\lstinputlisting{数据结构/ODT.cpp}

\chapter{数据结构}

\section{并查集}

\lstinputlisting{数据结构/并查集.cpp}

\section{链式前向星}
链式前向星又名邻接表,其实现在我已经几乎不会再手写链式前向星而是采用\verb|vector|来代替
\begin{lstlisting}
vector<int> e[N];// 无边权
vector< pair<int,int> > e[N]; 有边权

e[u].push_back(v);// 加边(u,v)
e[u].push_back( { v, w } ); //加有权边 (u,v,w)
// 无向边 反过来再做一次就好

for( auto v : e[u] ){ // 遍历
}
for( auto [ v , w ] : e[u] ) { // 遍历有权边
}

\end{lstlisting}

\section{Hash}
\subsection{Hash表}
对数字的 hash
\begin{lstlisting}
for( int i = 1 ; i <= n ; i ++ ) b[i] = a[i]; // 复制数组
sort( b + 1 , b + 1 + n ) , m = unique( b + 1 , b + 1 + n ) - b;// 排序去重
for( int i = 1 ; i <= n ; i ++ )//hash
    a[i] = lower_bound( b + 1 , b + 1 + m , a[i] ) - b;
\end{lstlisting}
除此之外,如果更加复杂的 hash 全部使用\verb|unordered_map|容器

\input{数据结构/栈/main.tex}

\input{数据结构/ST表/main.tex}

\input{数据结构/树状数组/main.tex}

\section{分块}
\lstinputlisting{数据结构/分块.cpp}

\section{ODT}
\lstinputlisting{数据结构/ODT.cpp}

\input{数据结构/线段树/main.tex}

\section{差分}
\subection{离散化差分}

\lstinputlisting{数据结构/离散化差分.cpp}

\subsection{二维前缀和、差分}
\lstinputlisting{数据结构/二维差分.cpp}

\input{数据结构/扫描线/扫描线.tex}

\input{数据结构/Splay/main.tex}


\section{差分}
\subection{离散化差分}

\lstinputlisting{数据结构/离散化差分.cpp}

\subsection{二维前缀和、差分}
\lstinputlisting{数据结构/二维差分.cpp}

\section{扫描线}

\subsection{求面积并}
求 $N$ 个矩形面积的并,每个矩形用$(x_a, y_a),(x_b, y_b)$表示。

将整个图形分为$2N$部分,这样的话每个矩形可以用两条线段$(x_a,y_a,y_b,1),(x_b,y_a,y_b,-1)$表示。

在需要用到扫描线的题目中$y$值通常很大,甚至可能不是整数,所以我们需要进行离散化。记$val(y)$为$y$离散化之后的值,$raw(i)$为$i$的原始坐标。在离散化之后有$tot$个$y$的坐标值,分别对应为$raw(1),raw(2),raw(3),\dots,raw(tot)$,则扫描线被分为$tot-1$段,其中第$i$段为$[raw(i),raw(i+1)]$。

将线段按照$x$值排序,初始每一段都是$0$。然后遍历每个线段$(x_i,y_a,y_b,v)$,如果到$x_{i-1}$的线段覆盖的中长度为$len$,则当前矩形的面积为$(x_i - x_{i-1})\times len$。然后给$[val(y_a),val(y_b)-1]$加$v$,相当于覆盖了$[x_i,x_{i+1}]$的部分。

\lstinputlisting{数据结构/扫描线/luoguP5490.cpp}

\subsection{二维数点}

单纯的二维数点数点问题,可以只用树状数组就可以维护。

$d(x,y)$表示从$(0,0)$到$(x,y)$中点的数量,因此从左下角$(a,b)$到右上角$(c,d)$中点的数量就可以表示为$d(c,d) - d(c,b-1) - d(a-1,d) + d( a-1,b-1)$ , 这个形式就是普通的二维前缀和。我们把式子稍作变形转换为$d((c,d) - d(c,b-1)) - (d(a-1,d) - d(a-1,b-1))$ 这样的话就可以用扫描线优化掉一维。

\lstinputlisting{数据结构/扫描线/luoguP2163.cpp}


\chapter{数据结构}

\section{并查集}

\lstinputlisting{数据结构/并查集.cpp}

\section{链式前向星}
链式前向星又名邻接表,其实现在我已经几乎不会再手写链式前向星而是采用\verb|vector|来代替
\begin{lstlisting}
vector<int> e[N];// 无边权
vector< pair<int,int> > e[N]; 有边权

e[u].push_back(v);// 加边(u,v)
e[u].push_back( { v, w } ); //加有权边 (u,v,w)
// 无向边 反过来再做一次就好

for( auto v : e[u] ){ // 遍历
}
for( auto [ v , w ] : e[u] ) { // 遍历有权边
}

\end{lstlisting}

\section{Hash}
\subsection{Hash表}
对数字的 hash
\begin{lstlisting}
for( int i = 1 ; i <= n ; i ++ ) b[i] = a[i]; // 复制数组
sort( b + 1 , b + 1 + n ) , m = unique( b + 1 , b + 1 + n ) - b;// 排序去重
for( int i = 1 ; i <= n ; i ++ )//hash
    a[i] = lower_bound( b + 1 , b + 1 + m , a[i] ) - b;
\end{lstlisting}
除此之外,如果更加复杂的 hash 全部使用\verb|unordered_map|容器

\input{数据结构/栈/main.tex}

\input{数据结构/ST表/main.tex}

\input{数据结构/树状数组/main.tex}

\section{分块}
\lstinputlisting{数据结构/分块.cpp}

\section{ODT}
\lstinputlisting{数据结构/ODT.cpp}

\input{数据结构/线段树/main.tex}

\section{差分}
\subection{离散化差分}

\lstinputlisting{数据结构/离散化差分.cpp}

\subsection{二维前缀和、差分}
\lstinputlisting{数据结构/二维差分.cpp}

\input{数据结构/扫描线/扫描线.tex}

\input{数据结构/Splay/main.tex}




\chapter{数据结构}

\section{并查集}

\lstinputlisting{数据结构/并查集.cpp}

\section{链式前向星}
链式前向星又名邻接表,其实现在我已经几乎不会再手写链式前向星而是采用\verb|vector|来代替
\begin{lstlisting}
vector<int> e[N];// 无边权
vector< pair<int,int> > e[N]; 有边权

e[u].push_back(v);// 加边(u,v)
e[u].push_back( { v, w } ); //加有权边 (u,v,w)
// 无向边 反过来再做一次就好

for( auto v : e[u] ){ // 遍历
}
for( auto [ v , w ] : e[u] ) { // 遍历有权边
}

\end{lstlisting}

\section{Hash}
\subsection{Hash表}
对数字的 hash
\begin{lstlisting}
for( int i = 1 ; i <= n ; i ++ ) b[i] = a[i]; // 复制数组
sort( b + 1 , b + 1 + n ) , m = unique( b + 1 , b + 1 + n ) - b;// 排序去重
for( int i = 1 ; i <= n ; i ++ )//hash
    a[i] = lower_bound( b + 1 , b + 1 + m , a[i] ) - b;
\end{lstlisting}
除此之外,如果更加复杂的 hash 全部使用\verb|unordered_map|容器

\chapter{数据结构}

\section{并查集}

\lstinputlisting{数据结构/并查集.cpp}

\section{链式前向星}
链式前向星又名邻接表,其实现在我已经几乎不会再手写链式前向星而是采用\verb|vector|来代替
\begin{lstlisting}
vector<int> e[N];// 无边权
vector< pair<int,int> > e[N]; 有边权

e[u].push_back(v);// 加边(u,v)
e[u].push_back( { v, w } ); //加有权边 (u,v,w)
// 无向边 反过来再做一次就好

for( auto v : e[u] ){ // 遍历
}
for( auto [ v , w ] : e[u] ) { // 遍历有权边
}

\end{lstlisting}

\section{Hash}
\subsection{Hash表}
对数字的 hash
\begin{lstlisting}
for( int i = 1 ; i <= n ; i ++ ) b[i] = a[i]; // 复制数组
sort( b + 1 , b + 1 + n ) , m = unique( b + 1 , b + 1 + n ) - b;// 排序去重
for( int i = 1 ; i <= n ; i ++ )//hash
    a[i] = lower_bound( b + 1 , b + 1 + m , a[i] ) - b;
\end{lstlisting}
除此之外,如果更加复杂的 hash 全部使用\verb|unordered_map|容器

\chapter{数据结构}

\section{并查集}

\lstinputlisting{数据结构/并查集.cpp}

\section{链式前向星}
链式前向星又名邻接表,其实现在我已经几乎不会再手写链式前向星而是采用\verb|vector|来代替
\begin{lstlisting}
vector<int> e[N];// 无边权
vector< pair<int,int> > e[N]; 有边权

e[u].push_back(v);// 加边(u,v)
e[u].push_back( { v, w } ); //加有权边 (u,v,w)
// 无向边 反过来再做一次就好

for( auto v : e[u] ){ // 遍历
}
for( auto [ v , w ] : e[u] ) { // 遍历有权边
}

\end{lstlisting}

\section{Hash}
\subsection{Hash表}
对数字的 hash
\begin{lstlisting}
for( int i = 1 ; i <= n ; i ++ ) b[i] = a[i]; // 复制数组
sort( b + 1 , b + 1 + n ) , m = unique( b + 1 , b + 1 + n ) - b;// 排序去重
for( int i = 1 ; i <= n ; i ++ )//hash
    a[i] = lower_bound( b + 1 , b + 1 + m , a[i] ) - b;
\end{lstlisting}
除此之外,如果更加复杂的 hash 全部使用\verb|unordered_map|容器

\input{数据结构/栈/main.tex}

\input{数据结构/ST表/main.tex}

\input{数据结构/树状数组/main.tex}

\section{分块}
\lstinputlisting{数据结构/分块.cpp}

\section{ODT}
\lstinputlisting{数据结构/ODT.cpp}

\input{数据结构/线段树/main.tex}

\section{差分}
\subection{离散化差分}

\lstinputlisting{数据结构/离散化差分.cpp}

\subsection{二维前缀和、差分}
\lstinputlisting{数据结构/二维差分.cpp}

\input{数据结构/扫描线/扫描线.tex}

\input{数据结构/Splay/main.tex}


\chapter{数据结构}

\section{并查集}

\lstinputlisting{数据结构/并查集.cpp}

\section{链式前向星}
链式前向星又名邻接表,其实现在我已经几乎不会再手写链式前向星而是采用\verb|vector|来代替
\begin{lstlisting}
vector<int> e[N];// 无边权
vector< pair<int,int> > e[N]; 有边权

e[u].push_back(v);// 加边(u,v)
e[u].push_back( { v, w } ); //加有权边 (u,v,w)
// 无向边 反过来再做一次就好

for( auto v : e[u] ){ // 遍历
}
for( auto [ v , w ] : e[u] ) { // 遍历有权边
}

\end{lstlisting}

\section{Hash}
\subsection{Hash表}
对数字的 hash
\begin{lstlisting}
for( int i = 1 ; i <= n ; i ++ ) b[i] = a[i]; // 复制数组
sort( b + 1 , b + 1 + n ) , m = unique( b + 1 , b + 1 + n ) - b;// 排序去重
for( int i = 1 ; i <= n ; i ++ )//hash
    a[i] = lower_bound( b + 1 , b + 1 + m , a[i] ) - b;
\end{lstlisting}
除此之外,如果更加复杂的 hash 全部使用\verb|unordered_map|容器

\input{数据结构/栈/main.tex}

\input{数据结构/ST表/main.tex}

\input{数据结构/树状数组/main.tex}

\section{分块}
\lstinputlisting{数据结构/分块.cpp}

\section{ODT}
\lstinputlisting{数据结构/ODT.cpp}

\input{数据结构/线段树/main.tex}

\section{差分}
\subection{离散化差分}

\lstinputlisting{数据结构/离散化差分.cpp}

\subsection{二维前缀和、差分}
\lstinputlisting{数据结构/二维差分.cpp}

\input{数据结构/扫描线/扫描线.tex}

\input{数据结构/Splay/main.tex}


\chapter{数据结构}

\section{并查集}

\lstinputlisting{数据结构/并查集.cpp}

\section{链式前向星}
链式前向星又名邻接表,其实现在我已经几乎不会再手写链式前向星而是采用\verb|vector|来代替
\begin{lstlisting}
vector<int> e[N];// 无边权
vector< pair<int,int> > e[N]; 有边权

e[u].push_back(v);// 加边(u,v)
e[u].push_back( { v, w } ); //加有权边 (u,v,w)
// 无向边 反过来再做一次就好

for( auto v : e[u] ){ // 遍历
}
for( auto [ v , w ] : e[u] ) { // 遍历有权边
}

\end{lstlisting}

\section{Hash}
\subsection{Hash表}
对数字的 hash
\begin{lstlisting}
for( int i = 1 ; i <= n ; i ++ ) b[i] = a[i]; // 复制数组
sort( b + 1 , b + 1 + n ) , m = unique( b + 1 , b + 1 + n ) - b;// 排序去重
for( int i = 1 ; i <= n ; i ++ )//hash
    a[i] = lower_bound( b + 1 , b + 1 + m , a[i] ) - b;
\end{lstlisting}
除此之外,如果更加复杂的 hash 全部使用\verb|unordered_map|容器

\input{数据结构/栈/main.tex}

\input{数据结构/ST表/main.tex}

\input{数据结构/树状数组/main.tex}

\section{分块}
\lstinputlisting{数据结构/分块.cpp}

\section{ODT}
\lstinputlisting{数据结构/ODT.cpp}

\input{数据结构/线段树/main.tex}

\section{差分}
\subection{离散化差分}

\lstinputlisting{数据结构/离散化差分.cpp}

\subsection{二维前缀和、差分}
\lstinputlisting{数据结构/二维差分.cpp}

\input{数据结构/扫描线/扫描线.tex}

\input{数据结构/Splay/main.tex}


\section{分块}
\lstinputlisting{数据结构/分块.cpp}

\section{ODT}
\lstinputlisting{数据结构/ODT.cpp}

\chapter{数据结构}

\section{并查集}

\lstinputlisting{数据结构/并查集.cpp}

\section{链式前向星}
链式前向星又名邻接表,其实现在我已经几乎不会再手写链式前向星而是采用\verb|vector|来代替
\begin{lstlisting}
vector<int> e[N];// 无边权
vector< pair<int,int> > e[N]; 有边权

e[u].push_back(v);// 加边(u,v)
e[u].push_back( { v, w } ); //加有权边 (u,v,w)
// 无向边 反过来再做一次就好

for( auto v : e[u] ){ // 遍历
}
for( auto [ v , w ] : e[u] ) { // 遍历有权边
}

\end{lstlisting}

\section{Hash}
\subsection{Hash表}
对数字的 hash
\begin{lstlisting}
for( int i = 1 ; i <= n ; i ++ ) b[i] = a[i]; // 复制数组
sort( b + 1 , b + 1 + n ) , m = unique( b + 1 , b + 1 + n ) - b;// 排序去重
for( int i = 1 ; i <= n ; i ++ )//hash
    a[i] = lower_bound( b + 1 , b + 1 + m , a[i] ) - b;
\end{lstlisting}
除此之外,如果更加复杂的 hash 全部使用\verb|unordered_map|容器

\input{数据结构/栈/main.tex}

\input{数据结构/ST表/main.tex}

\input{数据结构/树状数组/main.tex}

\section{分块}
\lstinputlisting{数据结构/分块.cpp}

\section{ODT}
\lstinputlisting{数据结构/ODT.cpp}

\input{数据结构/线段树/main.tex}

\section{差分}
\subection{离散化差分}

\lstinputlisting{数据结构/离散化差分.cpp}

\subsection{二维前缀和、差分}
\lstinputlisting{数据结构/二维差分.cpp}

\input{数据结构/扫描线/扫描线.tex}

\input{数据结构/Splay/main.tex}


\section{差分}
\subection{离散化差分}

\lstinputlisting{数据结构/离散化差分.cpp}

\subsection{二维前缀和、差分}
\lstinputlisting{数据结构/二维差分.cpp}

\section{扫描线}

\subsection{求面积并}
求 $N$ 个矩形面积的并,每个矩形用$(x_a, y_a),(x_b, y_b)$表示。

将整个图形分为$2N$部分,这样的话每个矩形可以用两条线段$(x_a,y_a,y_b,1),(x_b,y_a,y_b,-1)$表示。

在需要用到扫描线的题目中$y$值通常很大,甚至可能不是整数,所以我们需要进行离散化。记$val(y)$为$y$离散化之后的值,$raw(i)$为$i$的原始坐标。在离散化之后有$tot$个$y$的坐标值,分别对应为$raw(1),raw(2),raw(3),\dots,raw(tot)$,则扫描线被分为$tot-1$段,其中第$i$段为$[raw(i),raw(i+1)]$。

将线段按照$x$值排序,初始每一段都是$0$。然后遍历每个线段$(x_i,y_a,y_b,v)$,如果到$x_{i-1}$的线段覆盖的中长度为$len$,则当前矩形的面积为$(x_i - x_{i-1})\times len$。然后给$[val(y_a),val(y_b)-1]$加$v$,相当于覆盖了$[x_i,x_{i+1}]$的部分。

\lstinputlisting{数据结构/扫描线/luoguP5490.cpp}

\subsection{二维数点}

单纯的二维数点数点问题,可以只用树状数组就可以维护。

$d(x,y)$表示从$(0,0)$到$(x,y)$中点的数量,因此从左下角$(a,b)$到右上角$(c,d)$中点的数量就可以表示为$d(c,d) - d(c,b-1) - d(a-1,d) + d( a-1,b-1)$ , 这个形式就是普通的二维前缀和。我们把式子稍作变形转换为$d((c,d) - d(c,b-1)) - (d(a-1,d) - d(a-1,b-1))$ 这样的话就可以用扫描线优化掉一维。

\lstinputlisting{数据结构/扫描线/luoguP2163.cpp}


\chapter{数据结构}

\section{并查集}

\lstinputlisting{数据结构/并查集.cpp}

\section{链式前向星}
链式前向星又名邻接表,其实现在我已经几乎不会再手写链式前向星而是采用\verb|vector|来代替
\begin{lstlisting}
vector<int> e[N];// 无边权
vector< pair<int,int> > e[N]; 有边权

e[u].push_back(v);// 加边(u,v)
e[u].push_back( { v, w } ); //加有权边 (u,v,w)
// 无向边 反过来再做一次就好

for( auto v : e[u] ){ // 遍历
}
for( auto [ v , w ] : e[u] ) { // 遍历有权边
}

\end{lstlisting}

\section{Hash}
\subsection{Hash表}
对数字的 hash
\begin{lstlisting}
for( int i = 1 ; i <= n ; i ++ ) b[i] = a[i]; // 复制数组
sort( b + 1 , b + 1 + n ) , m = unique( b + 1 , b + 1 + n ) - b;// 排序去重
for( int i = 1 ; i <= n ; i ++ )//hash
    a[i] = lower_bound( b + 1 , b + 1 + m , a[i] ) - b;
\end{lstlisting}
除此之外,如果更加复杂的 hash 全部使用\verb|unordered_map|容器

\input{数据结构/栈/main.tex}

\input{数据结构/ST表/main.tex}

\input{数据结构/树状数组/main.tex}

\section{分块}
\lstinputlisting{数据结构/分块.cpp}

\section{ODT}
\lstinputlisting{数据结构/ODT.cpp}

\input{数据结构/线段树/main.tex}

\section{差分}
\subection{离散化差分}

\lstinputlisting{数据结构/离散化差分.cpp}

\subsection{二维前缀和、差分}
\lstinputlisting{数据结构/二维差分.cpp}

\input{数据结构/扫描线/扫描线.tex}

\input{数据结构/Splay/main.tex}



\chapter{数据结构}

\section{并查集}

\lstinputlisting{数据结构/并查集.cpp}

\section{链式前向星}
链式前向星又名邻接表,其实现在我已经几乎不会再手写链式前向星而是采用\verb|vector|来代替
\begin{lstlisting}
vector<int> e[N];// 无边权
vector< pair<int,int> > e[N]; 有边权

e[u].push_back(v);// 加边(u,v)
e[u].push_back( { v, w } ); //加有权边 (u,v,w)
// 无向边 反过来再做一次就好

for( auto v : e[u] ){ // 遍历
}
for( auto [ v , w ] : e[u] ) { // 遍历有权边
}

\end{lstlisting}

\section{Hash}
\subsection{Hash表}
对数字的 hash
\begin{lstlisting}
for( int i = 1 ; i <= n ; i ++ ) b[i] = a[i]; // 复制数组
sort( b + 1 , b + 1 + n ) , m = unique( b + 1 , b + 1 + n ) - b;// 排序去重
for( int i = 1 ; i <= n ; i ++ )//hash
    a[i] = lower_bound( b + 1 , b + 1 + m , a[i] ) - b;
\end{lstlisting}
除此之外,如果更加复杂的 hash 全部使用\verb|unordered_map|容器

\chapter{数据结构}

\section{并查集}

\lstinputlisting{数据结构/并查集.cpp}

\section{链式前向星}
链式前向星又名邻接表,其实现在我已经几乎不会再手写链式前向星而是采用\verb|vector|来代替
\begin{lstlisting}
vector<int> e[N];// 无边权
vector< pair<int,int> > e[N]; 有边权

e[u].push_back(v);// 加边(u,v)
e[u].push_back( { v, w } ); //加有权边 (u,v,w)
// 无向边 反过来再做一次就好

for( auto v : e[u] ){ // 遍历
}
for( auto [ v , w ] : e[u] ) { // 遍历有权边
}

\end{lstlisting}

\section{Hash}
\subsection{Hash表}
对数字的 hash
\begin{lstlisting}
for( int i = 1 ; i <= n ; i ++ ) b[i] = a[i]; // 复制数组
sort( b + 1 , b + 1 + n ) , m = unique( b + 1 , b + 1 + n ) - b;// 排序去重
for( int i = 1 ; i <= n ; i ++ )//hash
    a[i] = lower_bound( b + 1 , b + 1 + m , a[i] ) - b;
\end{lstlisting}
除此之外,如果更加复杂的 hash 全部使用\verb|unordered_map|容器

\input{数据结构/栈/main.tex}

\input{数据结构/ST表/main.tex}

\input{数据结构/树状数组/main.tex}

\section{分块}
\lstinputlisting{数据结构/分块.cpp}

\section{ODT}
\lstinputlisting{数据结构/ODT.cpp}

\input{数据结构/线段树/main.tex}

\section{差分}
\subection{离散化差分}

\lstinputlisting{数据结构/离散化差分.cpp}

\subsection{二维前缀和、差分}
\lstinputlisting{数据结构/二维差分.cpp}

\input{数据结构/扫描线/扫描线.tex}

\input{数据结构/Splay/main.tex}


\chapter{数据结构}

\section{并查集}

\lstinputlisting{数据结构/并查集.cpp}

\section{链式前向星}
链式前向星又名邻接表,其实现在我已经几乎不会再手写链式前向星而是采用\verb|vector|来代替
\begin{lstlisting}
vector<int> e[N];// 无边权
vector< pair<int,int> > e[N]; 有边权

e[u].push_back(v);// 加边(u,v)
e[u].push_back( { v, w } ); //加有权边 (u,v,w)
// 无向边 反过来再做一次就好

for( auto v : e[u] ){ // 遍历
}
for( auto [ v , w ] : e[u] ) { // 遍历有权边
}

\end{lstlisting}

\section{Hash}
\subsection{Hash表}
对数字的 hash
\begin{lstlisting}
for( int i = 1 ; i <= n ; i ++ ) b[i] = a[i]; // 复制数组
sort( b + 1 , b + 1 + n ) , m = unique( b + 1 , b + 1 + n ) - b;// 排序去重
for( int i = 1 ; i <= n ; i ++ )//hash
    a[i] = lower_bound( b + 1 , b + 1 + m , a[i] ) - b;
\end{lstlisting}
除此之外,如果更加复杂的 hash 全部使用\verb|unordered_map|容器

\input{数据结构/栈/main.tex}

\input{数据结构/ST表/main.tex}

\input{数据结构/树状数组/main.tex}

\section{分块}
\lstinputlisting{数据结构/分块.cpp}

\section{ODT}
\lstinputlisting{数据结构/ODT.cpp}

\input{数据结构/线段树/main.tex}

\section{差分}
\subection{离散化差分}

\lstinputlisting{数据结构/离散化差分.cpp}

\subsection{二维前缀和、差分}
\lstinputlisting{数据结构/二维差分.cpp}

\input{数据结构/扫描线/扫描线.tex}

\input{数据结构/Splay/main.tex}


\chapter{数据结构}

\section{并查集}

\lstinputlisting{数据结构/并查集.cpp}

\section{链式前向星}
链式前向星又名邻接表,其实现在我已经几乎不会再手写链式前向星而是采用\verb|vector|来代替
\begin{lstlisting}
vector<int> e[N];// 无边权
vector< pair<int,int> > e[N]; 有边权

e[u].push_back(v);// 加边(u,v)
e[u].push_back( { v, w } ); //加有权边 (u,v,w)
// 无向边 反过来再做一次就好

for( auto v : e[u] ){ // 遍历
}
for( auto [ v , w ] : e[u] ) { // 遍历有权边
}

\end{lstlisting}

\section{Hash}
\subsection{Hash表}
对数字的 hash
\begin{lstlisting}
for( int i = 1 ; i <= n ; i ++ ) b[i] = a[i]; // 复制数组
sort( b + 1 , b + 1 + n ) , m = unique( b + 1 , b + 1 + n ) - b;// 排序去重
for( int i = 1 ; i <= n ; i ++ )//hash
    a[i] = lower_bound( b + 1 , b + 1 + m , a[i] ) - b;
\end{lstlisting}
除此之外,如果更加复杂的 hash 全部使用\verb|unordered_map|容器

\input{数据结构/栈/main.tex}

\input{数据结构/ST表/main.tex}

\input{数据结构/树状数组/main.tex}

\section{分块}
\lstinputlisting{数据结构/分块.cpp}

\section{ODT}
\lstinputlisting{数据结构/ODT.cpp}

\input{数据结构/线段树/main.tex}

\section{差分}
\subection{离散化差分}

\lstinputlisting{数据结构/离散化差分.cpp}

\subsection{二维前缀和、差分}
\lstinputlisting{数据结构/二维差分.cpp}

\input{数据结构/扫描线/扫描线.tex}

\input{数据结构/Splay/main.tex}


\section{分块}
\lstinputlisting{数据结构/分块.cpp}

\section{ODT}
\lstinputlisting{数据结构/ODT.cpp}

\chapter{数据结构}

\section{并查集}

\lstinputlisting{数据结构/并查集.cpp}

\section{链式前向星}
链式前向星又名邻接表,其实现在我已经几乎不会再手写链式前向星而是采用\verb|vector|来代替
\begin{lstlisting}
vector<int> e[N];// 无边权
vector< pair<int,int> > e[N]; 有边权

e[u].push_back(v);// 加边(u,v)
e[u].push_back( { v, w } ); //加有权边 (u,v,w)
// 无向边 反过来再做一次就好

for( auto v : e[u] ){ // 遍历
}
for( auto [ v , w ] : e[u] ) { // 遍历有权边
}

\end{lstlisting}

\section{Hash}
\subsection{Hash表}
对数字的 hash
\begin{lstlisting}
for( int i = 1 ; i <= n ; i ++ ) b[i] = a[i]; // 复制数组
sort( b + 1 , b + 1 + n ) , m = unique( b + 1 , b + 1 + n ) - b;// 排序去重
for( int i = 1 ; i <= n ; i ++ )//hash
    a[i] = lower_bound( b + 1 , b + 1 + m , a[i] ) - b;
\end{lstlisting}
除此之外,如果更加复杂的 hash 全部使用\verb|unordered_map|容器

\input{数据结构/栈/main.tex}

\input{数据结构/ST表/main.tex}

\input{数据结构/树状数组/main.tex}

\section{分块}
\lstinputlisting{数据结构/分块.cpp}

\section{ODT}
\lstinputlisting{数据结构/ODT.cpp}

\input{数据结构/线段树/main.tex}

\section{差分}
\subection{离散化差分}

\lstinputlisting{数据结构/离散化差分.cpp}

\subsection{二维前缀和、差分}
\lstinputlisting{数据结构/二维差分.cpp}

\input{数据结构/扫描线/扫描线.tex}

\input{数据结构/Splay/main.tex}


\section{差分}
\subection{离散化差分}

\lstinputlisting{数据结构/离散化差分.cpp}

\subsection{二维前缀和、差分}
\lstinputlisting{数据结构/二维差分.cpp}

\section{扫描线}

\subsection{求面积并}
求 $N$ 个矩形面积的并,每个矩形用$(x_a, y_a),(x_b, y_b)$表示。

将整个图形分为$2N$部分,这样的话每个矩形可以用两条线段$(x_a,y_a,y_b,1),(x_b,y_a,y_b,-1)$表示。

在需要用到扫描线的题目中$y$值通常很大,甚至可能不是整数,所以我们需要进行离散化。记$val(y)$为$y$离散化之后的值,$raw(i)$为$i$的原始坐标。在离散化之后有$tot$个$y$的坐标值,分别对应为$raw(1),raw(2),raw(3),\dots,raw(tot)$,则扫描线被分为$tot-1$段,其中第$i$段为$[raw(i),raw(i+1)]$。

将线段按照$x$值排序,初始每一段都是$0$。然后遍历每个线段$(x_i,y_a,y_b,v)$,如果到$x_{i-1}$的线段覆盖的中长度为$len$,则当前矩形的面积为$(x_i - x_{i-1})\times len$。然后给$[val(y_a),val(y_b)-1]$加$v$,相当于覆盖了$[x_i,x_{i+1}]$的部分。

\lstinputlisting{数据结构/扫描线/luoguP5490.cpp}

\subsection{二维数点}

单纯的二维数点数点问题,可以只用树状数组就可以维护。

$d(x,y)$表示从$(0,0)$到$(x,y)$中点的数量,因此从左下角$(a,b)$到右上角$(c,d)$中点的数量就可以表示为$d(c,d) - d(c,b-1) - d(a-1,d) + d( a-1,b-1)$ , 这个形式就是普通的二维前缀和。我们把式子稍作变形转换为$d((c,d) - d(c,b-1)) - (d(a-1,d) - d(a-1,b-1))$ 这样的话就可以用扫描线优化掉一维。

\lstinputlisting{数据结构/扫描线/luoguP2163.cpp}


\chapter{数据结构}

\section{并查集}

\lstinputlisting{数据结构/并查集.cpp}

\section{链式前向星}
链式前向星又名邻接表,其实现在我已经几乎不会再手写链式前向星而是采用\verb|vector|来代替
\begin{lstlisting}
vector<int> e[N];// 无边权
vector< pair<int,int> > e[N]; 有边权

e[u].push_back(v);// 加边(u,v)
e[u].push_back( { v, w } ); //加有权边 (u,v,w)
// 无向边 反过来再做一次就好

for( auto v : e[u] ){ // 遍历
}
for( auto [ v , w ] : e[u] ) { // 遍历有权边
}

\end{lstlisting}

\section{Hash}
\subsection{Hash表}
对数字的 hash
\begin{lstlisting}
for( int i = 1 ; i <= n ; i ++ ) b[i] = a[i]; // 复制数组
sort( b + 1 , b + 1 + n ) , m = unique( b + 1 , b + 1 + n ) - b;// 排序去重
for( int i = 1 ; i <= n ; i ++ )//hash
    a[i] = lower_bound( b + 1 , b + 1 + m , a[i] ) - b;
\end{lstlisting}
除此之外,如果更加复杂的 hash 全部使用\verb|unordered_map|容器

\input{数据结构/栈/main.tex}

\input{数据结构/ST表/main.tex}

\input{数据结构/树状数组/main.tex}

\section{分块}
\lstinputlisting{数据结构/分块.cpp}

\section{ODT}
\lstinputlisting{数据结构/ODT.cpp}

\input{数据结构/线段树/main.tex}

\section{差分}
\subection{离散化差分}

\lstinputlisting{数据结构/离散化差分.cpp}

\subsection{二维前缀和、差分}
\lstinputlisting{数据结构/二维差分.cpp}

\input{数据结构/扫描线/扫描线.tex}

\input{数据结构/Splay/main.tex}



\chapter{数据结构}

\section{并查集}

\lstinputlisting{数据结构/并查集.cpp}

\section{链式前向星}
链式前向星又名邻接表,其实现在我已经几乎不会再手写链式前向星而是采用\verb|vector|来代替
\begin{lstlisting}
vector<int> e[N];// 无边权
vector< pair<int,int> > e[N]; 有边权

e[u].push_back(v);// 加边(u,v)
e[u].push_back( { v, w } ); //加有权边 (u,v,w)
// 无向边 反过来再做一次就好

for( auto v : e[u] ){ // 遍历
}
for( auto [ v , w ] : e[u] ) { // 遍历有权边
}

\end{lstlisting}

\section{Hash}
\subsection{Hash表}
对数字的 hash
\begin{lstlisting}
for( int i = 1 ; i <= n ; i ++ ) b[i] = a[i]; // 复制数组
sort( b + 1 , b + 1 + n ) , m = unique( b + 1 , b + 1 + n ) - b;// 排序去重
for( int i = 1 ; i <= n ; i ++ )//hash
    a[i] = lower_bound( b + 1 , b + 1 + m , a[i] ) - b;
\end{lstlisting}
除此之外,如果更加复杂的 hash 全部使用\verb|unordered_map|容器

\chapter{数据结构}

\section{并查集}

\lstinputlisting{数据结构/并查集.cpp}

\section{链式前向星}
链式前向星又名邻接表,其实现在我已经几乎不会再手写链式前向星而是采用\verb|vector|来代替
\begin{lstlisting}
vector<int> e[N];// 无边权
vector< pair<int,int> > e[N]; 有边权

e[u].push_back(v);// 加边(u,v)
e[u].push_back( { v, w } ); //加有权边 (u,v,w)
// 无向边 反过来再做一次就好

for( auto v : e[u] ){ // 遍历
}
for( auto [ v , w ] : e[u] ) { // 遍历有权边
}

\end{lstlisting}

\section{Hash}
\subsection{Hash表}
对数字的 hash
\begin{lstlisting}
for( int i = 1 ; i <= n ; i ++ ) b[i] = a[i]; // 复制数组
sort( b + 1 , b + 1 + n ) , m = unique( b + 1 , b + 1 + n ) - b;// 排序去重
for( int i = 1 ; i <= n ; i ++ )//hash
    a[i] = lower_bound( b + 1 , b + 1 + m , a[i] ) - b;
\end{lstlisting}
除此之外,如果更加复杂的 hash 全部使用\verb|unordered_map|容器

\input{数据结构/栈/main.tex}

\input{数据结构/ST表/main.tex}

\input{数据结构/树状数组/main.tex}

\section{分块}
\lstinputlisting{数据结构/分块.cpp}

\section{ODT}
\lstinputlisting{数据结构/ODT.cpp}

\input{数据结构/线段树/main.tex}

\section{差分}
\subection{离散化差分}

\lstinputlisting{数据结构/离散化差分.cpp}

\subsection{二维前缀和、差分}
\lstinputlisting{数据结构/二维差分.cpp}

\input{数据结构/扫描线/扫描线.tex}

\input{数据结构/Splay/main.tex}


\chapter{数据结构}

\section{并查集}

\lstinputlisting{数据结构/并查集.cpp}

\section{链式前向星}
链式前向星又名邻接表,其实现在我已经几乎不会再手写链式前向星而是采用\verb|vector|来代替
\begin{lstlisting}
vector<int> e[N];// 无边权
vector< pair<int,int> > e[N]; 有边权

e[u].push_back(v);// 加边(u,v)
e[u].push_back( { v, w } ); //加有权边 (u,v,w)
// 无向边 反过来再做一次就好

for( auto v : e[u] ){ // 遍历
}
for( auto [ v , w ] : e[u] ) { // 遍历有权边
}

\end{lstlisting}

\section{Hash}
\subsection{Hash表}
对数字的 hash
\begin{lstlisting}
for( int i = 1 ; i <= n ; i ++ ) b[i] = a[i]; // 复制数组
sort( b + 1 , b + 1 + n ) , m = unique( b + 1 , b + 1 + n ) - b;// 排序去重
for( int i = 1 ; i <= n ; i ++ )//hash
    a[i] = lower_bound( b + 1 , b + 1 + m , a[i] ) - b;
\end{lstlisting}
除此之外,如果更加复杂的 hash 全部使用\verb|unordered_map|容器

\input{数据结构/栈/main.tex}

\input{数据结构/ST表/main.tex}

\input{数据结构/树状数组/main.tex}

\section{分块}
\lstinputlisting{数据结构/分块.cpp}

\section{ODT}
\lstinputlisting{数据结构/ODT.cpp}

\input{数据结构/线段树/main.tex}

\section{差分}
\subection{离散化差分}

\lstinputlisting{数据结构/离散化差分.cpp}

\subsection{二维前缀和、差分}
\lstinputlisting{数据结构/二维差分.cpp}

\input{数据结构/扫描线/扫描线.tex}

\input{数据结构/Splay/main.tex}


\chapter{数据结构}

\section{并查集}

\lstinputlisting{数据结构/并查集.cpp}

\section{链式前向星}
链式前向星又名邻接表,其实现在我已经几乎不会再手写链式前向星而是采用\verb|vector|来代替
\begin{lstlisting}
vector<int> e[N];// 无边权
vector< pair<int,int> > e[N]; 有边权

e[u].push_back(v);// 加边(u,v)
e[u].push_back( { v, w } ); //加有权边 (u,v,w)
// 无向边 反过来再做一次就好

for( auto v : e[u] ){ // 遍历
}
for( auto [ v , w ] : e[u] ) { // 遍历有权边
}

\end{lstlisting}

\section{Hash}
\subsection{Hash表}
对数字的 hash
\begin{lstlisting}
for( int i = 1 ; i <= n ; i ++ ) b[i] = a[i]; // 复制数组
sort( b + 1 , b + 1 + n ) , m = unique( b + 1 , b + 1 + n ) - b;// 排序去重
for( int i = 1 ; i <= n ; i ++ )//hash
    a[i] = lower_bound( b + 1 , b + 1 + m , a[i] ) - b;
\end{lstlisting}
除此之外,如果更加复杂的 hash 全部使用\verb|unordered_map|容器

\input{数据结构/栈/main.tex}

\input{数据结构/ST表/main.tex}

\input{数据结构/树状数组/main.tex}

\section{分块}
\lstinputlisting{数据结构/分块.cpp}

\section{ODT}
\lstinputlisting{数据结构/ODT.cpp}

\input{数据结构/线段树/main.tex}

\section{差分}
\subection{离散化差分}

\lstinputlisting{数据结构/离散化差分.cpp}

\subsection{二维前缀和、差分}
\lstinputlisting{数据结构/二维差分.cpp}

\input{数据结构/扫描线/扫描线.tex}

\input{数据结构/Splay/main.tex}


\section{分块}
\lstinputlisting{数据结构/分块.cpp}

\section{ODT}
\lstinputlisting{数据结构/ODT.cpp}

\chapter{数据结构}

\section{并查集}

\lstinputlisting{数据结构/并查集.cpp}

\section{链式前向星}
链式前向星又名邻接表,其实现在我已经几乎不会再手写链式前向星而是采用\verb|vector|来代替
\begin{lstlisting}
vector<int> e[N];// 无边权
vector< pair<int,int> > e[N]; 有边权

e[u].push_back(v);// 加边(u,v)
e[u].push_back( { v, w } ); //加有权边 (u,v,w)
// 无向边 反过来再做一次就好

for( auto v : e[u] ){ // 遍历
}
for( auto [ v , w ] : e[u] ) { // 遍历有权边
}

\end{lstlisting}

\section{Hash}
\subsection{Hash表}
对数字的 hash
\begin{lstlisting}
for( int i = 1 ; i <= n ; i ++ ) b[i] = a[i]; // 复制数组
sort( b + 1 , b + 1 + n ) , m = unique( b + 1 , b + 1 + n ) - b;// 排序去重
for( int i = 1 ; i <= n ; i ++ )//hash
    a[i] = lower_bound( b + 1 , b + 1 + m , a[i] ) - b;
\end{lstlisting}
除此之外,如果更加复杂的 hash 全部使用\verb|unordered_map|容器

\input{数据结构/栈/main.tex}

\input{数据结构/ST表/main.tex}

\input{数据结构/树状数组/main.tex}

\section{分块}
\lstinputlisting{数据结构/分块.cpp}

\section{ODT}
\lstinputlisting{数据结构/ODT.cpp}

\input{数据结构/线段树/main.tex}

\section{差分}
\subection{离散化差分}

\lstinputlisting{数据结构/离散化差分.cpp}

\subsection{二维前缀和、差分}
\lstinputlisting{数据结构/二维差分.cpp}

\input{数据结构/扫描线/扫描线.tex}

\input{数据结构/Splay/main.tex}


\section{差分}
\subection{离散化差分}

\lstinputlisting{数据结构/离散化差分.cpp}

\subsection{二维前缀和、差分}
\lstinputlisting{数据结构/二维差分.cpp}

\section{扫描线}

\subsection{求面积并}
求 $N$ 个矩形面积的并,每个矩形用$(x_a, y_a),(x_b, y_b)$表示。

将整个图形分为$2N$部分,这样的话每个矩形可以用两条线段$(x_a,y_a,y_b,1),(x_b,y_a,y_b,-1)$表示。

在需要用到扫描线的题目中$y$值通常很大,甚至可能不是整数,所以我们需要进行离散化。记$val(y)$为$y$离散化之后的值,$raw(i)$为$i$的原始坐标。在离散化之后有$tot$个$y$的坐标值,分别对应为$raw(1),raw(2),raw(3),\dots,raw(tot)$,则扫描线被分为$tot-1$段,其中第$i$段为$[raw(i),raw(i+1)]$。

将线段按照$x$值排序,初始每一段都是$0$。然后遍历每个线段$(x_i,y_a,y_b,v)$,如果到$x_{i-1}$的线段覆盖的中长度为$len$,则当前矩形的面积为$(x_i - x_{i-1})\times len$。然后给$[val(y_a),val(y_b)-1]$加$v$,相当于覆盖了$[x_i,x_{i+1}]$的部分。

\lstinputlisting{数据结构/扫描线/luoguP5490.cpp}

\subsection{二维数点}

单纯的二维数点数点问题,可以只用树状数组就可以维护。

$d(x,y)$表示从$(0,0)$到$(x,y)$中点的数量,因此从左下角$(a,b)$到右上角$(c,d)$中点的数量就可以表示为$d(c,d) - d(c,b-1) - d(a-1,d) + d( a-1,b-1)$ , 这个形式就是普通的二维前缀和。我们把式子稍作变形转换为$d((c,d) - d(c,b-1)) - (d(a-1,d) - d(a-1,b-1))$ 这样的话就可以用扫描线优化掉一维。

\lstinputlisting{数据结构/扫描线/luoguP2163.cpp}


\chapter{数据结构}

\section{并查集}

\lstinputlisting{数据结构/并查集.cpp}

\section{链式前向星}
链式前向星又名邻接表,其实现在我已经几乎不会再手写链式前向星而是采用\verb|vector|来代替
\begin{lstlisting}
vector<int> e[N];// 无边权
vector< pair<int,int> > e[N]; 有边权

e[u].push_back(v);// 加边(u,v)
e[u].push_back( { v, w } ); //加有权边 (u,v,w)
// 无向边 反过来再做一次就好

for( auto v : e[u] ){ // 遍历
}
for( auto [ v , w ] : e[u] ) { // 遍历有权边
}

\end{lstlisting}

\section{Hash}
\subsection{Hash表}
对数字的 hash
\begin{lstlisting}
for( int i = 1 ; i <= n ; i ++ ) b[i] = a[i]; // 复制数组
sort( b + 1 , b + 1 + n ) , m = unique( b + 1 , b + 1 + n ) - b;// 排序去重
for( int i = 1 ; i <= n ; i ++ )//hash
    a[i] = lower_bound( b + 1 , b + 1 + m , a[i] ) - b;
\end{lstlisting}
除此之外,如果更加复杂的 hash 全部使用\verb|unordered_map|容器

\input{数据结构/栈/main.tex}

\input{数据结构/ST表/main.tex}

\input{数据结构/树状数组/main.tex}

\section{分块}
\lstinputlisting{数据结构/分块.cpp}

\section{ODT}
\lstinputlisting{数据结构/ODT.cpp}

\input{数据结构/线段树/main.tex}

\section{差分}
\subection{离散化差分}

\lstinputlisting{数据结构/离散化差分.cpp}

\subsection{二维前缀和、差分}
\lstinputlisting{数据结构/二维差分.cpp}

\input{数据结构/扫描线/扫描线.tex}

\input{数据结构/Splay/main.tex}



\section{分块}
\lstinputlisting{数据结构/分块.cpp}

\section{ODT}
\lstinputlisting{数据结构/ODT.cpp}

\chapter{数据结构}

\section{并查集}

\lstinputlisting{数据结构/并查集.cpp}

\section{链式前向星}
链式前向星又名邻接表,其实现在我已经几乎不会再手写链式前向星而是采用\verb|vector|来代替
\begin{lstlisting}
vector<int> e[N];// 无边权
vector< pair<int,int> > e[N]; 有边权

e[u].push_back(v);// 加边(u,v)
e[u].push_back( { v, w } ); //加有权边 (u,v,w)
// 无向边 反过来再做一次就好

for( auto v : e[u] ){ // 遍历
}
for( auto [ v , w ] : e[u] ) { // 遍历有权边
}

\end{lstlisting}

\section{Hash}
\subsection{Hash表}
对数字的 hash
\begin{lstlisting}
for( int i = 1 ; i <= n ; i ++ ) b[i] = a[i]; // 复制数组
sort( b + 1 , b + 1 + n ) , m = unique( b + 1 , b + 1 + n ) - b;// 排序去重
for( int i = 1 ; i <= n ; i ++ )//hash
    a[i] = lower_bound( b + 1 , b + 1 + m , a[i] ) - b;
\end{lstlisting}
除此之外,如果更加复杂的 hash 全部使用\verb|unordered_map|容器

\chapter{数据结构}

\section{并查集}

\lstinputlisting{数据结构/并查集.cpp}

\section{链式前向星}
链式前向星又名邻接表,其实现在我已经几乎不会再手写链式前向星而是采用\verb|vector|来代替
\begin{lstlisting}
vector<int> e[N];// 无边权
vector< pair<int,int> > e[N]; 有边权

e[u].push_back(v);// 加边(u,v)
e[u].push_back( { v, w } ); //加有权边 (u,v,w)
// 无向边 反过来再做一次就好

for( auto v : e[u] ){ // 遍历
}
for( auto [ v , w ] : e[u] ) { // 遍历有权边
}

\end{lstlisting}

\section{Hash}
\subsection{Hash表}
对数字的 hash
\begin{lstlisting}
for( int i = 1 ; i <= n ; i ++ ) b[i] = a[i]; // 复制数组
sort( b + 1 , b + 1 + n ) , m = unique( b + 1 , b + 1 + n ) - b;// 排序去重
for( int i = 1 ; i <= n ; i ++ )//hash
    a[i] = lower_bound( b + 1 , b + 1 + m , a[i] ) - b;
\end{lstlisting}
除此之外,如果更加复杂的 hash 全部使用\verb|unordered_map|容器

\input{数据结构/栈/main.tex}

\input{数据结构/ST表/main.tex}

\input{数据结构/树状数组/main.tex}

\section{分块}
\lstinputlisting{数据结构/分块.cpp}

\section{ODT}
\lstinputlisting{数据结构/ODT.cpp}

\input{数据结构/线段树/main.tex}

\section{差分}
\subection{离散化差分}

\lstinputlisting{数据结构/离散化差分.cpp}

\subsection{二维前缀和、差分}
\lstinputlisting{数据结构/二维差分.cpp}

\input{数据结构/扫描线/扫描线.tex}

\input{数据结构/Splay/main.tex}


\chapter{数据结构}

\section{并查集}

\lstinputlisting{数据结构/并查集.cpp}

\section{链式前向星}
链式前向星又名邻接表,其实现在我已经几乎不会再手写链式前向星而是采用\verb|vector|来代替
\begin{lstlisting}
vector<int> e[N];// 无边权
vector< pair<int,int> > e[N]; 有边权

e[u].push_back(v);// 加边(u,v)
e[u].push_back( { v, w } ); //加有权边 (u,v,w)
// 无向边 反过来再做一次就好

for( auto v : e[u] ){ // 遍历
}
for( auto [ v , w ] : e[u] ) { // 遍历有权边
}

\end{lstlisting}

\section{Hash}
\subsection{Hash表}
对数字的 hash
\begin{lstlisting}
for( int i = 1 ; i <= n ; i ++ ) b[i] = a[i]; // 复制数组
sort( b + 1 , b + 1 + n ) , m = unique( b + 1 , b + 1 + n ) - b;// 排序去重
for( int i = 1 ; i <= n ; i ++ )//hash
    a[i] = lower_bound( b + 1 , b + 1 + m , a[i] ) - b;
\end{lstlisting}
除此之外,如果更加复杂的 hash 全部使用\verb|unordered_map|容器

\input{数据结构/栈/main.tex}

\input{数据结构/ST表/main.tex}

\input{数据结构/树状数组/main.tex}

\section{分块}
\lstinputlisting{数据结构/分块.cpp}

\section{ODT}
\lstinputlisting{数据结构/ODT.cpp}

\input{数据结构/线段树/main.tex}

\section{差分}
\subection{离散化差分}

\lstinputlisting{数据结构/离散化差分.cpp}

\subsection{二维前缀和、差分}
\lstinputlisting{数据结构/二维差分.cpp}

\input{数据结构/扫描线/扫描线.tex}

\input{数据结构/Splay/main.tex}


\chapter{数据结构}

\section{并查集}

\lstinputlisting{数据结构/并查集.cpp}

\section{链式前向星}
链式前向星又名邻接表,其实现在我已经几乎不会再手写链式前向星而是采用\verb|vector|来代替
\begin{lstlisting}
vector<int> e[N];// 无边权
vector< pair<int,int> > e[N]; 有边权

e[u].push_back(v);// 加边(u,v)
e[u].push_back( { v, w } ); //加有权边 (u,v,w)
// 无向边 反过来再做一次就好

for( auto v : e[u] ){ // 遍历
}
for( auto [ v , w ] : e[u] ) { // 遍历有权边
}

\end{lstlisting}

\section{Hash}
\subsection{Hash表}
对数字的 hash
\begin{lstlisting}
for( int i = 1 ; i <= n ; i ++ ) b[i] = a[i]; // 复制数组
sort( b + 1 , b + 1 + n ) , m = unique( b + 1 , b + 1 + n ) - b;// 排序去重
for( int i = 1 ; i <= n ; i ++ )//hash
    a[i] = lower_bound( b + 1 , b + 1 + m , a[i] ) - b;
\end{lstlisting}
除此之外,如果更加复杂的 hash 全部使用\verb|unordered_map|容器

\input{数据结构/栈/main.tex}

\input{数据结构/ST表/main.tex}

\input{数据结构/树状数组/main.tex}

\section{分块}
\lstinputlisting{数据结构/分块.cpp}

\section{ODT}
\lstinputlisting{数据结构/ODT.cpp}

\input{数据结构/线段树/main.tex}

\section{差分}
\subection{离散化差分}

\lstinputlisting{数据结构/离散化差分.cpp}

\subsection{二维前缀和、差分}
\lstinputlisting{数据结构/二维差分.cpp}

\input{数据结构/扫描线/扫描线.tex}

\input{数据结构/Splay/main.tex}


\section{分块}
\lstinputlisting{数据结构/分块.cpp}

\section{ODT}
\lstinputlisting{数据结构/ODT.cpp}

\chapter{数据结构}

\section{并查集}

\lstinputlisting{数据结构/并查集.cpp}

\section{链式前向星}
链式前向星又名邻接表,其实现在我已经几乎不会再手写链式前向星而是采用\verb|vector|来代替
\begin{lstlisting}
vector<int> e[N];// 无边权
vector< pair<int,int> > e[N]; 有边权

e[u].push_back(v);// 加边(u,v)
e[u].push_back( { v, w } ); //加有权边 (u,v,w)
// 无向边 反过来再做一次就好

for( auto v : e[u] ){ // 遍历
}
for( auto [ v , w ] : e[u] ) { // 遍历有权边
}

\end{lstlisting}

\section{Hash}
\subsection{Hash表}
对数字的 hash
\begin{lstlisting}
for( int i = 1 ; i <= n ; i ++ ) b[i] = a[i]; // 复制数组
sort( b + 1 , b + 1 + n ) , m = unique( b + 1 , b + 1 + n ) - b;// 排序去重
for( int i = 1 ; i <= n ; i ++ )//hash
    a[i] = lower_bound( b + 1 , b + 1 + m , a[i] ) - b;
\end{lstlisting}
除此之外,如果更加复杂的 hash 全部使用\verb|unordered_map|容器

\input{数据结构/栈/main.tex}

\input{数据结构/ST表/main.tex}

\input{数据结构/树状数组/main.tex}

\section{分块}
\lstinputlisting{数据结构/分块.cpp}

\section{ODT}
\lstinputlisting{数据结构/ODT.cpp}

\input{数据结构/线段树/main.tex}

\section{差分}
\subection{离散化差分}

\lstinputlisting{数据结构/离散化差分.cpp}

\subsection{二维前缀和、差分}
\lstinputlisting{数据结构/二维差分.cpp}

\input{数据结构/扫描线/扫描线.tex}

\input{数据结构/Splay/main.tex}


\section{差分}
\subection{离散化差分}

\lstinputlisting{数据结构/离散化差分.cpp}

\subsection{二维前缀和、差分}
\lstinputlisting{数据结构/二维差分.cpp}

\section{扫描线}

\subsection{求面积并}
求 $N$ 个矩形面积的并,每个矩形用$(x_a, y_a),(x_b, y_b)$表示。

将整个图形分为$2N$部分,这样的话每个矩形可以用两条线段$(x_a,y_a,y_b,1),(x_b,y_a,y_b,-1)$表示。

在需要用到扫描线的题目中$y$值通常很大,甚至可能不是整数,所以我们需要进行离散化。记$val(y)$为$y$离散化之后的值,$raw(i)$为$i$的原始坐标。在离散化之后有$tot$个$y$的坐标值,分别对应为$raw(1),raw(2),raw(3),\dots,raw(tot)$,则扫描线被分为$tot-1$段,其中第$i$段为$[raw(i),raw(i+1)]$。

将线段按照$x$值排序,初始每一段都是$0$。然后遍历每个线段$(x_i,y_a,y_b,v)$,如果到$x_{i-1}$的线段覆盖的中长度为$len$,则当前矩形的面积为$(x_i - x_{i-1})\times len$。然后给$[val(y_a),val(y_b)-1]$加$v$,相当于覆盖了$[x_i,x_{i+1}]$的部分。

\lstinputlisting{数据结构/扫描线/luoguP5490.cpp}

\subsection{二维数点}

单纯的二维数点数点问题,可以只用树状数组就可以维护。

$d(x,y)$表示从$(0,0)$到$(x,y)$中点的数量,因此从左下角$(a,b)$到右上角$(c,d)$中点的数量就可以表示为$d(c,d) - d(c,b-1) - d(a-1,d) + d( a-1,b-1)$ , 这个形式就是普通的二维前缀和。我们把式子稍作变形转换为$d((c,d) - d(c,b-1)) - (d(a-1,d) - d(a-1,b-1))$ 这样的话就可以用扫描线优化掉一维。

\lstinputlisting{数据结构/扫描线/luoguP2163.cpp}


\chapter{数据结构}

\section{并查集}

\lstinputlisting{数据结构/并查集.cpp}

\section{链式前向星}
链式前向星又名邻接表,其实现在我已经几乎不会再手写链式前向星而是采用\verb|vector|来代替
\begin{lstlisting}
vector<int> e[N];// 无边权
vector< pair<int,int> > e[N]; 有边权

e[u].push_back(v);// 加边(u,v)
e[u].push_back( { v, w } ); //加有权边 (u,v,w)
// 无向边 反过来再做一次就好

for( auto v : e[u] ){ // 遍历
}
for( auto [ v , w ] : e[u] ) { // 遍历有权边
}

\end{lstlisting}

\section{Hash}
\subsection{Hash表}
对数字的 hash
\begin{lstlisting}
for( int i = 1 ; i <= n ; i ++ ) b[i] = a[i]; // 复制数组
sort( b + 1 , b + 1 + n ) , m = unique( b + 1 , b + 1 + n ) - b;// 排序去重
for( int i = 1 ; i <= n ; i ++ )//hash
    a[i] = lower_bound( b + 1 , b + 1 + m , a[i] ) - b;
\end{lstlisting}
除此之外,如果更加复杂的 hash 全部使用\verb|unordered_map|容器

\input{数据结构/栈/main.tex}

\input{数据结构/ST表/main.tex}

\input{数据结构/树状数组/main.tex}

\section{分块}
\lstinputlisting{数据结构/分块.cpp}

\section{ODT}
\lstinputlisting{数据结构/ODT.cpp}

\input{数据结构/线段树/main.tex}

\section{差分}
\subection{离散化差分}

\lstinputlisting{数据结构/离散化差分.cpp}

\subsection{二维前缀和、差分}
\lstinputlisting{数据结构/二维差分.cpp}

\input{数据结构/扫描线/扫描线.tex}

\input{数据结构/Splay/main.tex}



\section{差分}
\subection{离散化差分}

\lstinputlisting{数据结构/离散化差分.cpp}

\subsection{二维前缀和、差分}
\lstinputlisting{数据结构/二维差分.cpp}

\section{扫描线}

\subsection{求面积并}
求 $N$ 个矩形面积的并,每个矩形用$(x_a, y_a),(x_b, y_b)$表示。

将整个图形分为$2N$部分,这样的话每个矩形可以用两条线段$(x_a,y_a,y_b,1),(x_b,y_a,y_b,-1)$表示。

在需要用到扫描线的题目中$y$值通常很大,甚至可能不是整数,所以我们需要进行离散化。记$val(y)$为$y$离散化之后的值,$raw(i)$为$i$的原始坐标。在离散化之后有$tot$个$y$的坐标值,分别对应为$raw(1),raw(2),raw(3),\dots,raw(tot)$,则扫描线被分为$tot-1$段,其中第$i$段为$[raw(i),raw(i+1)]$。

将线段按照$x$值排序,初始每一段都是$0$。然后遍历每个线段$(x_i,y_a,y_b,v)$,如果到$x_{i-1}$的线段覆盖的中长度为$len$,则当前矩形的面积为$(x_i - x_{i-1})\times len$。然后给$[val(y_a),val(y_b)-1]$加$v$,相当于覆盖了$[x_i,x_{i+1}]$的部分。

\lstinputlisting{数据结构/扫描线/luoguP5490.cpp}

\subsection{二维数点}

单纯的二维数点数点问题,可以只用树状数组就可以维护。

$d(x,y)$表示从$(0,0)$到$(x,y)$中点的数量,因此从左下角$(a,b)$到右上角$(c,d)$中点的数量就可以表示为$d(c,d) - d(c,b-1) - d(a-1,d) + d( a-1,b-1)$ , 这个形式就是普通的二维前缀和。我们把式子稍作变形转换为$d((c,d) - d(c,b-1)) - (d(a-1,d) - d(a-1,b-1))$ 这样的话就可以用扫描线优化掉一维。

\lstinputlisting{数据结构/扫描线/luoguP2163.cpp}


\chapter{数据结构}

\section{并查集}

\lstinputlisting{数据结构/并查集.cpp}

\section{链式前向星}
链式前向星又名邻接表,其实现在我已经几乎不会再手写链式前向星而是采用\verb|vector|来代替
\begin{lstlisting}
vector<int> e[N];// 无边权
vector< pair<int,int> > e[N]; 有边权

e[u].push_back(v);// 加边(u,v)
e[u].push_back( { v, w } ); //加有权边 (u,v,w)
// 无向边 反过来再做一次就好

for( auto v : e[u] ){ // 遍历
}
for( auto [ v , w ] : e[u] ) { // 遍历有权边
}

\end{lstlisting}

\section{Hash}
\subsection{Hash表}
对数字的 hash
\begin{lstlisting}
for( int i = 1 ; i <= n ; i ++ ) b[i] = a[i]; // 复制数组
sort( b + 1 , b + 1 + n ) , m = unique( b + 1 , b + 1 + n ) - b;// 排序去重
for( int i = 1 ; i <= n ; i ++ )//hash
    a[i] = lower_bound( b + 1 , b + 1 + m , a[i] ) - b;
\end{lstlisting}
除此之外,如果更加复杂的 hash 全部使用\verb|unordered_map|容器

\chapter{数据结构}

\section{并查集}

\lstinputlisting{数据结构/并查集.cpp}

\section{链式前向星}
链式前向星又名邻接表,其实现在我已经几乎不会再手写链式前向星而是采用\verb|vector|来代替
\begin{lstlisting}
vector<int> e[N];// 无边权
vector< pair<int,int> > e[N]; 有边权

e[u].push_back(v);// 加边(u,v)
e[u].push_back( { v, w } ); //加有权边 (u,v,w)
// 无向边 反过来再做一次就好

for( auto v : e[u] ){ // 遍历
}
for( auto [ v , w ] : e[u] ) { // 遍历有权边
}

\end{lstlisting}

\section{Hash}
\subsection{Hash表}
对数字的 hash
\begin{lstlisting}
for( int i = 1 ; i <= n ; i ++ ) b[i] = a[i]; // 复制数组
sort( b + 1 , b + 1 + n ) , m = unique( b + 1 , b + 1 + n ) - b;// 排序去重
for( int i = 1 ; i <= n ; i ++ )//hash
    a[i] = lower_bound( b + 1 , b + 1 + m , a[i] ) - b;
\end{lstlisting}
除此之外,如果更加复杂的 hash 全部使用\verb|unordered_map|容器

\input{数据结构/栈/main.tex}

\input{数据结构/ST表/main.tex}

\input{数据结构/树状数组/main.tex}

\section{分块}
\lstinputlisting{数据结构/分块.cpp}

\section{ODT}
\lstinputlisting{数据结构/ODT.cpp}

\input{数据结构/线段树/main.tex}

\section{差分}
\subection{离散化差分}

\lstinputlisting{数据结构/离散化差分.cpp}

\subsection{二维前缀和、差分}
\lstinputlisting{数据结构/二维差分.cpp}

\input{数据结构/扫描线/扫描线.tex}

\input{数据结构/Splay/main.tex}


\chapter{数据结构}

\section{并查集}

\lstinputlisting{数据结构/并查集.cpp}

\section{链式前向星}
链式前向星又名邻接表,其实现在我已经几乎不会再手写链式前向星而是采用\verb|vector|来代替
\begin{lstlisting}
vector<int> e[N];// 无边权
vector< pair<int,int> > e[N]; 有边权

e[u].push_back(v);// 加边(u,v)
e[u].push_back( { v, w } ); //加有权边 (u,v,w)
// 无向边 反过来再做一次就好

for( auto v : e[u] ){ // 遍历
}
for( auto [ v , w ] : e[u] ) { // 遍历有权边
}

\end{lstlisting}

\section{Hash}
\subsection{Hash表}
对数字的 hash
\begin{lstlisting}
for( int i = 1 ; i <= n ; i ++ ) b[i] = a[i]; // 复制数组
sort( b + 1 , b + 1 + n ) , m = unique( b + 1 , b + 1 + n ) - b;// 排序去重
for( int i = 1 ; i <= n ; i ++ )//hash
    a[i] = lower_bound( b + 1 , b + 1 + m , a[i] ) - b;
\end{lstlisting}
除此之外,如果更加复杂的 hash 全部使用\verb|unordered_map|容器

\input{数据结构/栈/main.tex}

\input{数据结构/ST表/main.tex}

\input{数据结构/树状数组/main.tex}

\section{分块}
\lstinputlisting{数据结构/分块.cpp}

\section{ODT}
\lstinputlisting{数据结构/ODT.cpp}

\input{数据结构/线段树/main.tex}

\section{差分}
\subection{离散化差分}

\lstinputlisting{数据结构/离散化差分.cpp}

\subsection{二维前缀和、差分}
\lstinputlisting{数据结构/二维差分.cpp}

\input{数据结构/扫描线/扫描线.tex}

\input{数据结构/Splay/main.tex}


\chapter{数据结构}

\section{并查集}

\lstinputlisting{数据结构/并查集.cpp}

\section{链式前向星}
链式前向星又名邻接表,其实现在我已经几乎不会再手写链式前向星而是采用\verb|vector|来代替
\begin{lstlisting}
vector<int> e[N];// 无边权
vector< pair<int,int> > e[N]; 有边权

e[u].push_back(v);// 加边(u,v)
e[u].push_back( { v, w } ); //加有权边 (u,v,w)
// 无向边 反过来再做一次就好

for( auto v : e[u] ){ // 遍历
}
for( auto [ v , w ] : e[u] ) { // 遍历有权边
}

\end{lstlisting}

\section{Hash}
\subsection{Hash表}
对数字的 hash
\begin{lstlisting}
for( int i = 1 ; i <= n ; i ++ ) b[i] = a[i]; // 复制数组
sort( b + 1 , b + 1 + n ) , m = unique( b + 1 , b + 1 + n ) - b;// 排序去重
for( int i = 1 ; i <= n ; i ++ )//hash
    a[i] = lower_bound( b + 1 , b + 1 + m , a[i] ) - b;
\end{lstlisting}
除此之外,如果更加复杂的 hash 全部使用\verb|unordered_map|容器

\input{数据结构/栈/main.tex}

\input{数据结构/ST表/main.tex}

\input{数据结构/树状数组/main.tex}

\section{分块}
\lstinputlisting{数据结构/分块.cpp}

\section{ODT}
\lstinputlisting{数据结构/ODT.cpp}

\input{数据结构/线段树/main.tex}

\section{差分}
\subection{离散化差分}

\lstinputlisting{数据结构/离散化差分.cpp}

\subsection{二维前缀和、差分}
\lstinputlisting{数据结构/二维差分.cpp}

\input{数据结构/扫描线/扫描线.tex}

\input{数据结构/Splay/main.tex}


\section{分块}
\lstinputlisting{数据结构/分块.cpp}

\section{ODT}
\lstinputlisting{数据结构/ODT.cpp}

\chapter{数据结构}

\section{并查集}

\lstinputlisting{数据结构/并查集.cpp}

\section{链式前向星}
链式前向星又名邻接表,其实现在我已经几乎不会再手写链式前向星而是采用\verb|vector|来代替
\begin{lstlisting}
vector<int> e[N];// 无边权
vector< pair<int,int> > e[N]; 有边权

e[u].push_back(v);// 加边(u,v)
e[u].push_back( { v, w } ); //加有权边 (u,v,w)
// 无向边 反过来再做一次就好

for( auto v : e[u] ){ // 遍历
}
for( auto [ v , w ] : e[u] ) { // 遍历有权边
}

\end{lstlisting}

\section{Hash}
\subsection{Hash表}
对数字的 hash
\begin{lstlisting}
for( int i = 1 ; i <= n ; i ++ ) b[i] = a[i]; // 复制数组
sort( b + 1 , b + 1 + n ) , m = unique( b + 1 , b + 1 + n ) - b;// 排序去重
for( int i = 1 ; i <= n ; i ++ )//hash
    a[i] = lower_bound( b + 1 , b + 1 + m , a[i] ) - b;
\end{lstlisting}
除此之外,如果更加复杂的 hash 全部使用\verb|unordered_map|容器

\input{数据结构/栈/main.tex}

\input{数据结构/ST表/main.tex}

\input{数据结构/树状数组/main.tex}

\section{分块}
\lstinputlisting{数据结构/分块.cpp}

\section{ODT}
\lstinputlisting{数据结构/ODT.cpp}

\input{数据结构/线段树/main.tex}

\section{差分}
\subection{离散化差分}

\lstinputlisting{数据结构/离散化差分.cpp}

\subsection{二维前缀和、差分}
\lstinputlisting{数据结构/二维差分.cpp}

\input{数据结构/扫描线/扫描线.tex}

\input{数据结构/Splay/main.tex}


\section{差分}
\subection{离散化差分}

\lstinputlisting{数据结构/离散化差分.cpp}

\subsection{二维前缀和、差分}
\lstinputlisting{数据结构/二维差分.cpp}

\section{扫描线}

\subsection{求面积并}
求 $N$ 个矩形面积的并,每个矩形用$(x_a, y_a),(x_b, y_b)$表示。

将整个图形分为$2N$部分,这样的话每个矩形可以用两条线段$(x_a,y_a,y_b,1),(x_b,y_a,y_b,-1)$表示。

在需要用到扫描线的题目中$y$值通常很大,甚至可能不是整数,所以我们需要进行离散化。记$val(y)$为$y$离散化之后的值,$raw(i)$为$i$的原始坐标。在离散化之后有$tot$个$y$的坐标值,分别对应为$raw(1),raw(2),raw(3),\dots,raw(tot)$,则扫描线被分为$tot-1$段,其中第$i$段为$[raw(i),raw(i+1)]$。

将线段按照$x$值排序,初始每一段都是$0$。然后遍历每个线段$(x_i,y_a,y_b,v)$,如果到$x_{i-1}$的线段覆盖的中长度为$len$,则当前矩形的面积为$(x_i - x_{i-1})\times len$。然后给$[val(y_a),val(y_b)-1]$加$v$,相当于覆盖了$[x_i,x_{i+1}]$的部分。

\lstinputlisting{数据结构/扫描线/luoguP5490.cpp}

\subsection{二维数点}

单纯的二维数点数点问题,可以只用树状数组就可以维护。

$d(x,y)$表示从$(0,0)$到$(x,y)$中点的数量,因此从左下角$(a,b)$到右上角$(c,d)$中点的数量就可以表示为$d(c,d) - d(c,b-1) - d(a-1,d) + d( a-1,b-1)$ , 这个形式就是普通的二维前缀和。我们把式子稍作变形转换为$d((c,d) - d(c,b-1)) - (d(a-1,d) - d(a-1,b-1))$ 这样的话就可以用扫描线优化掉一维。

\lstinputlisting{数据结构/扫描线/luoguP2163.cpp}


\chapter{数据结构}

\section{并查集}

\lstinputlisting{数据结构/并查集.cpp}

\section{链式前向星}
链式前向星又名邻接表,其实现在我已经几乎不会再手写链式前向星而是采用\verb|vector|来代替
\begin{lstlisting}
vector<int> e[N];// 无边权
vector< pair<int,int> > e[N]; 有边权

e[u].push_back(v);// 加边(u,v)
e[u].push_back( { v, w } ); //加有权边 (u,v,w)
// 无向边 反过来再做一次就好

for( auto v : e[u] ){ // 遍历
}
for( auto [ v , w ] : e[u] ) { // 遍历有权边
}

\end{lstlisting}

\section{Hash}
\subsection{Hash表}
对数字的 hash
\begin{lstlisting}
for( int i = 1 ; i <= n ; i ++ ) b[i] = a[i]; // 复制数组
sort( b + 1 , b + 1 + n ) , m = unique( b + 1 , b + 1 + n ) - b;// 排序去重
for( int i = 1 ; i <= n ; i ++ )//hash
    a[i] = lower_bound( b + 1 , b + 1 + m , a[i] ) - b;
\end{lstlisting}
除此之外,如果更加复杂的 hash 全部使用\verb|unordered_map|容器

\input{数据结构/栈/main.tex}

\input{数据结构/ST表/main.tex}

\input{数据结构/树状数组/main.tex}

\section{分块}
\lstinputlisting{数据结构/分块.cpp}

\section{ODT}
\lstinputlisting{数据结构/ODT.cpp}

\input{数据结构/线段树/main.tex}

\section{差分}
\subection{离散化差分}

\lstinputlisting{数据结构/离散化差分.cpp}

\subsection{二维前缀和、差分}
\lstinputlisting{数据结构/二维差分.cpp}

\input{数据结构/扫描线/扫描线.tex}

\input{数据结构/Splay/main.tex}




\section{分块}
\lstinputlisting{数据结构/分块.cpp}

\section{ODT}
\lstinputlisting{数据结构/ODT.cpp}

\chapter{数据结构}

\section{并查集}

\lstinputlisting{数据结构/并查集.cpp}

\section{链式前向星}
链式前向星又名邻接表,其实现在我已经几乎不会再手写链式前向星而是采用\verb|vector|来代替
\begin{lstlisting}
vector<int> e[N];// 无边权
vector< pair<int,int> > e[N]; 有边权

e[u].push_back(v);// 加边(u,v)
e[u].push_back( { v, w } ); //加有权边 (u,v,w)
// 无向边 反过来再做一次就好

for( auto v : e[u] ){ // 遍历
}
for( auto [ v , w ] : e[u] ) { // 遍历有权边
}

\end{lstlisting}

\section{Hash}
\subsection{Hash表}
对数字的 hash
\begin{lstlisting}
for( int i = 1 ; i <= n ; i ++ ) b[i] = a[i]; // 复制数组
sort( b + 1 , b + 1 + n ) , m = unique( b + 1 , b + 1 + n ) - b;// 排序去重
for( int i = 1 ; i <= n ; i ++ )//hash
    a[i] = lower_bound( b + 1 , b + 1 + m , a[i] ) - b;
\end{lstlisting}
除此之外,如果更加复杂的 hash 全部使用\verb|unordered_map|容器

\chapter{数据结构}

\section{并查集}

\lstinputlisting{数据结构/并查集.cpp}

\section{链式前向星}
链式前向星又名邻接表,其实现在我已经几乎不会再手写链式前向星而是采用\verb|vector|来代替
\begin{lstlisting}
vector<int> e[N];// 无边权
vector< pair<int,int> > e[N]; 有边权

e[u].push_back(v);// 加边(u,v)
e[u].push_back( { v, w } ); //加有权边 (u,v,w)
// 无向边 反过来再做一次就好

for( auto v : e[u] ){ // 遍历
}
for( auto [ v , w ] : e[u] ) { // 遍历有权边
}

\end{lstlisting}

\section{Hash}
\subsection{Hash表}
对数字的 hash
\begin{lstlisting}
for( int i = 1 ; i <= n ; i ++ ) b[i] = a[i]; // 复制数组
sort( b + 1 , b + 1 + n ) , m = unique( b + 1 , b + 1 + n ) - b;// 排序去重
for( int i = 1 ; i <= n ; i ++ )//hash
    a[i] = lower_bound( b + 1 , b + 1 + m , a[i] ) - b;
\end{lstlisting}
除此之外,如果更加复杂的 hash 全部使用\verb|unordered_map|容器

\chapter{数据结构}

\section{并查集}

\lstinputlisting{数据结构/并查集.cpp}

\section{链式前向星}
链式前向星又名邻接表,其实现在我已经几乎不会再手写链式前向星而是采用\verb|vector|来代替
\begin{lstlisting}
vector<int> e[N];// 无边权
vector< pair<int,int> > e[N]; 有边权

e[u].push_back(v);// 加边(u,v)
e[u].push_back( { v, w } ); //加有权边 (u,v,w)
// 无向边 反过来再做一次就好

for( auto v : e[u] ){ // 遍历
}
for( auto [ v , w ] : e[u] ) { // 遍历有权边
}

\end{lstlisting}

\section{Hash}
\subsection{Hash表}
对数字的 hash
\begin{lstlisting}
for( int i = 1 ; i <= n ; i ++ ) b[i] = a[i]; // 复制数组
sort( b + 1 , b + 1 + n ) , m = unique( b + 1 , b + 1 + n ) - b;// 排序去重
for( int i = 1 ; i <= n ; i ++ )//hash
    a[i] = lower_bound( b + 1 , b + 1 + m , a[i] ) - b;
\end{lstlisting}
除此之外,如果更加复杂的 hash 全部使用\verb|unordered_map|容器

\input{数据结构/栈/main.tex}

\input{数据结构/ST表/main.tex}

\input{数据结构/树状数组/main.tex}

\section{分块}
\lstinputlisting{数据结构/分块.cpp}

\section{ODT}
\lstinputlisting{数据结构/ODT.cpp}

\input{数据结构/线段树/main.tex}

\section{差分}
\subection{离散化差分}

\lstinputlisting{数据结构/离散化差分.cpp}

\subsection{二维前缀和、差分}
\lstinputlisting{数据结构/二维差分.cpp}

\input{数据结构/扫描线/扫描线.tex}

\input{数据结构/Splay/main.tex}


\chapter{数据结构}

\section{并查集}

\lstinputlisting{数据结构/并查集.cpp}

\section{链式前向星}
链式前向星又名邻接表,其实现在我已经几乎不会再手写链式前向星而是采用\verb|vector|来代替
\begin{lstlisting}
vector<int> e[N];// 无边权
vector< pair<int,int> > e[N]; 有边权

e[u].push_back(v);// 加边(u,v)
e[u].push_back( { v, w } ); //加有权边 (u,v,w)
// 无向边 反过来再做一次就好

for( auto v : e[u] ){ // 遍历
}
for( auto [ v , w ] : e[u] ) { // 遍历有权边
}

\end{lstlisting}

\section{Hash}
\subsection{Hash表}
对数字的 hash
\begin{lstlisting}
for( int i = 1 ; i <= n ; i ++ ) b[i] = a[i]; // 复制数组
sort( b + 1 , b + 1 + n ) , m = unique( b + 1 , b + 1 + n ) - b;// 排序去重
for( int i = 1 ; i <= n ; i ++ )//hash
    a[i] = lower_bound( b + 1 , b + 1 + m , a[i] ) - b;
\end{lstlisting}
除此之外,如果更加复杂的 hash 全部使用\verb|unordered_map|容器

\input{数据结构/栈/main.tex}

\input{数据结构/ST表/main.tex}

\input{数据结构/树状数组/main.tex}

\section{分块}
\lstinputlisting{数据结构/分块.cpp}

\section{ODT}
\lstinputlisting{数据结构/ODT.cpp}

\input{数据结构/线段树/main.tex}

\section{差分}
\subection{离散化差分}

\lstinputlisting{数据结构/离散化差分.cpp}

\subsection{二维前缀和、差分}
\lstinputlisting{数据结构/二维差分.cpp}

\input{数据结构/扫描线/扫描线.tex}

\input{数据结构/Splay/main.tex}


\chapter{数据结构}

\section{并查集}

\lstinputlisting{数据结构/并查集.cpp}

\section{链式前向星}
链式前向星又名邻接表,其实现在我已经几乎不会再手写链式前向星而是采用\verb|vector|来代替
\begin{lstlisting}
vector<int> e[N];// 无边权
vector< pair<int,int> > e[N]; 有边权

e[u].push_back(v);// 加边(u,v)
e[u].push_back( { v, w } ); //加有权边 (u,v,w)
// 无向边 反过来再做一次就好

for( auto v : e[u] ){ // 遍历
}
for( auto [ v , w ] : e[u] ) { // 遍历有权边
}

\end{lstlisting}

\section{Hash}
\subsection{Hash表}
对数字的 hash
\begin{lstlisting}
for( int i = 1 ; i <= n ; i ++ ) b[i] = a[i]; // 复制数组
sort( b + 1 , b + 1 + n ) , m = unique( b + 1 , b + 1 + n ) - b;// 排序去重
for( int i = 1 ; i <= n ; i ++ )//hash
    a[i] = lower_bound( b + 1 , b + 1 + m , a[i] ) - b;
\end{lstlisting}
除此之外,如果更加复杂的 hash 全部使用\verb|unordered_map|容器

\input{数据结构/栈/main.tex}

\input{数据结构/ST表/main.tex}

\input{数据结构/树状数组/main.tex}

\section{分块}
\lstinputlisting{数据结构/分块.cpp}

\section{ODT}
\lstinputlisting{数据结构/ODT.cpp}

\input{数据结构/线段树/main.tex}

\section{差分}
\subection{离散化差分}

\lstinputlisting{数据结构/离散化差分.cpp}

\subsection{二维前缀和、差分}
\lstinputlisting{数据结构/二维差分.cpp}

\input{数据结构/扫描线/扫描线.tex}

\input{数据结构/Splay/main.tex}


\section{分块}
\lstinputlisting{数据结构/分块.cpp}

\section{ODT}
\lstinputlisting{数据结构/ODT.cpp}

\chapter{数据结构}

\section{并查集}

\lstinputlisting{数据结构/并查集.cpp}

\section{链式前向星}
链式前向星又名邻接表,其实现在我已经几乎不会再手写链式前向星而是采用\verb|vector|来代替
\begin{lstlisting}
vector<int> e[N];// 无边权
vector< pair<int,int> > e[N]; 有边权

e[u].push_back(v);// 加边(u,v)
e[u].push_back( { v, w } ); //加有权边 (u,v,w)
// 无向边 反过来再做一次就好

for( auto v : e[u] ){ // 遍历
}
for( auto [ v , w ] : e[u] ) { // 遍历有权边
}

\end{lstlisting}

\section{Hash}
\subsection{Hash表}
对数字的 hash
\begin{lstlisting}
for( int i = 1 ; i <= n ; i ++ ) b[i] = a[i]; // 复制数组
sort( b + 1 , b + 1 + n ) , m = unique( b + 1 , b + 1 + n ) - b;// 排序去重
for( int i = 1 ; i <= n ; i ++ )//hash
    a[i] = lower_bound( b + 1 , b + 1 + m , a[i] ) - b;
\end{lstlisting}
除此之外,如果更加复杂的 hash 全部使用\verb|unordered_map|容器

\input{数据结构/栈/main.tex}

\input{数据结构/ST表/main.tex}

\input{数据结构/树状数组/main.tex}

\section{分块}
\lstinputlisting{数据结构/分块.cpp}

\section{ODT}
\lstinputlisting{数据结构/ODT.cpp}

\input{数据结构/线段树/main.tex}

\section{差分}
\subection{离散化差分}

\lstinputlisting{数据结构/离散化差分.cpp}

\subsection{二维前缀和、差分}
\lstinputlisting{数据结构/二维差分.cpp}

\input{数据结构/扫描线/扫描线.tex}

\input{数据结构/Splay/main.tex}


\section{差分}
\subection{离散化差分}

\lstinputlisting{数据结构/离散化差分.cpp}

\subsection{二维前缀和、差分}
\lstinputlisting{数据结构/二维差分.cpp}

\section{扫描线}

\subsection{求面积并}
求 $N$ 个矩形面积的并,每个矩形用$(x_a, y_a),(x_b, y_b)$表示。

将整个图形分为$2N$部分,这样的话每个矩形可以用两条线段$(x_a,y_a,y_b,1),(x_b,y_a,y_b,-1)$表示。

在需要用到扫描线的题目中$y$值通常很大,甚至可能不是整数,所以我们需要进行离散化。记$val(y)$为$y$离散化之后的值,$raw(i)$为$i$的原始坐标。在离散化之后有$tot$个$y$的坐标值,分别对应为$raw(1),raw(2),raw(3),\dots,raw(tot)$,则扫描线被分为$tot-1$段,其中第$i$段为$[raw(i),raw(i+1)]$。

将线段按照$x$值排序,初始每一段都是$0$。然后遍历每个线段$(x_i,y_a,y_b,v)$,如果到$x_{i-1}$的线段覆盖的中长度为$len$,则当前矩形的面积为$(x_i - x_{i-1})\times len$。然后给$[val(y_a),val(y_b)-1]$加$v$,相当于覆盖了$[x_i,x_{i+1}]$的部分。

\lstinputlisting{数据结构/扫描线/luoguP5490.cpp}

\subsection{二维数点}

单纯的二维数点数点问题,可以只用树状数组就可以维护。

$d(x,y)$表示从$(0,0)$到$(x,y)$中点的数量,因此从左下角$(a,b)$到右上角$(c,d)$中点的数量就可以表示为$d(c,d) - d(c,b-1) - d(a-1,d) + d( a-1,b-1)$ , 这个形式就是普通的二维前缀和。我们把式子稍作变形转换为$d((c,d) - d(c,b-1)) - (d(a-1,d) - d(a-1,b-1))$ 这样的话就可以用扫描线优化掉一维。

\lstinputlisting{数据结构/扫描线/luoguP2163.cpp}


\chapter{数据结构}

\section{并查集}

\lstinputlisting{数据结构/并查集.cpp}

\section{链式前向星}
链式前向星又名邻接表,其实现在我已经几乎不会再手写链式前向星而是采用\verb|vector|来代替
\begin{lstlisting}
vector<int> e[N];// 无边权
vector< pair<int,int> > e[N]; 有边权

e[u].push_back(v);// 加边(u,v)
e[u].push_back( { v, w } ); //加有权边 (u,v,w)
// 无向边 反过来再做一次就好

for( auto v : e[u] ){ // 遍历
}
for( auto [ v , w ] : e[u] ) { // 遍历有权边
}

\end{lstlisting}

\section{Hash}
\subsection{Hash表}
对数字的 hash
\begin{lstlisting}
for( int i = 1 ; i <= n ; i ++ ) b[i] = a[i]; // 复制数组
sort( b + 1 , b + 1 + n ) , m = unique( b + 1 , b + 1 + n ) - b;// 排序去重
for( int i = 1 ; i <= n ; i ++ )//hash
    a[i] = lower_bound( b + 1 , b + 1 + m , a[i] ) - b;
\end{lstlisting}
除此之外,如果更加复杂的 hash 全部使用\verb|unordered_map|容器

\input{数据结构/栈/main.tex}

\input{数据结构/ST表/main.tex}

\input{数据结构/树状数组/main.tex}

\section{分块}
\lstinputlisting{数据结构/分块.cpp}

\section{ODT}
\lstinputlisting{数据结构/ODT.cpp}

\input{数据结构/线段树/main.tex}

\section{差分}
\subection{离散化差分}

\lstinputlisting{数据结构/离散化差分.cpp}

\subsection{二维前缀和、差分}
\lstinputlisting{数据结构/二维差分.cpp}

\input{数据结构/扫描线/扫描线.tex}

\input{数据结构/Splay/main.tex}



\chapter{数据结构}

\section{并查集}

\lstinputlisting{数据结构/并查集.cpp}

\section{链式前向星}
链式前向星又名邻接表,其实现在我已经几乎不会再手写链式前向星而是采用\verb|vector|来代替
\begin{lstlisting}
vector<int> e[N];// 无边权
vector< pair<int,int> > e[N]; 有边权

e[u].push_back(v);// 加边(u,v)
e[u].push_back( { v, w } ); //加有权边 (u,v,w)
// 无向边 反过来再做一次就好

for( auto v : e[u] ){ // 遍历
}
for( auto [ v , w ] : e[u] ) { // 遍历有权边
}

\end{lstlisting}

\section{Hash}
\subsection{Hash表}
对数字的 hash
\begin{lstlisting}
for( int i = 1 ; i <= n ; i ++ ) b[i] = a[i]; // 复制数组
sort( b + 1 , b + 1 + n ) , m = unique( b + 1 , b + 1 + n ) - b;// 排序去重
for( int i = 1 ; i <= n ; i ++ )//hash
    a[i] = lower_bound( b + 1 , b + 1 + m , a[i] ) - b;
\end{lstlisting}
除此之外,如果更加复杂的 hash 全部使用\verb|unordered_map|容器

\chapter{数据结构}

\section{并查集}

\lstinputlisting{数据结构/并查集.cpp}

\section{链式前向星}
链式前向星又名邻接表,其实现在我已经几乎不会再手写链式前向星而是采用\verb|vector|来代替
\begin{lstlisting}
vector<int> e[N];// 无边权
vector< pair<int,int> > e[N]; 有边权

e[u].push_back(v);// 加边(u,v)
e[u].push_back( { v, w } ); //加有权边 (u,v,w)
// 无向边 反过来再做一次就好

for( auto v : e[u] ){ // 遍历
}
for( auto [ v , w ] : e[u] ) { // 遍历有权边
}

\end{lstlisting}

\section{Hash}
\subsection{Hash表}
对数字的 hash
\begin{lstlisting}
for( int i = 1 ; i <= n ; i ++ ) b[i] = a[i]; // 复制数组
sort( b + 1 , b + 1 + n ) , m = unique( b + 1 , b + 1 + n ) - b;// 排序去重
for( int i = 1 ; i <= n ; i ++ )//hash
    a[i] = lower_bound( b + 1 , b + 1 + m , a[i] ) - b;
\end{lstlisting}
除此之外,如果更加复杂的 hash 全部使用\verb|unordered_map|容器

\input{数据结构/栈/main.tex}

\input{数据结构/ST表/main.tex}

\input{数据结构/树状数组/main.tex}

\section{分块}
\lstinputlisting{数据结构/分块.cpp}

\section{ODT}
\lstinputlisting{数据结构/ODT.cpp}

\input{数据结构/线段树/main.tex}

\section{差分}
\subection{离散化差分}

\lstinputlisting{数据结构/离散化差分.cpp}

\subsection{二维前缀和、差分}
\lstinputlisting{数据结构/二维差分.cpp}

\input{数据结构/扫描线/扫描线.tex}

\input{数据结构/Splay/main.tex}


\chapter{数据结构}

\section{并查集}

\lstinputlisting{数据结构/并查集.cpp}

\section{链式前向星}
链式前向星又名邻接表,其实现在我已经几乎不会再手写链式前向星而是采用\verb|vector|来代替
\begin{lstlisting}
vector<int> e[N];// 无边权
vector< pair<int,int> > e[N]; 有边权

e[u].push_back(v);// 加边(u,v)
e[u].push_back( { v, w } ); //加有权边 (u,v,w)
// 无向边 反过来再做一次就好

for( auto v : e[u] ){ // 遍历
}
for( auto [ v , w ] : e[u] ) { // 遍历有权边
}

\end{lstlisting}

\section{Hash}
\subsection{Hash表}
对数字的 hash
\begin{lstlisting}
for( int i = 1 ; i <= n ; i ++ ) b[i] = a[i]; // 复制数组
sort( b + 1 , b + 1 + n ) , m = unique( b + 1 , b + 1 + n ) - b;// 排序去重
for( int i = 1 ; i <= n ; i ++ )//hash
    a[i] = lower_bound( b + 1 , b + 1 + m , a[i] ) - b;
\end{lstlisting}
除此之外,如果更加复杂的 hash 全部使用\verb|unordered_map|容器

\input{数据结构/栈/main.tex}

\input{数据结构/ST表/main.tex}

\input{数据结构/树状数组/main.tex}

\section{分块}
\lstinputlisting{数据结构/分块.cpp}

\section{ODT}
\lstinputlisting{数据结构/ODT.cpp}

\input{数据结构/线段树/main.tex}

\section{差分}
\subection{离散化差分}

\lstinputlisting{数据结构/离散化差分.cpp}

\subsection{二维前缀和、差分}
\lstinputlisting{数据结构/二维差分.cpp}

\input{数据结构/扫描线/扫描线.tex}

\input{数据结构/Splay/main.tex}


\chapter{数据结构}

\section{并查集}

\lstinputlisting{数据结构/并查集.cpp}

\section{链式前向星}
链式前向星又名邻接表,其实现在我已经几乎不会再手写链式前向星而是采用\verb|vector|来代替
\begin{lstlisting}
vector<int> e[N];// 无边权
vector< pair<int,int> > e[N]; 有边权

e[u].push_back(v);// 加边(u,v)
e[u].push_back( { v, w } ); //加有权边 (u,v,w)
// 无向边 反过来再做一次就好

for( auto v : e[u] ){ // 遍历
}
for( auto [ v , w ] : e[u] ) { // 遍历有权边
}

\end{lstlisting}

\section{Hash}
\subsection{Hash表}
对数字的 hash
\begin{lstlisting}
for( int i = 1 ; i <= n ; i ++ ) b[i] = a[i]; // 复制数组
sort( b + 1 , b + 1 + n ) , m = unique( b + 1 , b + 1 + n ) - b;// 排序去重
for( int i = 1 ; i <= n ; i ++ )//hash
    a[i] = lower_bound( b + 1 , b + 1 + m , a[i] ) - b;
\end{lstlisting}
除此之外,如果更加复杂的 hash 全部使用\verb|unordered_map|容器

\input{数据结构/栈/main.tex}

\input{数据结构/ST表/main.tex}

\input{数据结构/树状数组/main.tex}

\section{分块}
\lstinputlisting{数据结构/分块.cpp}

\section{ODT}
\lstinputlisting{数据结构/ODT.cpp}

\input{数据结构/线段树/main.tex}

\section{差分}
\subection{离散化差分}

\lstinputlisting{数据结构/离散化差分.cpp}

\subsection{二维前缀和、差分}
\lstinputlisting{数据结构/二维差分.cpp}

\input{数据结构/扫描线/扫描线.tex}

\input{数据结构/Splay/main.tex}


\section{分块}
\lstinputlisting{数据结构/分块.cpp}

\section{ODT}
\lstinputlisting{数据结构/ODT.cpp}

\chapter{数据结构}

\section{并查集}

\lstinputlisting{数据结构/并查集.cpp}

\section{链式前向星}
链式前向星又名邻接表,其实现在我已经几乎不会再手写链式前向星而是采用\verb|vector|来代替
\begin{lstlisting}
vector<int> e[N];// 无边权
vector< pair<int,int> > e[N]; 有边权

e[u].push_back(v);// 加边(u,v)
e[u].push_back( { v, w } ); //加有权边 (u,v,w)
// 无向边 反过来再做一次就好

for( auto v : e[u] ){ // 遍历
}
for( auto [ v , w ] : e[u] ) { // 遍历有权边
}

\end{lstlisting}

\section{Hash}
\subsection{Hash表}
对数字的 hash
\begin{lstlisting}
for( int i = 1 ; i <= n ; i ++ ) b[i] = a[i]; // 复制数组
sort( b + 1 , b + 1 + n ) , m = unique( b + 1 , b + 1 + n ) - b;// 排序去重
for( int i = 1 ; i <= n ; i ++ )//hash
    a[i] = lower_bound( b + 1 , b + 1 + m , a[i] ) - b;
\end{lstlisting}
除此之外,如果更加复杂的 hash 全部使用\verb|unordered_map|容器

\input{数据结构/栈/main.tex}

\input{数据结构/ST表/main.tex}

\input{数据结构/树状数组/main.tex}

\section{分块}
\lstinputlisting{数据结构/分块.cpp}

\section{ODT}
\lstinputlisting{数据结构/ODT.cpp}

\input{数据结构/线段树/main.tex}

\section{差分}
\subection{离散化差分}

\lstinputlisting{数据结构/离散化差分.cpp}

\subsection{二维前缀和、差分}
\lstinputlisting{数据结构/二维差分.cpp}

\input{数据结构/扫描线/扫描线.tex}

\input{数据结构/Splay/main.tex}


\section{差分}
\subection{离散化差分}

\lstinputlisting{数据结构/离散化差分.cpp}

\subsection{二维前缀和、差分}
\lstinputlisting{数据结构/二维差分.cpp}

\section{扫描线}

\subsection{求面积并}
求 $N$ 个矩形面积的并,每个矩形用$(x_a, y_a),(x_b, y_b)$表示。

将整个图形分为$2N$部分,这样的话每个矩形可以用两条线段$(x_a,y_a,y_b,1),(x_b,y_a,y_b,-1)$表示。

在需要用到扫描线的题目中$y$值通常很大,甚至可能不是整数,所以我们需要进行离散化。记$val(y)$为$y$离散化之后的值,$raw(i)$为$i$的原始坐标。在离散化之后有$tot$个$y$的坐标值,分别对应为$raw(1),raw(2),raw(3),\dots,raw(tot)$,则扫描线被分为$tot-1$段,其中第$i$段为$[raw(i),raw(i+1)]$。

将线段按照$x$值排序,初始每一段都是$0$。然后遍历每个线段$(x_i,y_a,y_b,v)$,如果到$x_{i-1}$的线段覆盖的中长度为$len$,则当前矩形的面积为$(x_i - x_{i-1})\times len$。然后给$[val(y_a),val(y_b)-1]$加$v$,相当于覆盖了$[x_i,x_{i+1}]$的部分。

\lstinputlisting{数据结构/扫描线/luoguP5490.cpp}

\subsection{二维数点}

单纯的二维数点数点问题,可以只用树状数组就可以维护。

$d(x,y)$表示从$(0,0)$到$(x,y)$中点的数量,因此从左下角$(a,b)$到右上角$(c,d)$中点的数量就可以表示为$d(c,d) - d(c,b-1) - d(a-1,d) + d( a-1,b-1)$ , 这个形式就是普通的二维前缀和。我们把式子稍作变形转换为$d((c,d) - d(c,b-1)) - (d(a-1,d) - d(a-1,b-1))$ 这样的话就可以用扫描线优化掉一维。

\lstinputlisting{数据结构/扫描线/luoguP2163.cpp}


\chapter{数据结构}

\section{并查集}

\lstinputlisting{数据结构/并查集.cpp}

\section{链式前向星}
链式前向星又名邻接表,其实现在我已经几乎不会再手写链式前向星而是采用\verb|vector|来代替
\begin{lstlisting}
vector<int> e[N];// 无边权
vector< pair<int,int> > e[N]; 有边权

e[u].push_back(v);// 加边(u,v)
e[u].push_back( { v, w } ); //加有权边 (u,v,w)
// 无向边 反过来再做一次就好

for( auto v : e[u] ){ // 遍历
}
for( auto [ v , w ] : e[u] ) { // 遍历有权边
}

\end{lstlisting}

\section{Hash}
\subsection{Hash表}
对数字的 hash
\begin{lstlisting}
for( int i = 1 ; i <= n ; i ++ ) b[i] = a[i]; // 复制数组
sort( b + 1 , b + 1 + n ) , m = unique( b + 1 , b + 1 + n ) - b;// 排序去重
for( int i = 1 ; i <= n ; i ++ )//hash
    a[i] = lower_bound( b + 1 , b + 1 + m , a[i] ) - b;
\end{lstlisting}
除此之外,如果更加复杂的 hash 全部使用\verb|unordered_map|容器

\input{数据结构/栈/main.tex}

\input{数据结构/ST表/main.tex}

\input{数据结构/树状数组/main.tex}

\section{分块}
\lstinputlisting{数据结构/分块.cpp}

\section{ODT}
\lstinputlisting{数据结构/ODT.cpp}

\input{数据结构/线段树/main.tex}

\section{差分}
\subection{离散化差分}

\lstinputlisting{数据结构/离散化差分.cpp}

\subsection{二维前缀和、差分}
\lstinputlisting{数据结构/二维差分.cpp}

\input{数据结构/扫描线/扫描线.tex}

\input{数据结构/Splay/main.tex}



\chapter{数据结构}

\section{并查集}

\lstinputlisting{数据结构/并查集.cpp}

\section{链式前向星}
链式前向星又名邻接表,其实现在我已经几乎不会再手写链式前向星而是采用\verb|vector|来代替
\begin{lstlisting}
vector<int> e[N];// 无边权
vector< pair<int,int> > e[N]; 有边权

e[u].push_back(v);// 加边(u,v)
e[u].push_back( { v, w } ); //加有权边 (u,v,w)
// 无向边 反过来再做一次就好

for( auto v : e[u] ){ // 遍历
}
for( auto [ v , w ] : e[u] ) { // 遍历有权边
}

\end{lstlisting}

\section{Hash}
\subsection{Hash表}
对数字的 hash
\begin{lstlisting}
for( int i = 1 ; i <= n ; i ++ ) b[i] = a[i]; // 复制数组
sort( b + 1 , b + 1 + n ) , m = unique( b + 1 , b + 1 + n ) - b;// 排序去重
for( int i = 1 ; i <= n ; i ++ )//hash
    a[i] = lower_bound( b + 1 , b + 1 + m , a[i] ) - b;
\end{lstlisting}
除此之外,如果更加复杂的 hash 全部使用\verb|unordered_map|容器

\chapter{数据结构}

\section{并查集}

\lstinputlisting{数据结构/并查集.cpp}

\section{链式前向星}
链式前向星又名邻接表,其实现在我已经几乎不会再手写链式前向星而是采用\verb|vector|来代替
\begin{lstlisting}
vector<int> e[N];// 无边权
vector< pair<int,int> > e[N]; 有边权

e[u].push_back(v);// 加边(u,v)
e[u].push_back( { v, w } ); //加有权边 (u,v,w)
// 无向边 反过来再做一次就好

for( auto v : e[u] ){ // 遍历
}
for( auto [ v , w ] : e[u] ) { // 遍历有权边
}

\end{lstlisting}

\section{Hash}
\subsection{Hash表}
对数字的 hash
\begin{lstlisting}
for( int i = 1 ; i <= n ; i ++ ) b[i] = a[i]; // 复制数组
sort( b + 1 , b + 1 + n ) , m = unique( b + 1 , b + 1 + n ) - b;// 排序去重
for( int i = 1 ; i <= n ; i ++ )//hash
    a[i] = lower_bound( b + 1 , b + 1 + m , a[i] ) - b;
\end{lstlisting}
除此之外,如果更加复杂的 hash 全部使用\verb|unordered_map|容器

\input{数据结构/栈/main.tex}

\input{数据结构/ST表/main.tex}

\input{数据结构/树状数组/main.tex}

\section{分块}
\lstinputlisting{数据结构/分块.cpp}

\section{ODT}
\lstinputlisting{数据结构/ODT.cpp}

\input{数据结构/线段树/main.tex}

\section{差分}
\subection{离散化差分}

\lstinputlisting{数据结构/离散化差分.cpp}

\subsection{二维前缀和、差分}
\lstinputlisting{数据结构/二维差分.cpp}

\input{数据结构/扫描线/扫描线.tex}

\input{数据结构/Splay/main.tex}


\chapter{数据结构}

\section{并查集}

\lstinputlisting{数据结构/并查集.cpp}

\section{链式前向星}
链式前向星又名邻接表,其实现在我已经几乎不会再手写链式前向星而是采用\verb|vector|来代替
\begin{lstlisting}
vector<int> e[N];// 无边权
vector< pair<int,int> > e[N]; 有边权

e[u].push_back(v);// 加边(u,v)
e[u].push_back( { v, w } ); //加有权边 (u,v,w)
// 无向边 反过来再做一次就好

for( auto v : e[u] ){ // 遍历
}
for( auto [ v , w ] : e[u] ) { // 遍历有权边
}

\end{lstlisting}

\section{Hash}
\subsection{Hash表}
对数字的 hash
\begin{lstlisting}
for( int i = 1 ; i <= n ; i ++ ) b[i] = a[i]; // 复制数组
sort( b + 1 , b + 1 + n ) , m = unique( b + 1 , b + 1 + n ) - b;// 排序去重
for( int i = 1 ; i <= n ; i ++ )//hash
    a[i] = lower_bound( b + 1 , b + 1 + m , a[i] ) - b;
\end{lstlisting}
除此之外,如果更加复杂的 hash 全部使用\verb|unordered_map|容器

\input{数据结构/栈/main.tex}

\input{数据结构/ST表/main.tex}

\input{数据结构/树状数组/main.tex}

\section{分块}
\lstinputlisting{数据结构/分块.cpp}

\section{ODT}
\lstinputlisting{数据结构/ODT.cpp}

\input{数据结构/线段树/main.tex}

\section{差分}
\subection{离散化差分}

\lstinputlisting{数据结构/离散化差分.cpp}

\subsection{二维前缀和、差分}
\lstinputlisting{数据结构/二维差分.cpp}

\input{数据结构/扫描线/扫描线.tex}

\input{数据结构/Splay/main.tex}


\chapter{数据结构}

\section{并查集}

\lstinputlisting{数据结构/并查集.cpp}

\section{链式前向星}
链式前向星又名邻接表,其实现在我已经几乎不会再手写链式前向星而是采用\verb|vector|来代替
\begin{lstlisting}
vector<int> e[N];// 无边权
vector< pair<int,int> > e[N]; 有边权

e[u].push_back(v);// 加边(u,v)
e[u].push_back( { v, w } ); //加有权边 (u,v,w)
// 无向边 反过来再做一次就好

for( auto v : e[u] ){ // 遍历
}
for( auto [ v , w ] : e[u] ) { // 遍历有权边
}

\end{lstlisting}

\section{Hash}
\subsection{Hash表}
对数字的 hash
\begin{lstlisting}
for( int i = 1 ; i <= n ; i ++ ) b[i] = a[i]; // 复制数组
sort( b + 1 , b + 1 + n ) , m = unique( b + 1 , b + 1 + n ) - b;// 排序去重
for( int i = 1 ; i <= n ; i ++ )//hash
    a[i] = lower_bound( b + 1 , b + 1 + m , a[i] ) - b;
\end{lstlisting}
除此之外,如果更加复杂的 hash 全部使用\verb|unordered_map|容器

\input{数据结构/栈/main.tex}

\input{数据结构/ST表/main.tex}

\input{数据结构/树状数组/main.tex}

\section{分块}
\lstinputlisting{数据结构/分块.cpp}

\section{ODT}
\lstinputlisting{数据结构/ODT.cpp}

\input{数据结构/线段树/main.tex}

\section{差分}
\subection{离散化差分}

\lstinputlisting{数据结构/离散化差分.cpp}

\subsection{二维前缀和、差分}
\lstinputlisting{数据结构/二维差分.cpp}

\input{数据结构/扫描线/扫描线.tex}

\input{数据结构/Splay/main.tex}


\section{分块}
\lstinputlisting{数据结构/分块.cpp}

\section{ODT}
\lstinputlisting{数据结构/ODT.cpp}

\chapter{数据结构}

\section{并查集}

\lstinputlisting{数据结构/并查集.cpp}

\section{链式前向星}
链式前向星又名邻接表,其实现在我已经几乎不会再手写链式前向星而是采用\verb|vector|来代替
\begin{lstlisting}
vector<int> e[N];// 无边权
vector< pair<int,int> > e[N]; 有边权

e[u].push_back(v);// 加边(u,v)
e[u].push_back( { v, w } ); //加有权边 (u,v,w)
// 无向边 反过来再做一次就好

for( auto v : e[u] ){ // 遍历
}
for( auto [ v , w ] : e[u] ) { // 遍历有权边
}

\end{lstlisting}

\section{Hash}
\subsection{Hash表}
对数字的 hash
\begin{lstlisting}
for( int i = 1 ; i <= n ; i ++ ) b[i] = a[i]; // 复制数组
sort( b + 1 , b + 1 + n ) , m = unique( b + 1 , b + 1 + n ) - b;// 排序去重
for( int i = 1 ; i <= n ; i ++ )//hash
    a[i] = lower_bound( b + 1 , b + 1 + m , a[i] ) - b;
\end{lstlisting}
除此之外,如果更加复杂的 hash 全部使用\verb|unordered_map|容器

\input{数据结构/栈/main.tex}

\input{数据结构/ST表/main.tex}

\input{数据结构/树状数组/main.tex}

\section{分块}
\lstinputlisting{数据结构/分块.cpp}

\section{ODT}
\lstinputlisting{数据结构/ODT.cpp}

\input{数据结构/线段树/main.tex}

\section{差分}
\subection{离散化差分}

\lstinputlisting{数据结构/离散化差分.cpp}

\subsection{二维前缀和、差分}
\lstinputlisting{数据结构/二维差分.cpp}

\input{数据结构/扫描线/扫描线.tex}

\input{数据结构/Splay/main.tex}


\section{差分}
\subection{离散化差分}

\lstinputlisting{数据结构/离散化差分.cpp}

\subsection{二维前缀和、差分}
\lstinputlisting{数据结构/二维差分.cpp}

\section{扫描线}

\subsection{求面积并}
求 $N$ 个矩形面积的并,每个矩形用$(x_a, y_a),(x_b, y_b)$表示。

将整个图形分为$2N$部分,这样的话每个矩形可以用两条线段$(x_a,y_a,y_b,1),(x_b,y_a,y_b,-1)$表示。

在需要用到扫描线的题目中$y$值通常很大,甚至可能不是整数,所以我们需要进行离散化。记$val(y)$为$y$离散化之后的值,$raw(i)$为$i$的原始坐标。在离散化之后有$tot$个$y$的坐标值,分别对应为$raw(1),raw(2),raw(3),\dots,raw(tot)$,则扫描线被分为$tot-1$段,其中第$i$段为$[raw(i),raw(i+1)]$。

将线段按照$x$值排序,初始每一段都是$0$。然后遍历每个线段$(x_i,y_a,y_b,v)$,如果到$x_{i-1}$的线段覆盖的中长度为$len$,则当前矩形的面积为$(x_i - x_{i-1})\times len$。然后给$[val(y_a),val(y_b)-1]$加$v$,相当于覆盖了$[x_i,x_{i+1}]$的部分。

\lstinputlisting{数据结构/扫描线/luoguP5490.cpp}

\subsection{二维数点}

单纯的二维数点数点问题,可以只用树状数组就可以维护。

$d(x,y)$表示从$(0,0)$到$(x,y)$中点的数量,因此从左下角$(a,b)$到右上角$(c,d)$中点的数量就可以表示为$d(c,d) - d(c,b-1) - d(a-1,d) + d( a-1,b-1)$ , 这个形式就是普通的二维前缀和。我们把式子稍作变形转换为$d((c,d) - d(c,b-1)) - (d(a-1,d) - d(a-1,b-1))$ 这样的话就可以用扫描线优化掉一维。

\lstinputlisting{数据结构/扫描线/luoguP2163.cpp}


\chapter{数据结构}

\section{并查集}

\lstinputlisting{数据结构/并查集.cpp}

\section{链式前向星}
链式前向星又名邻接表,其实现在我已经几乎不会再手写链式前向星而是采用\verb|vector|来代替
\begin{lstlisting}
vector<int> e[N];// 无边权
vector< pair<int,int> > e[N]; 有边权

e[u].push_back(v);// 加边(u,v)
e[u].push_back( { v, w } ); //加有权边 (u,v,w)
// 无向边 反过来再做一次就好

for( auto v : e[u] ){ // 遍历
}
for( auto [ v , w ] : e[u] ) { // 遍历有权边
}

\end{lstlisting}

\section{Hash}
\subsection{Hash表}
对数字的 hash
\begin{lstlisting}
for( int i = 1 ; i <= n ; i ++ ) b[i] = a[i]; // 复制数组
sort( b + 1 , b + 1 + n ) , m = unique( b + 1 , b + 1 + n ) - b;// 排序去重
for( int i = 1 ; i <= n ; i ++ )//hash
    a[i] = lower_bound( b + 1 , b + 1 + m , a[i] ) - b;
\end{lstlisting}
除此之外,如果更加复杂的 hash 全部使用\verb|unordered_map|容器

\input{数据结构/栈/main.tex}

\input{数据结构/ST表/main.tex}

\input{数据结构/树状数组/main.tex}

\section{分块}
\lstinputlisting{数据结构/分块.cpp}

\section{ODT}
\lstinputlisting{数据结构/ODT.cpp}

\input{数据结构/线段树/main.tex}

\section{差分}
\subection{离散化差分}

\lstinputlisting{数据结构/离散化差分.cpp}

\subsection{二维前缀和、差分}
\lstinputlisting{数据结构/二维差分.cpp}

\input{数据结构/扫描线/扫描线.tex}

\input{数据结构/Splay/main.tex}



\section{分块}
\lstinputlisting{数据结构/分块.cpp}

\section{ODT}
\lstinputlisting{数据结构/ODT.cpp}

\chapter{数据结构}

\section{并查集}

\lstinputlisting{数据结构/并查集.cpp}

\section{链式前向星}
链式前向星又名邻接表,其实现在我已经几乎不会再手写链式前向星而是采用\verb|vector|来代替
\begin{lstlisting}
vector<int> e[N];// 无边权
vector< pair<int,int> > e[N]; 有边权

e[u].push_back(v);// 加边(u,v)
e[u].push_back( { v, w } ); //加有权边 (u,v,w)
// 无向边 反过来再做一次就好

for( auto v : e[u] ){ // 遍历
}
for( auto [ v , w ] : e[u] ) { // 遍历有权边
}

\end{lstlisting}

\section{Hash}
\subsection{Hash表}
对数字的 hash
\begin{lstlisting}
for( int i = 1 ; i <= n ; i ++ ) b[i] = a[i]; // 复制数组
sort( b + 1 , b + 1 + n ) , m = unique( b + 1 , b + 1 + n ) - b;// 排序去重
for( int i = 1 ; i <= n ; i ++ )//hash
    a[i] = lower_bound( b + 1 , b + 1 + m , a[i] ) - b;
\end{lstlisting}
除此之外,如果更加复杂的 hash 全部使用\verb|unordered_map|容器

\chapter{数据结构}

\section{并查集}

\lstinputlisting{数据结构/并查集.cpp}

\section{链式前向星}
链式前向星又名邻接表,其实现在我已经几乎不会再手写链式前向星而是采用\verb|vector|来代替
\begin{lstlisting}
vector<int> e[N];// 无边权
vector< pair<int,int> > e[N]; 有边权

e[u].push_back(v);// 加边(u,v)
e[u].push_back( { v, w } ); //加有权边 (u,v,w)
// 无向边 反过来再做一次就好

for( auto v : e[u] ){ // 遍历
}
for( auto [ v , w ] : e[u] ) { // 遍历有权边
}

\end{lstlisting}

\section{Hash}
\subsection{Hash表}
对数字的 hash
\begin{lstlisting}
for( int i = 1 ; i <= n ; i ++ ) b[i] = a[i]; // 复制数组
sort( b + 1 , b + 1 + n ) , m = unique( b + 1 , b + 1 + n ) - b;// 排序去重
for( int i = 1 ; i <= n ; i ++ )//hash
    a[i] = lower_bound( b + 1 , b + 1 + m , a[i] ) - b;
\end{lstlisting}
除此之外,如果更加复杂的 hash 全部使用\verb|unordered_map|容器

\input{数据结构/栈/main.tex}

\input{数据结构/ST表/main.tex}

\input{数据结构/树状数组/main.tex}

\section{分块}
\lstinputlisting{数据结构/分块.cpp}

\section{ODT}
\lstinputlisting{数据结构/ODT.cpp}

\input{数据结构/线段树/main.tex}

\section{差分}
\subection{离散化差分}

\lstinputlisting{数据结构/离散化差分.cpp}

\subsection{二维前缀和、差分}
\lstinputlisting{数据结构/二维差分.cpp}

\input{数据结构/扫描线/扫描线.tex}

\input{数据结构/Splay/main.tex}


\chapter{数据结构}

\section{并查集}

\lstinputlisting{数据结构/并查集.cpp}

\section{链式前向星}
链式前向星又名邻接表,其实现在我已经几乎不会再手写链式前向星而是采用\verb|vector|来代替
\begin{lstlisting}
vector<int> e[N];// 无边权
vector< pair<int,int> > e[N]; 有边权

e[u].push_back(v);// 加边(u,v)
e[u].push_back( { v, w } ); //加有权边 (u,v,w)
// 无向边 反过来再做一次就好

for( auto v : e[u] ){ // 遍历
}
for( auto [ v , w ] : e[u] ) { // 遍历有权边
}

\end{lstlisting}

\section{Hash}
\subsection{Hash表}
对数字的 hash
\begin{lstlisting}
for( int i = 1 ; i <= n ; i ++ ) b[i] = a[i]; // 复制数组
sort( b + 1 , b + 1 + n ) , m = unique( b + 1 , b + 1 + n ) - b;// 排序去重
for( int i = 1 ; i <= n ; i ++ )//hash
    a[i] = lower_bound( b + 1 , b + 1 + m , a[i] ) - b;
\end{lstlisting}
除此之外,如果更加复杂的 hash 全部使用\verb|unordered_map|容器

\input{数据结构/栈/main.tex}

\input{数据结构/ST表/main.tex}

\input{数据结构/树状数组/main.tex}

\section{分块}
\lstinputlisting{数据结构/分块.cpp}

\section{ODT}
\lstinputlisting{数据结构/ODT.cpp}

\input{数据结构/线段树/main.tex}

\section{差分}
\subection{离散化差分}

\lstinputlisting{数据结构/离散化差分.cpp}

\subsection{二维前缀和、差分}
\lstinputlisting{数据结构/二维差分.cpp}

\input{数据结构/扫描线/扫描线.tex}

\input{数据结构/Splay/main.tex}


\chapter{数据结构}

\section{并查集}

\lstinputlisting{数据结构/并查集.cpp}

\section{链式前向星}
链式前向星又名邻接表,其实现在我已经几乎不会再手写链式前向星而是采用\verb|vector|来代替
\begin{lstlisting}
vector<int> e[N];// 无边权
vector< pair<int,int> > e[N]; 有边权

e[u].push_back(v);// 加边(u,v)
e[u].push_back( { v, w } ); //加有权边 (u,v,w)
// 无向边 反过来再做一次就好

for( auto v : e[u] ){ // 遍历
}
for( auto [ v , w ] : e[u] ) { // 遍历有权边
}

\end{lstlisting}

\section{Hash}
\subsection{Hash表}
对数字的 hash
\begin{lstlisting}
for( int i = 1 ; i <= n ; i ++ ) b[i] = a[i]; // 复制数组
sort( b + 1 , b + 1 + n ) , m = unique( b + 1 , b + 1 + n ) - b;// 排序去重
for( int i = 1 ; i <= n ; i ++ )//hash
    a[i] = lower_bound( b + 1 , b + 1 + m , a[i] ) - b;
\end{lstlisting}
除此之外,如果更加复杂的 hash 全部使用\verb|unordered_map|容器

\input{数据结构/栈/main.tex}

\input{数据结构/ST表/main.tex}

\input{数据结构/树状数组/main.tex}

\section{分块}
\lstinputlisting{数据结构/分块.cpp}

\section{ODT}
\lstinputlisting{数据结构/ODT.cpp}

\input{数据结构/线段树/main.tex}

\section{差分}
\subection{离散化差分}

\lstinputlisting{数据结构/离散化差分.cpp}

\subsection{二维前缀和、差分}
\lstinputlisting{数据结构/二维差分.cpp}

\input{数据结构/扫描线/扫描线.tex}

\input{数据结构/Splay/main.tex}


\section{分块}
\lstinputlisting{数据结构/分块.cpp}

\section{ODT}
\lstinputlisting{数据结构/ODT.cpp}

\chapter{数据结构}

\section{并查集}

\lstinputlisting{数据结构/并查集.cpp}

\section{链式前向星}
链式前向星又名邻接表,其实现在我已经几乎不会再手写链式前向星而是采用\verb|vector|来代替
\begin{lstlisting}
vector<int> e[N];// 无边权
vector< pair<int,int> > e[N]; 有边权

e[u].push_back(v);// 加边(u,v)
e[u].push_back( { v, w } ); //加有权边 (u,v,w)
// 无向边 反过来再做一次就好

for( auto v : e[u] ){ // 遍历
}
for( auto [ v , w ] : e[u] ) { // 遍历有权边
}

\end{lstlisting}

\section{Hash}
\subsection{Hash表}
对数字的 hash
\begin{lstlisting}
for( int i = 1 ; i <= n ; i ++ ) b[i] = a[i]; // 复制数组
sort( b + 1 , b + 1 + n ) , m = unique( b + 1 , b + 1 + n ) - b;// 排序去重
for( int i = 1 ; i <= n ; i ++ )//hash
    a[i] = lower_bound( b + 1 , b + 1 + m , a[i] ) - b;
\end{lstlisting}
除此之外,如果更加复杂的 hash 全部使用\verb|unordered_map|容器

\input{数据结构/栈/main.tex}

\input{数据结构/ST表/main.tex}

\input{数据结构/树状数组/main.tex}

\section{分块}
\lstinputlisting{数据结构/分块.cpp}

\section{ODT}
\lstinputlisting{数据结构/ODT.cpp}

\input{数据结构/线段树/main.tex}

\section{差分}
\subection{离散化差分}

\lstinputlisting{数据结构/离散化差分.cpp}

\subsection{二维前缀和、差分}
\lstinputlisting{数据结构/二维差分.cpp}

\input{数据结构/扫描线/扫描线.tex}

\input{数据结构/Splay/main.tex}


\section{差分}
\subection{离散化差分}

\lstinputlisting{数据结构/离散化差分.cpp}

\subsection{二维前缀和、差分}
\lstinputlisting{数据结构/二维差分.cpp}

\section{扫描线}

\subsection{求面积并}
求 $N$ 个矩形面积的并,每个矩形用$(x_a, y_a),(x_b, y_b)$表示。

将整个图形分为$2N$部分,这样的话每个矩形可以用两条线段$(x_a,y_a,y_b,1),(x_b,y_a,y_b,-1)$表示。

在需要用到扫描线的题目中$y$值通常很大,甚至可能不是整数,所以我们需要进行离散化。记$val(y)$为$y$离散化之后的值,$raw(i)$为$i$的原始坐标。在离散化之后有$tot$个$y$的坐标值,分别对应为$raw(1),raw(2),raw(3),\dots,raw(tot)$,则扫描线被分为$tot-1$段,其中第$i$段为$[raw(i),raw(i+1)]$。

将线段按照$x$值排序,初始每一段都是$0$。然后遍历每个线段$(x_i,y_a,y_b,v)$,如果到$x_{i-1}$的线段覆盖的中长度为$len$,则当前矩形的面积为$(x_i - x_{i-1})\times len$。然后给$[val(y_a),val(y_b)-1]$加$v$,相当于覆盖了$[x_i,x_{i+1}]$的部分。

\lstinputlisting{数据结构/扫描线/luoguP5490.cpp}

\subsection{二维数点}

单纯的二维数点数点问题,可以只用树状数组就可以维护。

$d(x,y)$表示从$(0,0)$到$(x,y)$中点的数量,因此从左下角$(a,b)$到右上角$(c,d)$中点的数量就可以表示为$d(c,d) - d(c,b-1) - d(a-1,d) + d( a-1,b-1)$ , 这个形式就是普通的二维前缀和。我们把式子稍作变形转换为$d((c,d) - d(c,b-1)) - (d(a-1,d) - d(a-1,b-1))$ 这样的话就可以用扫描线优化掉一维。

\lstinputlisting{数据结构/扫描线/luoguP2163.cpp}


\chapter{数据结构}

\section{并查集}

\lstinputlisting{数据结构/并查集.cpp}

\section{链式前向星}
链式前向星又名邻接表,其实现在我已经几乎不会再手写链式前向星而是采用\verb|vector|来代替
\begin{lstlisting}
vector<int> e[N];// 无边权
vector< pair<int,int> > e[N]; 有边权

e[u].push_back(v);// 加边(u,v)
e[u].push_back( { v, w } ); //加有权边 (u,v,w)
// 无向边 反过来再做一次就好

for( auto v : e[u] ){ // 遍历
}
for( auto [ v , w ] : e[u] ) { // 遍历有权边
}

\end{lstlisting}

\section{Hash}
\subsection{Hash表}
对数字的 hash
\begin{lstlisting}
for( int i = 1 ; i <= n ; i ++ ) b[i] = a[i]; // 复制数组
sort( b + 1 , b + 1 + n ) , m = unique( b + 1 , b + 1 + n ) - b;// 排序去重
for( int i = 1 ; i <= n ; i ++ )//hash
    a[i] = lower_bound( b + 1 , b + 1 + m , a[i] ) - b;
\end{lstlisting}
除此之外,如果更加复杂的 hash 全部使用\verb|unordered_map|容器

\input{数据结构/栈/main.tex}

\input{数据结构/ST表/main.tex}

\input{数据结构/树状数组/main.tex}

\section{分块}
\lstinputlisting{数据结构/分块.cpp}

\section{ODT}
\lstinputlisting{数据结构/ODT.cpp}

\input{数据结构/线段树/main.tex}

\section{差分}
\subection{离散化差分}

\lstinputlisting{数据结构/离散化差分.cpp}

\subsection{二维前缀和、差分}
\lstinputlisting{数据结构/二维差分.cpp}

\input{数据结构/扫描线/扫描线.tex}

\input{数据结构/Splay/main.tex}



\section{差分}
\subection{离散化差分}

\lstinputlisting{数据结构/离散化差分.cpp}

\subsection{二维前缀和、差分}
\lstinputlisting{数据结构/二维差分.cpp}

\section{扫描线}

\subsection{求面积并}
求 $N$ 个矩形面积的并,每个矩形用$(x_a, y_a),(x_b, y_b)$表示。

将整个图形分为$2N$部分,这样的话每个矩形可以用两条线段$(x_a,y_a,y_b,1),(x_b,y_a,y_b,-1)$表示。

在需要用到扫描线的题目中$y$值通常很大,甚至可能不是整数,所以我们需要进行离散化。记$val(y)$为$y$离散化之后的值,$raw(i)$为$i$的原始坐标。在离散化之后有$tot$个$y$的坐标值,分别对应为$raw(1),raw(2),raw(3),\dots,raw(tot)$,则扫描线被分为$tot-1$段,其中第$i$段为$[raw(i),raw(i+1)]$。

将线段按照$x$值排序,初始每一段都是$0$。然后遍历每个线段$(x_i,y_a,y_b,v)$,如果到$x_{i-1}$的线段覆盖的中长度为$len$,则当前矩形的面积为$(x_i - x_{i-1})\times len$。然后给$[val(y_a),val(y_b)-1]$加$v$,相当于覆盖了$[x_i,x_{i+1}]$的部分。

\lstinputlisting{数据结构/扫描线/luoguP5490.cpp}

\subsection{二维数点}

单纯的二维数点数点问题,可以只用树状数组就可以维护。

$d(x,y)$表示从$(0,0)$到$(x,y)$中点的数量,因此从左下角$(a,b)$到右上角$(c,d)$中点的数量就可以表示为$d(c,d) - d(c,b-1) - d(a-1,d) + d( a-1,b-1)$ , 这个形式就是普通的二维前缀和。我们把式子稍作变形转换为$d((c,d) - d(c,b-1)) - (d(a-1,d) - d(a-1,b-1))$ 这样的话就可以用扫描线优化掉一维。

\lstinputlisting{数据结构/扫描线/luoguP2163.cpp}


\chapter{数据结构}

\section{并查集}

\lstinputlisting{数据结构/并查集.cpp}

\section{链式前向星}
链式前向星又名邻接表,其实现在我已经几乎不会再手写链式前向星而是采用\verb|vector|来代替
\begin{lstlisting}
vector<int> e[N];// 无边权
vector< pair<int,int> > e[N]; 有边权

e[u].push_back(v);// 加边(u,v)
e[u].push_back( { v, w } ); //加有权边 (u,v,w)
// 无向边 反过来再做一次就好

for( auto v : e[u] ){ // 遍历
}
for( auto [ v , w ] : e[u] ) { // 遍历有权边
}

\end{lstlisting}

\section{Hash}
\subsection{Hash表}
对数字的 hash
\begin{lstlisting}
for( int i = 1 ; i <= n ; i ++ ) b[i] = a[i]; // 复制数组
sort( b + 1 , b + 1 + n ) , m = unique( b + 1 , b + 1 + n ) - b;// 排序去重
for( int i = 1 ; i <= n ; i ++ )//hash
    a[i] = lower_bound( b + 1 , b + 1 + m , a[i] ) - b;
\end{lstlisting}
除此之外,如果更加复杂的 hash 全部使用\verb|unordered_map|容器

\chapter{数据结构}

\section{并查集}

\lstinputlisting{数据结构/并查集.cpp}

\section{链式前向星}
链式前向星又名邻接表,其实现在我已经几乎不会再手写链式前向星而是采用\verb|vector|来代替
\begin{lstlisting}
vector<int> e[N];// 无边权
vector< pair<int,int> > e[N]; 有边权

e[u].push_back(v);// 加边(u,v)
e[u].push_back( { v, w } ); //加有权边 (u,v,w)
// 无向边 反过来再做一次就好

for( auto v : e[u] ){ // 遍历
}
for( auto [ v , w ] : e[u] ) { // 遍历有权边
}

\end{lstlisting}

\section{Hash}
\subsection{Hash表}
对数字的 hash
\begin{lstlisting}
for( int i = 1 ; i <= n ; i ++ ) b[i] = a[i]; // 复制数组
sort( b + 1 , b + 1 + n ) , m = unique( b + 1 , b + 1 + n ) - b;// 排序去重
for( int i = 1 ; i <= n ; i ++ )//hash
    a[i] = lower_bound( b + 1 , b + 1 + m , a[i] ) - b;
\end{lstlisting}
除此之外,如果更加复杂的 hash 全部使用\verb|unordered_map|容器

\input{数据结构/栈/main.tex}

\input{数据结构/ST表/main.tex}

\input{数据结构/树状数组/main.tex}

\section{分块}
\lstinputlisting{数据结构/分块.cpp}

\section{ODT}
\lstinputlisting{数据结构/ODT.cpp}

\input{数据结构/线段树/main.tex}

\section{差分}
\subection{离散化差分}

\lstinputlisting{数据结构/离散化差分.cpp}

\subsection{二维前缀和、差分}
\lstinputlisting{数据结构/二维差分.cpp}

\input{数据结构/扫描线/扫描线.tex}

\input{数据结构/Splay/main.tex}


\chapter{数据结构}

\section{并查集}

\lstinputlisting{数据结构/并查集.cpp}

\section{链式前向星}
链式前向星又名邻接表,其实现在我已经几乎不会再手写链式前向星而是采用\verb|vector|来代替
\begin{lstlisting}
vector<int> e[N];// 无边权
vector< pair<int,int> > e[N]; 有边权

e[u].push_back(v);// 加边(u,v)
e[u].push_back( { v, w } ); //加有权边 (u,v,w)
// 无向边 反过来再做一次就好

for( auto v : e[u] ){ // 遍历
}
for( auto [ v , w ] : e[u] ) { // 遍历有权边
}

\end{lstlisting}

\section{Hash}
\subsection{Hash表}
对数字的 hash
\begin{lstlisting}
for( int i = 1 ; i <= n ; i ++ ) b[i] = a[i]; // 复制数组
sort( b + 1 , b + 1 + n ) , m = unique( b + 1 , b + 1 + n ) - b;// 排序去重
for( int i = 1 ; i <= n ; i ++ )//hash
    a[i] = lower_bound( b + 1 , b + 1 + m , a[i] ) - b;
\end{lstlisting}
除此之外,如果更加复杂的 hash 全部使用\verb|unordered_map|容器

\input{数据结构/栈/main.tex}

\input{数据结构/ST表/main.tex}

\input{数据结构/树状数组/main.tex}

\section{分块}
\lstinputlisting{数据结构/分块.cpp}

\section{ODT}
\lstinputlisting{数据结构/ODT.cpp}

\input{数据结构/线段树/main.tex}

\section{差分}
\subection{离散化差分}

\lstinputlisting{数据结构/离散化差分.cpp}

\subsection{二维前缀和、差分}
\lstinputlisting{数据结构/二维差分.cpp}

\input{数据结构/扫描线/扫描线.tex}

\input{数据结构/Splay/main.tex}


\chapter{数据结构}

\section{并查集}

\lstinputlisting{数据结构/并查集.cpp}

\section{链式前向星}
链式前向星又名邻接表,其实现在我已经几乎不会再手写链式前向星而是采用\verb|vector|来代替
\begin{lstlisting}
vector<int> e[N];// 无边权
vector< pair<int,int> > e[N]; 有边权

e[u].push_back(v);// 加边(u,v)
e[u].push_back( { v, w } ); //加有权边 (u,v,w)
// 无向边 反过来再做一次就好

for( auto v : e[u] ){ // 遍历
}
for( auto [ v , w ] : e[u] ) { // 遍历有权边
}

\end{lstlisting}

\section{Hash}
\subsection{Hash表}
对数字的 hash
\begin{lstlisting}
for( int i = 1 ; i <= n ; i ++ ) b[i] = a[i]; // 复制数组
sort( b + 1 , b + 1 + n ) , m = unique( b + 1 , b + 1 + n ) - b;// 排序去重
for( int i = 1 ; i <= n ; i ++ )//hash
    a[i] = lower_bound( b + 1 , b + 1 + m , a[i] ) - b;
\end{lstlisting}
除此之外,如果更加复杂的 hash 全部使用\verb|unordered_map|容器

\input{数据结构/栈/main.tex}

\input{数据结构/ST表/main.tex}

\input{数据结构/树状数组/main.tex}

\section{分块}
\lstinputlisting{数据结构/分块.cpp}

\section{ODT}
\lstinputlisting{数据结构/ODT.cpp}

\input{数据结构/线段树/main.tex}

\section{差分}
\subection{离散化差分}

\lstinputlisting{数据结构/离散化差分.cpp}

\subsection{二维前缀和、差分}
\lstinputlisting{数据结构/二维差分.cpp}

\input{数据结构/扫描线/扫描线.tex}

\input{数据结构/Splay/main.tex}


\section{分块}
\lstinputlisting{数据结构/分块.cpp}

\section{ODT}
\lstinputlisting{数据结构/ODT.cpp}

\chapter{数据结构}

\section{并查集}

\lstinputlisting{数据结构/并查集.cpp}

\section{链式前向星}
链式前向星又名邻接表,其实现在我已经几乎不会再手写链式前向星而是采用\verb|vector|来代替
\begin{lstlisting}
vector<int> e[N];// 无边权
vector< pair<int,int> > e[N]; 有边权

e[u].push_back(v);// 加边(u,v)
e[u].push_back( { v, w } ); //加有权边 (u,v,w)
// 无向边 反过来再做一次就好

for( auto v : e[u] ){ // 遍历
}
for( auto [ v , w ] : e[u] ) { // 遍历有权边
}

\end{lstlisting}

\section{Hash}
\subsection{Hash表}
对数字的 hash
\begin{lstlisting}
for( int i = 1 ; i <= n ; i ++ ) b[i] = a[i]; // 复制数组
sort( b + 1 , b + 1 + n ) , m = unique( b + 1 , b + 1 + n ) - b;// 排序去重
for( int i = 1 ; i <= n ; i ++ )//hash
    a[i] = lower_bound( b + 1 , b + 1 + m , a[i] ) - b;
\end{lstlisting}
除此之外,如果更加复杂的 hash 全部使用\verb|unordered_map|容器

\input{数据结构/栈/main.tex}

\input{数据结构/ST表/main.tex}

\input{数据结构/树状数组/main.tex}

\section{分块}
\lstinputlisting{数据结构/分块.cpp}

\section{ODT}
\lstinputlisting{数据结构/ODT.cpp}

\input{数据结构/线段树/main.tex}

\section{差分}
\subection{离散化差分}

\lstinputlisting{数据结构/离散化差分.cpp}

\subsection{二维前缀和、差分}
\lstinputlisting{数据结构/二维差分.cpp}

\input{数据结构/扫描线/扫描线.tex}

\input{数据结构/Splay/main.tex}


\section{差分}
\subection{离散化差分}

\lstinputlisting{数据结构/离散化差分.cpp}

\subsection{二维前缀和、差分}
\lstinputlisting{数据结构/二维差分.cpp}

\section{扫描线}

\subsection{求面积并}
求 $N$ 个矩形面积的并,每个矩形用$(x_a, y_a),(x_b, y_b)$表示。

将整个图形分为$2N$部分,这样的话每个矩形可以用两条线段$(x_a,y_a,y_b,1),(x_b,y_a,y_b,-1)$表示。

在需要用到扫描线的题目中$y$值通常很大,甚至可能不是整数,所以我们需要进行离散化。记$val(y)$为$y$离散化之后的值,$raw(i)$为$i$的原始坐标。在离散化之后有$tot$个$y$的坐标值,分别对应为$raw(1),raw(2),raw(3),\dots,raw(tot)$,则扫描线被分为$tot-1$段,其中第$i$段为$[raw(i),raw(i+1)]$。

将线段按照$x$值排序,初始每一段都是$0$。然后遍历每个线段$(x_i,y_a,y_b,v)$,如果到$x_{i-1}$的线段覆盖的中长度为$len$,则当前矩形的面积为$(x_i - x_{i-1})\times len$。然后给$[val(y_a),val(y_b)-1]$加$v$,相当于覆盖了$[x_i,x_{i+1}]$的部分。

\lstinputlisting{数据结构/扫描线/luoguP5490.cpp}

\subsection{二维数点}

单纯的二维数点数点问题,可以只用树状数组就可以维护。

$d(x,y)$表示从$(0,0)$到$(x,y)$中点的数量,因此从左下角$(a,b)$到右上角$(c,d)$中点的数量就可以表示为$d(c,d) - d(c,b-1) - d(a-1,d) + d( a-1,b-1)$ , 这个形式就是普通的二维前缀和。我们把式子稍作变形转换为$d((c,d) - d(c,b-1)) - (d(a-1,d) - d(a-1,b-1))$ 这样的话就可以用扫描线优化掉一维。

\lstinputlisting{数据结构/扫描线/luoguP2163.cpp}


\chapter{数据结构}

\section{并查集}

\lstinputlisting{数据结构/并查集.cpp}

\section{链式前向星}
链式前向星又名邻接表,其实现在我已经几乎不会再手写链式前向星而是采用\verb|vector|来代替
\begin{lstlisting}
vector<int> e[N];// 无边权
vector< pair<int,int> > e[N]; 有边权

e[u].push_back(v);// 加边(u,v)
e[u].push_back( { v, w } ); //加有权边 (u,v,w)
// 无向边 反过来再做一次就好

for( auto v : e[u] ){ // 遍历
}
for( auto [ v , w ] : e[u] ) { // 遍历有权边
}

\end{lstlisting}

\section{Hash}
\subsection{Hash表}
对数字的 hash
\begin{lstlisting}
for( int i = 1 ; i <= n ; i ++ ) b[i] = a[i]; // 复制数组
sort( b + 1 , b + 1 + n ) , m = unique( b + 1 , b + 1 + n ) - b;// 排序去重
for( int i = 1 ; i <= n ; i ++ )//hash
    a[i] = lower_bound( b + 1 , b + 1 + m , a[i] ) - b;
\end{lstlisting}
除此之外,如果更加复杂的 hash 全部使用\verb|unordered_map|容器

\input{数据结构/栈/main.tex}

\input{数据结构/ST表/main.tex}

\input{数据结构/树状数组/main.tex}

\section{分块}
\lstinputlisting{数据结构/分块.cpp}

\section{ODT}
\lstinputlisting{数据结构/ODT.cpp}

\input{数据结构/线段树/main.tex}

\section{差分}
\subection{离散化差分}

\lstinputlisting{数据结构/离散化差分.cpp}

\subsection{二维前缀和、差分}
\lstinputlisting{数据结构/二维差分.cpp}

\input{数据结构/扫描线/扫描线.tex}

\input{数据结构/Splay/main.tex}




\section{差分}
\subection{离散化差分}

\lstinputlisting{数据结构/离散化差分.cpp}

\subsection{二维前缀和、差分}
\lstinputlisting{数据结构/二维差分.cpp}

