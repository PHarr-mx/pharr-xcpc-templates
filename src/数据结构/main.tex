\chapter{数据结构}

\section{并查集}

\lstinputlisting{数据结构/并查集.cpp}

\section{链式前向星}
链式前向星又名邻接表,其实现在我已经几乎不会再手写链式前向星而是采用\verb|vector|来代替
\begin{lstlisting}
vector<int> e[N];// 无边权
vector< pair<int,int> > e[N]; 有边权

e[u].push_back(v);// 加边(u,v)
e[u].push_back( { v, w } ); //加有权边 (u,v,w)
// 无向边 反过来再做一次就好

for( auto v : e[u] ){ // 遍历
}
for( auto [ v , w ] : e[u] ) { // 遍历有权边
}

\end{lstlisting}

\section{Hash}
\subsection{Hash表}
对数字的 hash
\begin{lstlisting}
for( int i = 1 ; i <= n ; i ++ ) b[i] = a[i]; // 复制数组
sort( b + 1 , b + 1 + n ) , m = unique( b + 1 , b + 1 + n ) - b;// 排序去重
for( int i = 1 ; i <= n ; i ++ )//hash
    a[i] = lower_bound( b + 1 , b + 1 + m , a[i] ) - b;
\end{lstlisting}
除此之外,如果更加复杂的 hash 全部使用\verb|unordered_map|容器
\subsection{字符串 Hash}
把字符串当做一个 p 进制数,如果哈希冲突就双哈希

如果\verb|s = "abc"|那么$H(s) = a\times p^2 + b\times p + c $
\lstinputlisting{数据结构/字符串 Hash.cpp}

\section{拓扑排序}
在一个 DAG 中,将图中的顶点以线性的方式排序,使得对于任何一条 u 到 v 的有向边,u 都可以出现在 v 的前面

\textbf{拓扑序判环} 如果图中已经没有入度为零的点但是依旧有点时,即说明图中存在环。

\textbf{拓扑序判链} 如果求拓扑序的过程中队列中同时存在两个及以上的元素,说明拓扑序不唯一,不是一条链

\textbf{字典序最大、最小的拓扑序} 把 Kahn 中队列换成是大根堆、小根堆实现的优先队列就好
\subsection{DFS算法}
\lstinputlisting{图论/拓扑排序/DFS求拓扑序.cpp}
\subsection{Kahn算法}
\lstinputlisting{图论/拓扑排序/Kahn求拓扑序.cpp}

\section{拓扑排序}
在一个 DAG 中,将图中的顶点以线性的方式排序,使得对于任何一条 u 到 v 的有向边,u 都可以出现在 v 的前面

\textbf{拓扑序判环} 如果图中已经没有入度为零的点但是依旧有点时,即说明图中存在环。

\textbf{拓扑序判链} 如果求拓扑序的过程中队列中同时存在两个及以上的元素,说明拓扑序不唯一,不是一条链

\textbf{字典序最大、最小的拓扑序} 把 Kahn 中队列换成是大根堆、小根堆实现的优先队列就好
\subsection{DFS算法}
\lstinputlisting{图论/拓扑排序/DFS求拓扑序.cpp}
\subsection{Kahn算法}
\lstinputlisting{图论/拓扑排序/Kahn求拓扑序.cpp}

\section{拓扑排序}
在一个 DAG 中,将图中的顶点以线性的方式排序,使得对于任何一条 u 到 v 的有向边,u 都可以出现在 v 的前面

\textbf{拓扑序判环} 如果图中已经没有入度为零的点但是依旧有点时,即说明图中存在环。

\textbf{拓扑序判链} 如果求拓扑序的过程中队列中同时存在两个及以上的元素,说明拓扑序不唯一,不是一条链

\textbf{字典序最大、最小的拓扑序} 把 Kahn 中队列换成是大根堆、小根堆实现的优先队列就好
\subsection{DFS算法}
\lstinputlisting{图论/拓扑排序/DFS求拓扑序.cpp}
\subsection{Kahn算法}
\lstinputlisting{图论/拓扑排序/Kahn求拓扑序.cpp}

\section{分块}
\lstinputlisting{数据结构/分块.cpp}

\input{数据结构/线段树/main.tex

\section{差分}
\subection{离散化差分}
\lstinputlisting{数据结构/离散化差分.cpp}
\subsection{二维前缀和、差分}
\lstinputlisting{数据结构/二维差分.cpp}

