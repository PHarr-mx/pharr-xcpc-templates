\section{扫描线}

\subsection{求面积并}
求 $N$ 个矩形面积的并,每个矩形用$(x_a, y_a),(x_b, y_b)$表示。

将整个图形分为$2N$部分,这样的话每个矩形可以用两条线段$(x_a,y_a,y_b,1),(x_b,y_a,y_b,-1)$表示。

在需要用到扫描线的题目中$y$值通常很大,甚至可能不是整数,所以我们需要进行离散化。记$val(y)$为$y$离散化之后的值,$raw(i)$为$i$的原始坐标。在离散化之后有$tot$个$y$的坐标值,分别对应为$raw(1),raw(2),raw(3),\dots,raw(tot)$,则扫描线被分为$tot-1$段,其中第$i$段为$[raw(i),raw(i+1)]$。

将线段按照$x$值排序,初始每一段都是$0$。然后遍历每个线段$(x_i,y_a,y_b,v)$,如果到$x_{i-1}$的线段覆盖的中长度为$len$,则当前矩形的面积为$(x_i - x_{i-1})\times len$。然后给$[val(y_a),val(y_b)-1]$加$v$,相当于覆盖了$[x_i,x_{i+1}]$的部分。

\lstinputlisting{数据结构/扫描线/luoguP5490.cpp}

\subsection{二维数点}

单纯的二维数点数点问题,可以只用树状数组就可以维护。

$d(x,y)$表示从$(0,0)$到$(x,y)$中点的数量,因此从左下角$(a,b)$到右上角$(c,d)$中点的数量就可以表示为$d(c,d) - d(c,b-1) - d(a-1,d) + d( a-1,b-1)$ , 这个形式就是普通的二维前缀和。我们把式子稍作变形转换为$d((c,d) - d(c,b-1)) - (d(a-1,d) - d(a-1,b-1))$ 这样的话就可以用扫描线优化掉一维。

\lstinputlisting{数据结构/扫描线/luoguP2163.cpp}
