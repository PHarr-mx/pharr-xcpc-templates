\section{Educational DP Contest}

\subsection{A - Frog 1}
\begin{framed}
    有$N$块石头,编号为$1, 2, \ldots, N$。每块$i$($1 \leq i \leq N$),石头$i$的高度为$h_i$。

    有一只青蛙,它最初在石块 $1$ 上。它会重复下面的动作若干次以到达石块$N$:
    \begin{itemize}
        \item 如果青蛙目前在石块$i$上,则跳到石块$i + 1$或石块$i + 2$上。这里需要付出$|h_i - h_j|$的代价,其中$j$是要降落的石块。
    \end{itemize}

    求青蛙到达石块$N$之前可能产生的最小总成本。
\end{framed}
简单的线性dp,$f[i]$为到达$i$的最小的代价,所以转移方程就是:

$f[i]=\min( f[i-1] - |h_i-h_{i-1}|,f[i-2]-|h_i-h_{i-2}|)$
\lstinputlisting{动态规划/Educational DP Contest/A.cpp}


\subsection{B - Frog 2}
\begin{framed}
    有$N$块石头,编号为$1, 2, \ldots, N$。每块$i$($1 \leq i \leq N$),石头$i$的高度为$h_i$。

    有一只青蛙,它最初在石块 $1$ 上。它会重复下面的动作若干次以到达石块$N$:
    \begin{itemize}
        \item 如果青蛙目前在石块$i$上,请跳到以下其中一个位置:石块$i + 1, i + 2, \ldots, i + K$。这里会产生$|h_i - h_j|$的代价,其中$j$是要降落的石头。
    \end{itemize}
    求青蛙到达石块$N$之前可能产生的最小总成本。
\end{framed}
与上一题不同的是,这次转移的前驱很多,但依旧可以直接转移,复杂度$O(NK)$
\lstinputlisting{动态规划/Educational DP Contest/B.cpp}


\subsection{C - Vacation}
\begin{framed}
    有 $N$ 天。每$i$ ($1 \leq i \leq N$)天,第$i$天有三种活动$A,B,C$,只能进行一种活动,每种活动会获得$a_i,b_i,c_i$的快乐值,相邻两天的活动不能相同,请求快乐值之和的最大值。
\end{framed}
$f[i][j]$表示前$i$天,且第$i$天进行活动$j$的最大快乐值。只需要$3\times 3$ 的枚举状态和前驱进行转移即可。
\lstinputlisting{动态规划/Educational DP Contest/C.cpp}


\subsection{D - Knapsack 1}
\begin{framed}
    有 $N$ 个项目,编号为 $1, 2, \ldots, N$。对于每个$i$($1 \leq i \leq N$),项目$i$的权重为$w_i$,值为$v_i$。

    太郎决定从$N$件物品中选择一些装进背包里带回家。背包的容量为 $W$,这意味着所取物品的权重之和最多为 $W$。

    求太郎带回家的物品价值的最大可能和。
\end{framed}
01背包
\lstinputlisting{动态规划/Educational DP Contest/D.cpp}


\subsection{E - Knapsack 2}
\begin{framed}
    有 $N$ 个项目,编号为 $1, 2, \ldots, N$。对于每个$i$($1 \leq i \leq N$),项目$i$的权重为$w_i$,值为$v_i$。

    太郎决定从$N$件物品中选择一些装进背包里带回家。背包的容量为 $W$,这意味着所取物品的权重之和最多为 $W$。

    求太郎带回家的物品价值的最大可能和。
\end{framed}
还是 01 背包,但是本题中$W$范围非常大,无法枚举。这也用到了一个常用的优化思路,考虑$N\times v_i\le 10^5$,所以可以背包求出价值为$i$的最小代价,然后找到合法的最大值即可。
\lstinputlisting{动态规划/Educational DP Contest/E.cpp}


\subsection{F - LCS}
\begin{framed}
    给你字符串 $s$ 和 $t$。请找出一个最长的字符串,它同时是 $s$ 和 $t$ 的子串。
\end{framed}
典题求 LCS 并还原。
\lstinputlisting{动态规划/Educational DP Contest/F.cpp}

\subsection{G - Longest Path}
\begin{framed}
    有一个有向图$G$,它有$N$个顶点和$M$条边。顶点编号为 $1, 2, \ldots, N$,对于每个 $i$ ($1 \leq i \leq M$),$i$条有向边从顶点 $x_i$ 到 $y_i$。$G$不包含有向循环。

    求$G$中最长有向路径的长度。这里,有向路径的长度就是其中边的数量。
\end{framed}
因为不存在有向环,所以最长的路径起点一定入度为 0终点一定出度为 0。想到这个结论后,比较容易想的就是在拓扑序上线性递推即可。
\lstinputlisting{动态规划/Educational DP Contest/G.cpp}

