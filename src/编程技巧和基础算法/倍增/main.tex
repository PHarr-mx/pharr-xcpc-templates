\section{倍增}

\emph{天才 ACM}

给定一个整数 M,对于任意一个整数集合 S,定义“校验值”如下:

从集合 S 中取出 M 对数(即 2×M 个数,不能重复使用集合中的数,如果 S 中的整数不够 M 对,则取到不能取为止),使得“每对数的差的平方”之和最大,这个最大值就称为集合 S 的“校验值”。

现在给定一个长度为 N 的数列 A 以及一个整数 T。 我们要把 A 分成若干段,使得每一段的“校验值”都不超过 T。求最少需要分成几段。
\begin{enumerate}
\item 初始化\verb|p = 1 , r = l = 1|
\item 求出\verb|[l,r+p]|这一段的校验值,若校验值小于等于 T 则\verb|r+=p,p*=2|,否则\verb|p/=2|
\item 重复上一步知道p的值变为0此时的r即为所求
\end{enumerate}

\lstinputlisting[]{编程技巧和基础算法/倍增/天才 ACM.cpp}