\section{MT 19937}

\verb|mt19937|是一个很便捷的随机数生成算法,在 \verb|c++11|中使用非常便捷
\begin{lstlisting}
mt19937 rd(seed) ; // 这样就填入了一个随机数种子
mt19937 rd(random_device{}()); // 一个性能比较好的随机数种子生成
rd(); // 这样就会返回一个随机数
mt19937_64 rd(); // 相同用法,不过返回是一个 64 位整形
\end{lstlisting}

如果要在一个闭区间$[a,b]$中随机生成一个数
\begin{lstlisting}
mt19937 mt{random_device()()};
uniform_int_distribution rd(a,b);
x = rd(mt); // 这样会返回一个在$[a,b]$范围内的随机数
\end{lstlisting}

随机数往往需要计时,计时可以使用\verb|chrono|库实现。
\begin{lstlisting}
#include <iostream>
#include <chrono>
using namespace std;
int main() {
    auto start = chrono::high_resolution_clock::now();

    int n = 1e8;
    while(n --);

    auto end = chrono::high_resolution_clock::now();

    auto duration = chrono::duration_cast<chrono::milliseconds>(end - start);
    cerr << duration.count() << " milliseconds" << endl;
    return 0;
}
\end{lstlisting}