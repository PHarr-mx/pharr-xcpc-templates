\section{Lambda 表达式}
以下内容绝大部分使用与\verb|c++14|及更新的标准

Lambda的组成部分是
\begin{lstlisting}
[capture] (parameters) mutable -> return-type {statement};
\end{lstlisting}
首先\verb|caputre|是捕获列表可以从所在代码块中捕获变量。

什么都不写\verb|[]|就是不进行任何捕获,\verb|[=]|是值捕获,\verb|[&]|是引用捕获,值捕获不能修改变量的值,引用捕获可以。特别的,如果值捕获希望在函数内部修改可以使用\verb|mutable|关键字

同时捕获列表也可以单独针对某一个变量\verb|[a]|、\verb|[&a]|分别是值捕获和引用捕获。 当然也可以混用\verb|[=,&a]|对所有变量值捕获,但\verb|a|除外,\verb|a|是引用捕获。

然后就是\verb|parameters|参数列表和\verb|statement|函数主体,这里与普通的函数没有区别。

\verb|-> return-type|,函数范围值类型,如果不写可以自动推断,但是如果有多个\verb|return|且返回类型不同就会\verb|CE|

\verb|c++14|之后可以用\verb|auto|来自动的把函数赋值给变量,\verb|c++11|中则需要自己写