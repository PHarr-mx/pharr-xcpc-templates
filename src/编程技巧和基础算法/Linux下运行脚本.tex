\section{Linux 下运行脚本}

将以下脚本保存为\verb|run.sh|,如果要编译则 \verb|./run.sh A.cpp|。

如果遇到了权限不足的情况可以 \verb|chmod +x ./run.sh| 或者使用 \verb|sudo|

还有一种使用方法是\verb|bash ./run.sh A|

\begin{lstlisting}
#!/bin/bash
g++ $1.cpp -o $1 -g -O2 -std=c++20 \
-Wall -fsanitize=undefined -fsanitize=address \
&& echo compile_successfully >&2 && ./$1
\end{lstlisting}

如果使用文件输出输出可以使用
\begin{lstlisting}
#!/bin/bash
g++ $1.cpp -o $1 -g -O2 -std=c++20 \
-Wall -fsanitize=undefined -fsanitize=address \
&& echo compile_successfully >&2 && ./$1 < in.txt > out.txt
\end{lstlisting}

以下是一个在Mac OS 下可以使用的脚本
\begin{lstlisting}
#!/bin/zsh
g++-11 $1.cpp -o $1 -g -O2 -std=c++20 \
-Wall -fsanitize=undefined -fsanitize=address \
&& echo compile_successfully >&2 && ./$1
\end{lstlisting}