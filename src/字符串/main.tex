\chapter{字符串}

\section{拓扑排序}
在一个 DAG 中,将图中的顶点以线性的方式排序,使得对于任何一条 u 到 v 的有向边,u 都可以出现在 v 的前面

\textbf{拓扑序判环} 如果图中已经没有入度为零的点但是依旧有点时,即说明图中存在环。

\textbf{拓扑序判链} 如果求拓扑序的过程中队列中同时存在两个及以上的元素,说明拓扑序不唯一,不是一条链

\textbf{字典序最大、最小的拓扑序} 把 Kahn 中队列换成是大根堆、小根堆实现的优先队列就好
\subsection{DFS算法}
\lstinputlisting{图论/拓扑排序/DFS求拓扑序.cpp}
\subsection{Kahn算法}
\lstinputlisting{图论/拓扑排序/Kahn求拓扑序.cpp}

\section{KMP}

首先对字符串首先要求一个前缀函数$\pi[i]$。$\pi[i]$简单来说就是子串$s[0\dots i]$最长的相等的真前缀与真后缀的长度。

\begin{lstlisting}[language = c]
vector<int> prefix_function(const string &s) {
    int n = s.size();
    vector<int> pi(n);
    for (int i = 1, j; i < n; i++) {
        j = pi[i - 1];
        while (j > 0 && s[i] != s[j]) j = pi[j - 1];
        if (s[i] == s[j]) j++;
        pi[i] = j;
    }
    return pi;
}
\end{lstlisting}

然后就是 KMP 算法的实现有两种,两种做法效率实际上一样的

\begin{lstlisting}[language = c]
// pattern 在 text 中出现的位置
vector<int> kmp(const string &text, const string &pattern) {
    string cur = pattern + '#' + text;
    int n = text.size(), m = pattern.size();
    vector<int> v, lps = prefix_function(cur);
    for (int i = m + 1; i <= n + m; i++)
        if (lps[i] == m) v.push_back(i - 2 * m);
    return v;
}

vector<int> kmp(const string &text, const string &pattern) {
    vector<int> v, lps = prefix_function(pattern);
    for (int i = 0, j = 0; i < text.size(); i++) {
        while (j && text[i] != pattern[j]) j = lps[j - 1];
        if (text[i] == pattern[j]) j++;
        if (j == pattern.size())
            v.push_back(i - j + 1), j = lps[j - 1];
    }
    return v;
}
\end{lstlisting}

除了这样做之外,还有一种做法是求不重复的匹配位置

\begin{lstlisting}[language = c]
vector<int> kmp(const string &text, const string &pattern) {
    vector<int> v, lps = prefix_function(pattern);
    for (int i = 0, j = 0; i < text.size(); i++) {
        while (j && text[i] != pattern[j]) j = lps[j - 1];
        if (text[i] == pattern[j]) j++;
        if (j == pattern.size())
            v.push_back(i - j + 1), j = 0;
    }
    return v;
}
\end{lstlisting}

以上的几种写法全部是下标从 0 开始的。下面再给出一种下标从 1 开始的。
\begin{lstlisting}[language = c]
vector<int> prefix_function(const string &s) {
    vector<int> pi(s.size());
    for (int i = 2, j = 0; i < s.size(); i++) {
        while (j > 0 && s[i] != s[j + 1]) j = pi[j];
        if (s[i] == s[j + 1]) j++;
        pi[i] = j;
    }
    return pi;
}

vector<int> kmp(const string &text, const string &pattern) {
    vector<int> v, lps = prefix_function(pattern);
    for (int i = 1, j = 0; i < text.size(); i++) {
        while (j > 0 && text[i] != pattern[j + 1]) j = lps[j];
        if (text[i] == pattern[j + 1]) j++;
        if (j == pattern.size() - 1)
            v.push_back(i - j + 1), j = lps[j];
    }
    return v;
}
\end{lstlisting}


\section{Tire}

\lstinputlisting{字符串/Tire.cpp}


\section{最小表示法}

\subsection{循环同构}
如果字符串$S$选择一个位置$i$满足
$$
S[i...n]+S[1...i-1] = T
$$
则称$S$与$T$循环同构

\subsection{最小表示法}
对于一对字符串$A,B$,他们在原串中的起始位置分别为$i,j$,且前$k$个字符均相同,即
$$
S[i...i+k-1] =S[j...j+k-1]
$$
若$S[i+k]>S[j+k]$,则其实下表$l\in[i,i+k]$的字符串均不可能为最优解,因为如果有$l=i+p$则一定有$j+p$字典序更小,所以可以直接把$i$移动到$i+k+1$进行比较。

\lstinputlisting{字符串/minNotation.cpp}

\subsection{Manacher}

其中 $p_i$ 表示 以位置$i$为中心有$\left \lfloor \frac{p_i}{2} \right \rfloor $个回文串,且最长的长度为$p_i-1$
\lstinputlisting{字符串/Manacher.cpp}
