\section{数论}

\subsection{整除}

定义:若整数 b 除以非零整数 a ,商为整数且余数为零我们就说 b 能被 a 整除,或 a 整除 b 记作$a|b$

\textbf{性质}
\begin{enumerate}
    \item 传递性,若$a|b,b|c$,则$a|c$
    \item 组合性,若$a|b,a|c$则对于任意整数$m,n$均满足$a|mb+nc$
    \item 自反性,对于任意的$n$,均有$n|n$
    \item 对称性,若$a|b,b|a$则$a=b$
\end{enumerate}

\subsection{约数}
\textbf{定义}
若整数$n$除以整数$x$的余数为$0$,即$d$能整除$n$,则称$d$是$n$的约数,$n$是$d$的倍数,记为$d|n$
\emph{算数基本定理}
由算数基本定理得正整数N可以写作$N=p_1^{C_1}\times p_2^{C_2} \times p_3^{C_3} \cdots \times p_m^{C_m}$

\textbf{分解质因数}

分解成 $p_{1}\times _{2}\times p_{3}\times \cdots \ p_{n}$这种形式
\lstinputlisting{数学知识/数论/分解质因数 1.cpp}

分解成 $p_{1}^{k_{1}} \times p_{2}^{k_{2}} \times p_{3}^{k_{3}} \times \cdots \ p_{n}^{k_{n}}$
\lstinputlisting{数学知识/数论/分解质因数 2.cpp}
N的正约数个数为($\Pi$是连乘积的符号,类似$\sum$)

\[
    (c_1+1)\times (c_2+1)\times \cdots (c_m+1)=\Pi_{i=1}^{m}(ci+1)
\]

$N$的所有正约数和为
\[
    (1+p_1+p_1^2+\cdots +p_1^{c_1})\times\cdots\times(1+p_m+p_m^2+\cdots +p_m^{c_m})=\prod_{i=1}^{m}(\sum_{j=0}^{c_i}(p_i)^j)
\]

\subsection{ GCD 和 LCM }

\textbf{性质} $a\times b = gcd(a,b)\times\lcm(a,b)$

通过性质可以得到最小公倍数的求法就是
\begin{lstlisting}[language = c]
    int lcm( int x , int y ){
        return a / gcd( x , y ) * b;
    }
\end{lstlisting}
最大公倍数的求法有\textbf{更相减损术}和\textbf{辗转相除法}

\lstinputlisting{数学知识/数论/GCD.cpp}

一般情况下直接用库函数,库函数的实现是辗转相除法,如果遇到高精度的话(高精度取模分困难)可以用更相减损术来代替

\textbf{定理} 对于斐波那契数列$Feb_i$有$Feb_{gcd(a,b)}=gcd(Feb_a , Feb_b)$

\subsection{质数}
判断质数
\lstinputlisting{数学知识/数论/判断质数.cpp}

米勒-拉宾
\lstinputlisting{数学知识/数论/米勒拉宾.cpp}


埃式筛
\lstinputlisting{数学知识/数论/埃式筛.cpp}
欧拉筛
\lstinputlisting{数学知识/数论/欧拉筛.cpp}

\textbf{证明质数有无限个}

反证法 假设数是$n$个,每个素数是$p_i$,令$P = \Pi_{i=1}^{n} p_i + 1$

因为任何一个数都可以分解成多个质数相乘

所以$P$除以任何一个质数都余$1$ ,显然$P$就也是一个质数 ,与假设矛盾,所以假设错误

所以质数是无限个

\textbf{性质2}

设$\pi(n)$为不超过n的质数个数,则$\pi(n) \approx \frac{n}{\ln{n}}$


\section{拓扑排序}
在一个 DAG 中,将图中的顶点以线性的方式排序,使得对于任何一条 u 到 v 的有向边,u 都可以出现在 v 的前面

\textbf{拓扑序判环} 如果图中已经没有入度为零的点但是依旧有点时,即说明图中存在环。

\textbf{拓扑序判链} 如果求拓扑序的过程中队列中同时存在两个及以上的元素,说明拓扑序不唯一,不是一条链

\textbf{字典序最大、最小的拓扑序} 把 Kahn 中队列换成是大根堆、小根堆实现的优先队列就好
\subsection{DFS算法}
\lstinputlisting{图论/拓扑排序/DFS求拓扑序.cpp}
\subsection{Kahn算法}
\lstinputlisting{图论/拓扑排序/Kahn求拓扑序.cpp}

\section{拓扑排序}
在一个 DAG 中,将图中的顶点以线性的方式排序,使得对于任何一条 u 到 v 的有向边,u 都可以出现在 v 的前面

\textbf{拓扑序判环} 如果图中已经没有入度为零的点但是依旧有点时,即说明图中存在环。

\textbf{拓扑序判链} 如果求拓扑序的过程中队列中同时存在两个及以上的元素,说明拓扑序不唯一,不是一条链

\textbf{字典序最大、最小的拓扑序} 把 Kahn 中队列换成是大根堆、小根堆实现的优先队列就好
\subsection{DFS算法}
\lstinputlisting{图论/拓扑排序/DFS求拓扑序.cpp}
\subsection{Kahn算法}
\lstinputlisting{图论/拓扑排序/Kahn求拓扑序.cpp}
