\subsection{扩展欧几里得}
\textbf{裴蜀定理}

设$a,b$是不全为零的整数,则存在整数$x,y$,使得$ax+by=\gcd(a,b)$
\lstinputlisting{数学知识/数论/扩展欧几里得/exgcd.cpp}

\textbf{丢番图方程}

$ax+by=c$

定义变量$d,x_0,y_0$,调用\verb|d = exgcd(a, b, x0, y0)|。对于方程的特解为
\[
(x=\frac c d x_0 , y = \frac c d y_0)
\]

对于方程的通解为
\[
(x = \frac c d x_0 + k \frac b d , y = \frac c d y_0 + k \frac a d ) ,k \in Z
\]

\textbf{线性同余方程}

$a\times x\equiv b(\mod m)$

线性同余方程等价于$a\times x - b$是$m$的倍数,设为$-y$倍,方程可改写为丢番图方程$a \times x + m\times y=b$

线性同余方程有解的充要条件$\gcd(a,m)|b$

在有解时用扩偶求得$x_0,y_0$满足$a\times x_0+m\times y_0=\gcd(a,m)$,则方程的特解$x=x_0\times \frac b{\gcd(a,m)}$

通解是$x=x_0\times \frac b{\gcd(a,m)} + k\times \frac{m}{\gcd(a,m)},k\in \Z$

\lstinputlisting{数学知识/数论/扩展欧几里得/calc.cpp}

\textbf{扩展欧几里得求逆元}

本质上是解同余方程$a\times a^{-1} \equiv 1 (\mod m)$

\lstinputlisting{数学知识/数论/扩展欧几里得/inv.cpp}

\textbf{线性同余方程组(中国剩余定理)}

设$m_1,m_2,\dots,m_n$是两两互质的整数。$m=\Pi_{i=1}^nm_i,M_i=\frac m m_i$,$t_i$是线性同余方程组$M_it_i\equiv 1(\mod m)$的一个解,对于任意的$n$个整数$a_1,a_2,\dots,a_n$,方程组
\[
\left\{\begin{matrix}
           x\equiv a_1(\mod m_1)\\
           x\equiv a_2(\mod m_2)\\
           \vdots \\
           x\equiv a_n(\mod m_n)
\end{matrix}\right.
\]


有整数解,解为$x=\sum_{i=1}^na_iM_it_i(\mod m)$

\lstinputlisting{数学知识/数论/扩展欧几里得/CRT.cpp}

\textbf{高次同余方程(BSGS)}

$a^x\equiv b (\mod p)$,要求$a,p$互质。求非负整数$x$。复杂度$O(\sqrt p)$
\lstinputlisting{数学知识/数论/扩展欧几里得/BSGS.cpp}