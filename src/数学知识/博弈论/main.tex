\section{博弈论}

\subsection{Introduction}

\textbf{组合游戏(Combinatorial Game)}

两个玩家,一个状态集合,游戏的规则是指明玩家在一个状态下可以移动的到那些其它状态。玩家轮流进行移动,如果出于某种状态下,玩家无法根据规则移动,则游戏结束。

\textbf{P 态和 N 态}

P 态:走到这个状态的玩家(Previous Player)获胜

N 态:从这个装填出发的玩家(Next Player)获胜

性质:至少能走到一个 P 态的状态是 N 态,只能走到 N 态的状态是 P 态

\textbf{正常规则与反常规则}

正常规则:终态是 P 态(无法走的人输)

反常规则:终态是 N 态(无法走的人赢)

\textbf{平等游戏(Impartial Game)} :规则对两个玩家是一样的

\textbf{不平等游戏(Partizan Game)}:规则对两个玩家是不一样的

在无向图上,平等游戏两个玩家的边集是一样的,不平等游戏两个玩家的边集不同。

\subsection{Sprague-Grundy}

\textbf{图游戏} 给一个有向图$G=(V,E)$,其中$V$是非空的结点集,$E$是有向边集。两个人在有向图$G$上进行游戏。从起点$x_0$出发,轮流移动。每一轮,玩家可以移动一步,把$x$可以一步移动到点集记为$F(x)$。如果$F(x)$为空集,游戏结束。

\textbf{SG函数} 定义有向无环图$G$,则$sg(x) = \mathrm{mex}\{ sg(y) | y \in F(x) \}$

正常情况下
\begin{itemize}
    \item $x$是$P$态当且仅当$sg(x) = 0$
    \item $x$是$N$态当且仅当$sg(x) > 0$
\end{itemize}

\textbf{组合游戏的和} 给定多个组合游戏,每个游戏都有一个初始状态,每次当前玩家选择一个组合游戏,并按照该游戏的规则移动一次。称这个新的组合游戏为这些\textbf{组合游戏的和}。

\textbf{SG定理} $n$个组合游戏的和$G=G_1+G_2+\dots + C_n$这个游戏的$SG$函数为$sg(x) = sg_1(x_1) \oplus sg_2(x_2) \oplus \dots \oplus sg_n(x_n)$

\textbf{Nim Game}

$n$堆石子,每次可以从一堆里面任意取任意个石子。一堆石子,SG 函数就是石子数,整个游戏的 SG 函数就是每一堆石子数的异或和。

先手必胜:SG 不为 0。先手必败:SG 为 0。

\textbf{Bash Game}

每次最多取$m$个石子,其同 Nim。一堆石子的 SG 函数为石子数 $\mod(m + 1)$。

先手必胜:SG 不为 0。先手必败:SG 为 0。

\textbf{Nim-k Game}

每次最多可以同时对$k$堆石子操作,$k$堆石子每堆可以取不用数量石子。一堆石子的 SG 函数为石子数。整个游戏的 SG 函数需要对每一个二进制位计算,SG 函数的一个二进制位的值为所有石子 SG 函数当前二进制位为 1 的个数 $\mod(k + 1)$。

先手必胜:SG 不为 0。先手必败:SG 为 0。

\textbf{Anti-Nim Game}

不能取石子的一方获胜。

先手必胜:SG 不为 0,且至少有一堆石子数大于 1。或 SG 为 0 且每一堆石子数都不超过 1。

先手必败:其余情况。

\textbf{Staircase Game}

阶梯博弈,每次可以从上一个阶梯上拿任意数量石子放在下一层,不能操作输。

SG 函数为奇数层阶梯上石子的异或和,如果石子从偶数层移动到奇数层,对手一定可以继续移动至偶数层使得 SG 函数不变。

先手必胜:SG 不为 0。先手必败:SG 为 0。

